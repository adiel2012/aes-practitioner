\chapter{Common Exam Scenarios and Real-World Solutions}

\section{Scenario-Based Learning}

\subsection{Scenario 1: Cost Optimization for Predictable Workloads}

\textbf{Situation}: A company runs a web application on EC2 instances that experiences predictable traffic Monday-Friday 9 AM-5 PM EST. Traffic is minimal on weekends and nights.

\textbf{Current Setup}:
\begin{itemize}
  \item 10 m5.large instances running 24/7
  \item On-Demand pricing
  \item Monthly cost: \$1,200
\end{itemize}

\textbf{Question}: What's the MOST cost-effective solution?

\textbf{Analysis}:
\begin{itemize}
  \item Predictable schedule = opportunity for optimization
  \item Not running 24/7 = On-Demand might be wasteful
  \item Regular business hours = scheduled scaling
  \item Baseline capacity needed = Reserved Instances candidate
\end{itemize}

\textbf{Recommended Solution}:
\begin{enumerate}
  \item Purchase \textbf{3-year Standard Reserved Instances} for 2-3 instances (baseline capacity)
  \begin{itemize}
    \item Savings: Up to 75\% on these instances
  \end{itemize}

  \item Configure \textbf{EC2 Auto Scaling with scheduled actions}:
  \begin{itemize}
    \item Scale up Monday-Friday 8:30 AM EST (before traffic starts)
    \item Scale down at 5:30 PM EST (after traffic ends)
    \item Minimum capacity on weekends: 2-3 instances
  \end{itemize}

  \item Use \textbf{On-Demand for peak periods} during business hours

  \item Store session data in \textbf{ElastiCache or DynamoDB} (not on instances)
\end{enumerate}

\textbf{Expected Savings}: 40-60\% reduction in monthly costs

\subsection{Scenario 2: Designing for High Availability}

\textbf{Situation}: An e-commerce company's application must remain available even if an entire Availability Zone fails. The application currently runs on a single EC2 instance with a MySQL database.

\textbf{Question}: How should you architect this for high availability?

\textbf{Current Problems}:
\begin{itemize}
  \item Single point of failure (one EC2 instance)
  \item Database not redundant
  \item No automatic failover
  \item Session data tied to instance
\end{itemize}

\textbf{Recommended Solution}:

\begin{enumerate}
  \item \textbf{Multi-AZ Application Tier}:
  \begin{itemize}
    \item Deploy \textbf{Application Load Balancer} spanning multiple AZs
    \item Create \textbf{Auto Scaling group} with minimum 2 instances across different AZs
    \item Set desired capacity based on traffic patterns
    \item Configure health checks on ALB and Auto Scaling
  \end{itemize}

  \item \textbf{Multi-AZ Database}:
  \begin{itemize}
    \item Migrate MySQL to \textbf{Amazon RDS Multi-AZ}
    \item Automatic failover to standby in different AZ
    \item Synchronous replication
    \item Minimal downtime during failover
  \end{itemize}

  \item \textbf{Stateless Application Design}:
  \begin{itemize}
    \item Store session data in \textbf{ElastiCache} (Redis with Multi-AZ)
    \item Or use \textbf{DynamoDB} for session storage
    \item Enable sticky sessions on ALB if needed (but prefer stateless)
  \end{itemize}

  \item \textbf{Static Assets}:
  \begin{itemize}
    \item Store in \textbf{S3} (automatically multi-AZ)
    \item Use \textbf{CloudFront} for global distribution
  \end{itemize}

  \item \textbf{Monitoring}:
  \begin{itemize}
    \item Set up \textbf{CloudWatch alarms} for health checks
    \item Configure \textbf{SNS notifications} for failures
  \end{itemize}
\end{enumerate}

\textbf{Architecture Benefits}:
\begin{itemize}
  \item Survives AZ failure
  \item Automatic scaling for traffic spikes
  \item Automatic failover for database
  \item No single point of failure
\end{itemize}

\subsection{Scenario 3: Large Data Migration}

\textbf{Situation}: A healthcare company needs to migrate 80 TB of medical imaging data from on-premises storage to S3. Compliance requires data to be encrypted and migration completed within 2 weeks.

\textbf{Constraints}:
\begin{itemize}
  \item Internet connection: 100 Mbps
  \item Upload via internet would take: ~74 days
  \item Deadline: 2 weeks
  \item Data must be encrypted
  \item HIPAA compliance required
\end{itemize}

\textbf{Question}: What's the best migration approach?

\textbf{Analysis}:
\begin{itemize}
  \item Data volume too large for internet upload
  \item Time constraint eliminates internet-based solutions
  \item Security and compliance requirements
  \item Need physical device for transfer
\end{itemize}

\textbf{Recommended Solution}:

\begin{enumerate}
  \item Use \textbf{AWS Snowball Edge Storage Optimized}:
  \begin{itemize}
    \item 80 TB usable capacity per device
    \item Order 1-2 devices (for redundancy)
    \item 256-bit encryption built-in
    \item HIPAA compliant
  \end{itemize}

  \item \textbf{Migration Process}:
  \begin{enumerate}
    \item Order Snowball device via AWS Console
    \item AWS ships device (2-3 days)
    \item Connect to network, unlock with credentials
    \item Copy data using Snowball client (2-4 days for 80 TB)
    \item Ship device back to AWS (2-3 days)
    \item AWS uploads to S3 (1-2 days)
  \end{enumerate}

  \item \textbf{S3 Configuration}:
  \begin{itemize}
    \item Enable \textbf{S3 server-side encryption (SSE-S3 or SSE-KMS)}
    \item Enable \textbf{versioning} for data protection
    \item Configure \textbf{lifecycle policies} to transition older data to Glacier
    \item Enable \textbf{S3 Object Lock} for compliance (WORM)
  \end{itemize}

  \item \textbf{Compliance}:
  \begin{itemize}
    \item Use \textbf{AWS Artifact} to access HIPAA BAA
    \item Sign Business Associate Addendum (BAA)
    \item Enable \textbf{CloudTrail} for audit logging
    \item Use \textbf{AWS Config} for compliance monitoring
  \end{itemize}
\end{enumerate}

\textbf{Timeline}: 7-12 days (meets 2-week deadline)

\textbf{Alternative for >100 PB}: Use \textbf{AWS Snowmobile}

\subsection{Scenario 4: Serverless Application Architecture}

\textbf{Situation}: A startup wants to build a mobile app backend with REST API. They have limited DevOps resources and want to minimize operational overhead while paying only for actual usage.

\textbf{Requirements}:
\begin{itemize}
  \item REST API for mobile app
  \item User authentication
  \item Data storage
  \item Image storage
  \item Scalable to millions of users
  \item Minimal operational management
  \item Pay-per-use pricing
\end{itemize}

\textbf{Question}: What AWS services should they use?

\textbf{Recommended Serverless Architecture}:

\begin{enumerate}
  \item \textbf{API Layer}:
  \begin{itemize}
    \item \textbf{Amazon API Gateway}: Create and manage REST API
    \item Features: Request throttling, API keys, caching, CORS
    \item Pay per million API calls
  \end{itemize}

  \item \textbf{Compute Layer}:
  \begin{itemize}
    \item \textbf{AWS Lambda}: Run business logic without servers
    \item Languages: Node.js, Python, Java, Go, etc.
    \item Auto-scaling built-in
    \item Pay only for execution time
  \end{itemize}

  \item \textbf{Authentication}:
  \begin{itemize}
    \item \textbf{Amazon Cognito}: User sign-up, sign-in, access control
    \item User pools for authentication
    \item Identity pools for AWS resource access
    \item Social identity providers (Facebook, Google)
    \item Free tier: 50,000 MAUs
  \end{itemize}

  \item \textbf{Data Storage}:
  \begin{itemize}
    \item \textbf{Amazon DynamoDB}: NoSQL database
    \item Single-digit millisecond latency
    \item Automatic scaling
    \item On-demand or provisioned capacity
    \item Always-free tier: 25 GB storage
  \end{itemize}

  \item \textbf{Image Storage}:
  \begin{itemize}
    \item \textbf{Amazon S3}: Store user-uploaded images
    \item Lifecycle policies to move old images to Glacier
    \item CloudFront for fast image delivery
  \end{itemize}

  \item \textbf{Optional Enhancements}:
  \begin{itemize}
    \item \textbf{Amazon CloudFront}: CDN for API and static assets
    \item \textbf{AWS AppSync}: GraphQL API (alternative to API Gateway + Lambda)
    \item \textbf{Amazon SES}: Send transactional emails
    \item \textbf{Amazon SNS}: Push notifications to mobile devices
  \end{itemize}
\end{enumerate}

\textbf{Benefits}:
\begin{itemize}
  \item Zero server management
  \item Automatic scaling from 0 to millions of users
  \item Pay only for actual usage
  \item High availability built-in
  \item Focus on application code, not infrastructure
  \item Fast deployment and iteration
\end{itemize}

\textbf{Cost Example}:
\begin{itemize}
  \item 1 million API requests: ~\$3.50
  \item Lambda executions: ~\$0.20
  \item DynamoDB: ~\$1.25
  \item S3 storage (100 GB): ~\$2.30
  \item Total: ~\$7.25/month for 1M requests
\end{itemize}

\subsection{Scenario 5: Compliance and Governance}

\textbf{Situation}: A financial services company with 50 AWS accounts needs to ensure no S3 buckets are publicly accessible across the organization. They also need to track all changes and demonstrate compliance.

\textbf{Requirements}:
\begin{itemize}
  \item Enforce no public S3 buckets
  \item Apply to all accounts
  \item Monitor compliance continuously
  \item Audit all changes
  \item Automated remediation preferred
\end{itemize}

\textbf{Question}: How can they enforce and monitor this policy?

\textbf{Recommended Solution}:

\begin{enumerate}
  \item \textbf{AWS Organizations Setup}:
  \begin{itemize}
    \item Group accounts using \textbf{Organizational Units (OUs)}
    \item Example structure: Production OU, Development OU, Test OU
  \end{itemize}

  \item \textbf{Service Control Policies (SCPs)}:
  \begin{itemize}
    \item Create SCP denying \texttt{s3:PutBucketPublicAccessBlock} with value False
    \item Deny \texttt{s3:PutBucketPolicy} if it allows public access
    \item Apply to root or specific OUs
    \item SCPs define maximum permissions (even admins can't override)
  \end{itemize}

  \item \textbf{S3 Block Public Access}:
  \begin{itemize}
    \item Enable \textbf{S3 Block Public Access} at organization level
    \item Applies to all accounts in organization
    \item Prevents accidental public exposure
  \end{itemize}

  \item \textbf{Continuous Monitoring}:
  \begin{itemize}
    \item Enable \textbf{AWS Config} across all accounts
    \item Deploy \textbf{s3-bucket-public-read-prohibited} rule
    \item Deploy \textbf{s3-bucket-public-write-prohibited} rule
    \item Automatic compliance reporting
  \end{itemize}

  \item \textbf{Automated Remediation}:
  \begin{itemize}
    \item Configure \textbf{AWS Config auto-remediation}
    \item Use AWS Systems Manager Automation documents
    \item Automatically disable public access when detected
  \end{itemize}

  \item \textbf{Audit and Logging}:
  \begin{itemize}
    \item Enable \textbf{CloudTrail} in all accounts
    \item Centralize logs in dedicated security account
    \item Track all S3 API calls
    \item Set up \textbf{CloudWatch alarms} for policy violations
  \end{itemize}

  \item \textbf{Centralized Security}:
  \begin{itemize}
    \item Use \textbf{AWS Security Hub} for centralized security view
    \item Aggregates findings from Config, GuardDuty, Inspector
    \item Compliance dashboards for standards (PCI DSS, CIS)
  \end{itemize}
\end{enumerate}

\textbf{Additional Recommendations}:
\begin{itemize}
  \item Regular compliance reports using \textbf{AWS Artifact}
  \item Periodic access reviews
  \item Employee training on security best practices
  \item Implement least privilege IAM policies
\end{itemize}

\subsection{Scenario 6: Disaster Recovery Strategy}

\textbf{Situation}: An e-commerce company needs disaster recovery for their application. Their business requires:
\begin{itemize}
  \item RPO (Recovery Point Objective): 1 hour
  \item RTO (Recovery Time Objective): 4 hours
  \item Currently running in us-east-1
\end{itemize}

\textbf{Question}: What DR strategy should they implement?

\textbf{DR Strategy Options}:

\begin{enumerate}
  \item \textbf{Backup and Restore} (Lowest cost, highest RTO)
  \begin{itemize}
    \item RPO: Hours to days
    \item RTO: Hours to days
    \item Good for: Non-critical workloads
    \item Not suitable for this scenario
  \end{itemize}

  \item \textbf{Pilot Light} (Recommended for this scenario)
  \begin{itemize}
    \item RPO: Minutes to hours
    \item RTO: Hours
    \item Keep minimal version running in DR region
    \item Scale up during disaster
  \end{itemize}

  \item \textbf{Warm Standby}
  \begin{itemize}
    \item RPO: Seconds to minutes
    \item RTO: Minutes
    \item Reduced version always running
    \item Higher cost than Pilot Light
  \end{itemize}

  \item \textbf{Multi-Site Active/Active}
  \begin{itemize}
    \item RPO: Near zero
    \item RTO: Near zero
    \item Highest cost
    \item Overkill for 4-hour RTO requirement
  \end{itemize}
\end{enumerate}

\textbf{Recommended Pilot Light Implementation}:

\begin{enumerate}
  \item \textbf{Data Replication}:
  \begin{itemize}
    \item Use \textbf{RDS cross-region read replicas}
    \item Replicate from us-east-1 to us-west-2
    \item Meets 1-hour RPO requirement
  \end{itemize}

  \item \textbf{Application AMIs}:
  \begin{itemize}
    \item Regularly copy AMIs to DR region
    \item Keep AMIs up-to-date
    \item Automate with Lambda
  \end{itemize}

  \item \textbf{Infrastructure as Code}:
  \begin{itemize}
    \item Use \textbf{CloudFormation templates}
    \item Pre-create VPC, subnets, security groups in DR region
    \item Keep Auto Scaling groups in DR region with 0 capacity
  \end{itemize}

  \item \textbf{DNS Failover}:
  \begin{itemize}
    \item Use \textbf{Route 53 health checks}
    \item Configure failover routing policy
    \item Automatic DNS failover to DR region
  \end{itemize}

  \item \textbf{Testing}:
  \begin{itemize}
    \item Quarterly DR drills
    \item Document runbooks
    \item Measure actual RTO/RPO
  \end{itemize}
\end{enumerate}

\textbf{Failover Process}:
\begin{enumerate}
  \item Detect primary region failure (Route 53 health check)
  \item Promote RDS read replica to master
  \item Update CloudFormation stack to scale up Auto Scaling
  \item Route 53 automatically redirects traffic
  \item Total time: ~2-3 hours (meets 4-hour RTO)
\end{enumerate}

\subsection{Scenario 7: Hybrid Cloud Connectivity}

\textbf{Situation}: A manufacturing company wants to extend their on-premises data center to AWS while maintaining consistent network performance for their ERP system.

\textbf{Requirements}:
\begin{itemize}
  \item Consistent network latency
  \item Private connection (no internet)
  \item Bandwidth: 1 Gbps
  \item Access to multiple VPCs
\end{itemize}

\textbf{Connection Options Analysis}:

\begin{table}[h]
\centering
\begin{tabular}{|p{3cm}|p{4cm}|p{4cm}|p{3cm}|}
\hline
\textbf{Solution} & \textbf{Pros} & \textbf{Cons} & \textbf{Best For} \\
\hline

Site-to-Site VPN & Quick setup (hours), low cost, encrypted & Variable latency, internet-based, limited bandwidth & Dev/test, temporary \\
\hline

AWS Direct Connect & Consistent performance, high bandwidth, private & Expensive, takes weeks, not encrypted by default & Production, high bandwidth \\
\hline

Direct Connect + VPN & Best of both worlds & Most expensive, complex & Regulated industries \\
\hline
\end{tabular}
\end{table}

\textbf{Recommended Solution: AWS Direct Connect}

\begin{enumerate}
  \item \textbf{Direct Connect Setup}:
  \begin{itemize}
    \item Order 1 Gbps Direct Connect port
    \item Work with AWS Direct Connect Partner
    \item Provision takes 2-4 weeks
    \item Set up cross-connect at colocation facility
  \end{itemize}

  \item \textbf{Multiple VPC Access}:
  \begin{itemize}
    \item Use \textbf{Direct Connect Gateway}
    \item Connect to multiple VPCs across regions
    \item Single Direct Connect connection
    \item Simplifies connectivity
  \end{itemize}

  \item \textbf{High Availability}:
  \begin{itemize}
    \item Order second Direct Connect connection (different location)
    \item Configure BGP for automatic failover
    \item Or use VPN as backup connection
  \end{itemize}

  \item \textbf{Security}:
  \begin{itemize}
    \item Layer VPN over Direct Connect for encryption
    \item Or use \textbf{MACsec} encryption
    \item Private VIF for VPC access
    \item Public VIF for public AWS services
  \end{itemize}
\end{enumerate}

\section{Common Troubleshooting Scenarios}

\subsection{Cannot Connect to EC2 Instance}

\textbf{Symptoms}: SSH or RDP connection times out or refused

\textbf{Troubleshooting Steps}:

\begin{enumerate}
  \item \textbf{Verify Instance Status}:
  \begin{itemize}
    \item Check instance state is "running"
    \item Check status checks are passing
    \item View system log for boot errors
  \end{itemize}

  \item \textbf{Check Security Group}:
  \begin{itemize}
    \item Ensure inbound rule allows SSH (22) or RDP (3389)
    \item Verify source IP is allowed (0.0.0.0/0 or your IP)
    \item Check if security group changed recently
  \end{itemize}

  \item \textbf{Check Network ACL}:
  \begin{itemize}
    \item Ensure NACL allows inbound traffic on port
    \item Ensure NACL allows ephemeral outbound ports (1024-65535)
    \item NACLs are stateless!
  \end{itemize}

  \item \textbf{Verify Network Configuration}:
  \begin{itemize}
    \item Instance has public IP (if connecting from internet)
    \item Instance in public subnet (has IGW route)
    \item Or using bastion host for private subnet
  \end{itemize}

  \item \textbf{Check Key Pair}:
  \begin{itemize}
    \item Using correct .pem/.ppk file
    \item File permissions correct (chmod 400 for .pem)
    \item Key pair matches instance
  \end{itemize}

  \item \textbf{Check Route Table}:
  \begin{itemize}
    \item Subnet has route to IGW (0.0.0.0/0 → igw-xxx)
    \item Or route to NAT Gateway for private subnet
  \end{itemize}
\end{enumerate}

\subsection{S3 Access Denied Errors}

\textbf{Common Causes and Solutions}:

\begin{enumerate}
  \item \textbf{IAM Permissions}:
  \begin{itemize}
    \item Verify IAM policy grants s3:GetObject, s3:PutObject
    \item Check for explicit Deny statements
    \item Verify resource ARN in policy matches bucket
  \end{itemize}

  \item \textbf{Bucket Policy}:
  \begin{itemize}
    \item Check bucket policy doesn't deny access
    \item Verify Principal in policy
    \item Check for IP-based restrictions
  \end{itemize}

  \item \textbf{Block Public Access}:
  \begin{itemize}
    \item If public access needed, disable Block Public Access
    \item Check both bucket-level and account-level settings
  \end{itemize}

  \item \textbf{Encryption}:
  \begin{itemize}
    \item If using SSE-KMS, verify KMS key policy
    \item Ensure user has kms:Decrypt permission
  \end{itemize}

  \item \textbf{Cross-Account Access}:
  \begin{itemize}
    \item Bucket policy must allow cross-account access
    \item Assume role with correct permissions
  \end{itemize}
\end{enumerate}

\subsection{Lambda Function Issues}

\textbf{Issue 1: Function Timing Out}

\textbf{Solutions}:
\begin{itemize}
  \item Increase timeout (default 3 sec, max 15 min)
  \item Optimize code performance
  \item Check VPC configuration (can add latency)
  \item Increase memory (also increases CPU)
  \item Investigate cold start delays
\end{itemize}

\textbf{Issue 2: Insufficient Permissions}

\textbf{Solutions}:
\begin{itemize}
  \item Check Lambda execution role has required permissions
  \item Review CloudWatch Logs for permission errors
  \item Add necessary IAM policies to execution role
  \item For VPC: Ensure role has VPC execution permissions
\end{itemize}

\textbf{Issue 3: Throttling}

\textbf{Solutions}:
\begin{itemize}
  \item Request concurrency limit increase
  \item Implement exponential backoff in calling application
  \item Use SQS to buffer requests
  \item Consider reserved concurrency for critical functions
\end{itemize}
