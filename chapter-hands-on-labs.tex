\chapter{Hands-On Practice Labs}




\subsection{Table of Contents}

\begin{itemize}
  \item \href{\#introduction}{Introduction}
  \item \href{\#pre-lab-checklist}{Pre-Lab Checklist}
  \item \href{\#prerequisites}{Prerequisites}
  \item \href{\#important-notes}{Important Notes}
  \item \href{\#lab-difficulty-guide}{Lab Difficulty Guide}
  \item \href{\#lab-1-set-up-billing-alerts-and-budget}{Lab 1: Set Up Billing Alerts and Budget}
  \item \href{\#lab-2-iam-users-groups-roles-and-mfa}{Lab 2: IAM Users, Groups, Roles, and MFA}
  \item \href{\#lab-3-launch-and-configure-ec2-instance}{Lab 3: Launch and Configure EC2 Instance}
  \item \href{\#lab-4-amazon-s3-storage-and-website-hosting}{Lab 4: Amazon S3 Storage and Website Hosting}
  \item \href{\#lab-5-vpc-subnets-and-network-configuration}{Lab 5: VPC, Subnets, and Network Configuration}
  \item \href{\#lab-6-amazon-rds-database}{Lab 6: Amazon RDS Database}
  \item \href{\#lab-7-cloudwatch-monitoring-and-alarms}{Lab 7: CloudWatch Monitoring and Alarms}
  \item \href{\#lab-8-aws-cost-management-tools}{Lab 8: AWS Cost Management Tools}
  \item \href{\#lab-9-lambda-serverless-function}{Lab 9: Lambda Serverless Function}
  \item \href{\#lab-10-cloudformation-infrastructure-as-code}{Lab 10: CloudFormation Infrastructure as Code}
  \item \href{\#lab-11-auto-scaling-and-load-balancing}{Lab 11: Auto Scaling and Load Balancing}
  \item \href{\#lab-12-dynamodb-hands-on}{Lab 12: DynamoDB Hands-On}
  \item \href{\#lab-13-sns-and-sqs-messaging}{Lab 13: SNS and SQS Messaging}
  \item \href{\#lab-14-route-53-dns-and-health-checks}{Lab 14: Route 53 DNS and Health Checks}
  \item \href{\#lab-15-aws-organizations-and-multi-account-setup}{Lab 15: AWS Organizations and Multi-Account Setup}
  \item \href{\#troubleshooting-faq}{Troubleshooting FAQ}
  \item \href{\#additional-practice-recommendations}{Additional Practice Recommendations}
\end{itemize}


---

\subsection{Introduction}


\begin{examtip}
\textbf{Important:} Hands-on experience is crucial for exam success. These labs will help you understand AWS services beyond theory and are designed to stay within Free Tier limits when done carefully.
\end{examtip}


Practical experience with AWS services provides:
\begin{itemize}
  \item Deeper understanding of service capabilities
  \item Familiarity with AWS Management Console
  \item Confidence during the exam
  \item Real-world skills applicable to jobs
  \item Better retention of concepts
\end{itemize}


\textbf{Total Time Investment:} Approximately 11-12 hours for all 15 labs

---

\subsection{Pre-Lab Checklist}


Before starting any lab, ensure you have completed the following:

\subsubsection{Essential Requirements}

\begin{itemize}
  \item [ ] \textbf{AWS Account created and activated} (may take up to 24 hours)
  \item [ ] \textbf{Valid credit/debit card} on file (required even for Free Tier)
  \item [ ] \textbf{Email access} to confirm SNS subscriptions and receive alerts
  \item [ ] \textbf{Billing alerts set up} (Lab 1 - do this FIRST!)
  \item [ ] \textbf{Free Tier dashboard bookmarked} for easy monitoring
  \item [ ] \textbf{Account ID noted} and saved securely
\end{itemize}


\subsubsection{Technical Setup}

\begin{itemize}
  \item [ ] \textbf{Modern web browser} (Chrome, Firefox, Safari, Edge - latest version)
  \item [ ] \textbf{Stable internet connection} (minimum 5 Mbps recommended)
  \item [ ] \textbf{Text editor} installed (VS Code, Sublime, Notepad++, or any editor)
  \item [ ] \textbf{SSH client available} (built-in on Mac/Linux, PuTTY or OpenSSH for Windows)
  \item [ ] \textbf{Terminal/command prompt} access and basic familiarity
\end{itemize}


\subsubsection{Knowledge Prerequisites}

\begin{itemize}
  \item [ ] \textbf{Basic understanding} of cloud computing concepts
  \item [ ] \textbf{Familiarity with IP addresses} and networking basics (for VPC labs)
  \item [ ] \textbf{Basic command line} experience (helpful but not required)
  \item [ ] \textbf{Understanding of JSON/YAML} formats (for CloudFormation)
\end{itemize}


\subsubsection{Safety Measures}

\begin{itemize}
  \item [ ] \textbf{Password manager} or secure location for credentials
  \item [ ] \textbf{Notepad ready} for documenting resource IDs and URLs
  \item [ ] \textbf{Calendar reminder} set to check and clean up resources daily
  \item [ ] \textbf{Budget limit decided} (recommend \$10-20 maximum)
  \item [ ] \textbf{Understanding of charges} - know what costs money vs. what's free
\end{itemize}


\subsubsection{Best Practices Before Starting}

\begin{itemize}
  \item [ ] \textbf{Read entire lab} before executing any steps
  \item [ ] \textbf{Take screenshots} as you progress for documentation
  \item [ ] \textbf{Use consistent naming} conventions (include date or lab number)
  \item [ ] \textbf{Tag all resources} with project name for easy identification
  \item [ ] \textbf{Set aside uninterrupted time} for each lab
  \item [ ] \textbf{Prepare to take notes} on errors and solutions
\end{itemize}


\subsubsection{Region Selection}

\begin{itemize}
  \item [ ] \textbf{Choose your primary region} (recommend us-east-1 or us-west-2)
  \item [ ] \textbf{Verify Free Tier availability} in chosen region
  \item [ ] \textbf{Note region for all labs} to maintain consistency
  \item [ ] \textbf{Bookmark region selector} for quick access
\end{itemize}


\subsubsection{Time Planning}

\begin{itemize}
  \item [ ] \textbf{Review lab duration} estimates
  \item [ ] \textbf{Add 25\% buffer time} for troubleshooting
  \item [ ] \textbf{Plan cleanup time} (5-10 minutes per lab)
  \item [ ] \textbf{Schedule breaks} between complex labs
  \item [ ] \textbf{Avoid starting labs} late at night (may forget cleanup)
\end{itemize}


---

\subsection{Prerequisites}


\subsubsection{Create Your AWS Account}


Follow these steps to create your AWS account:

\begin{enumerate}
  \item Visit \href{https://aws.amazon.com}{https://aws.amazon.com}
  \item Click \textbf{"Create an AWS Account"}
  \item Provide:
\end{enumerate}

\begin{itemize}
  \item Email address
  \item Password
  \item AWS account name
\end{itemize}

\begin{enumerate}
  \item Enter contact information
  \item Provide payment method
\end{enumerate}

\begin{itemize}
  \item Credit or debit card required
  \item You won't be charged if you stay within Free Tier
\end{itemize}

\begin{enumerate}
  \item Verify identity via phone call or SMS
  \item Select Support Plan: \textbf{Basic (Free)}
  \item Wait for account activation (can take up to 24 hours)
  \item Check email for confirmation
\end{enumerate}


\subsubsection{AWS Free Tier}


\textbf{Duration:} 12 months from account creation date

\textbf{Key Free Tier Services:}
\begin{itemize}
  \item \textbf{EC2:} 750 hours/month of t2.micro or t3.micro instances
  \item \textbf{S3:} 5 GB of standard storage
  \item \textbf{RDS:} 750 hours/month of db.t2.micro, db.t3.micro, or db.t4g.micro
  \item \textbf{Lambda:} 1 million free requests per month
  \item \textbf{CloudWatch:} 10 custom metrics and alarms
\end{itemize}


\textbf{Always Free Services:}
\begin{itemize}
  \item DynamoDB: 25 GB storage
  \item Lambda: 1 million requests/month
  \item CloudFormation: No charge (pay for resources created)
\end{itemize}


\begin{examtip}
Set up billing alerts immediately to avoid unexpected charges. AWS Free Tier is generous, but mistakes can incur costs.
\end{examtip}


---

\subsection{Important Notes}


\subsubsection{Safety and Cost Management}


\begin{enumerate}
  \item \textbf{Always clean up resources} after each lab to avoid charges
  \item \textbf{Set billing alerts} before starting any hands-on work
  \item \textbf{Use t2.micro or t3.micro} instance types (Free Tier eligible)
  \item \textbf{Choose Free Tier regions} (us-east-1, us-west-2 recommended)
  \item \textbf{Monitor Free Tier usage} in billing dashboard regularly
  \item \textbf{Don't leave resources running} overnight or when not in use
\end{enumerate}


\subsubsection{Best Practices}


\begin{itemize}
  \item Complete labs in order (dependencies exist)
  \item Take notes and screenshots for future reference
  \item Read error messages carefully - they often contain solutions
  \item Use tags to identify lab resources for easy cleanup
  \item Stop services rather than terminate if you plan to return
\end{itemize}


\subsubsection{Troubleshooting Resources}


\begin{itemize}
  \item AWS Documentation: \href{https://docs.aws.amazon.com}{https://docs.aws.amazon.com}
  \item AWS re:Post (community forum): \href{https://repost.aws}{https://repost.aws}
  \item Service Health Dashboard: \href{https://status.aws.amazon.com}{https://status.aws.amazon.com}
  \item AWS Support (if you have paid support plan)
\end{itemize}


---

\subsection{Lab Difficulty Guide}


\subsubsection{Understanding Difficulty Ratings}


Each lab is rated on three dimensions:

\textbf{Technical Complexity:}
\begin{itemize}
  \item \textbf{Beginner:} Straightforward steps, minimal technical knowledge required
  \item \textbf{Intermediate:} Some technical concepts, may require troubleshooting
  \item \textbf{Advanced:} Complex networking/architecture, requires careful attention
\end{itemize}


\textbf{Time Investment:}
\begin{itemize}
  \item Short: 15-30 minutes
  \item Medium: 30-60 minutes
  \item Long: 60+ minutes
\end{itemize}


\textbf{Prerequisites:}
\begin{itemize}
  \item Minimal: Just AWS account
  \item Moderate: Completion of earlier labs helpful
  \item Extensive: Requires specific knowledge or completed labs
\end{itemize}


\subsubsection{Lab Difficulty Matrix}


\begin{longtable}{lllll}
\toprule
\textbf{Lab} & \textbf{Difficulty} & \textbf{Duration} & \textbf{Prerequisites} & \textbf{Complexity Score} \\
\midrule
Lab 1: Billing Alerts & Beginner & 15 min & None & 1/5 \\
Lab 2: IAM \& MFA & Beginner & 30 min & None & 2/5 \\
Lab 3: EC2 Instance & Intermediate & 45 min & Lab 2 recommended & 3/5 \\
Lab 4: S3 Website & Intermediate & 40 min & Basic HTML & 2/5 \\
Lab 5: VPC Network & Advanced & 60 min & Networking basics & 4/5 \\
Lab 6: RDS Database & Intermediate & 30 min & Lab 3 \& 5 & 3/5 \\
Lab 7: CloudWatch & Beginner & 25 min & Lab 3 & 2/5 \\
Lab 8: Cost Tools & Beginner & 30 min & None & 1/5 \\
Lab 9: Lambda & Intermediate & 25 min & Basic Python & 2/5 \\
Lab 10: CloudFormation & Intermediate & 20 min & YAML/JSON & 3/5 \\
Lab 11: Auto Scaling & Advanced & 45 min & Lab 3 \& 5 & 4/5 \\
Lab 12: DynamoDB & Intermediate & 35 min & Database concepts & 2/5 \\
Lab 13: SNS \& SQS & Intermediate & 30 min & Lab 9 helpful & 3/5 \\
Lab 14: Route 53 & Intermediate & 25 min & DNS knowledge & 3/5 \\
Lab 15: Organizations & Advanced & 40 min & Multiple accounts & 4/5 \\
\bottomrule
\end{longtable}

\subsubsection{Recommended Learning Paths}


\textbf{Path 1: Absolute Beginners}
\begin{enumerate}
  \item Lab 1 (Billing) → Lab 2 (IAM) → Lab 8 (Cost Tools) → Lab 4 (S3)
  \item Then proceed with: Lab 3 → Lab 7 → Lab 9 → Lab 12
\end{enumerate}


\textbf{Path 2: Developers}
\begin{enumerate}
  \item Lab 1 → Lab 2 → Lab 3 → Lab 9 (Lambda)
  \item Then: Lab 4 → Lab 13 (Messaging) → Lab 12 (DynamoDB) → Lab 10 (IaC)
\end{enumerate}


\textbf{Path 3: Infrastructure/Operations}
\begin{enumerate}
  \item Lab 1 → Lab 2 → Lab 3 → Lab 5 (VPC)
  \item Then: Lab 11 (Auto Scaling) → Lab 6 (RDS) → Lab 14 (Route 53) → Lab 15
\end{enumerate}


\textbf{Path 4: Complete Sequential} (Recommended for Most)
\begin{itemize}
  \item Follow labs 1-15 in order for comprehensive understanding
\end{itemize}


\subsubsection{Skill Development Tracking}


After completing each lab, self-assess your confidence:

\textbf{Rating Scale:}
\begin{itemize}
  \item 1 = Need to review
  \item 2 = Understood with help
  \item 3 = Comfortable, could explain to others
  \item 4 = Expert, could troubleshoot independently
  \item 5 = Could teach this topic
\end{itemize}


\textbf{Target Scores for Exam:}
\begin{itemize}
  \item Core labs (1-10): Aim for 4/5
  \item Advanced labs (11-15): Aim for 3/5
\end{itemize}


---

\subsection{Lab 1: Set Up Billing Alerts and Budget}


\textbf{Duration:} 15 minutes
\textbf{Cost:} Free
\textbf{Difficulty:} Beginner

\subsubsection{Learning Objectives}


By the end of this lab, you will be able to:

\begin{enumerate}
  \item Enable IAM user access to billing information
  \item Create CloudWatch billing alarms with SNS notifications
  \item Configure AWS Budgets with multiple alert thresholds
  \item Understand the difference between CloudWatch billing metrics and AWS Budgets
  \item Monitor Free Tier usage to prevent unexpected charges
  \item Set up proactive cost monitoring for AWS account safety
\end{enumerate}


\subsubsection{Why This Lab Matters}


\textbf{Real-World Scenario:} A developer left an RDS database running in a personal AWS account and accumulated \$450 in charges over a weekend. Proper billing alerts would have caught this within hours, limiting damage to less than \$20.

\textbf{Exam Relevance:} The Cloud Practitioner exam heavily emphasizes cost management, billing, and monitoring. Questions often test your understanding of:
\begin{itemize}
  \item AWS Budgets vs. Cost Explorer vs. CloudWatch billing alarms
  \item Free Tier limits and monitoring
  \item Billing alert best practices
  \item SNS notification mechanisms
\end{itemize}


\subsubsection{Objective}


Protect yourself from unexpected charges by setting up billing alerts and budgets.

\subsubsection{Prerequisites}


\begin{itemize}
  \item Active AWS account
  \item Access to root account or IAM user with billing permissions
\end{itemize}


\subsubsection{Step-by-Step Instructions}


\paragraph{Part 1: Enable Billing Alerts}


\begin{keypoint}
\textbf{What You'll See:} The AWS Management Console with the navigation bar at the top showing your account name and region selector.
\end{keypoint}


\begin{enumerate}
  \item Sign in to \textbf{AWS Management Console} as root user or admin
\end{enumerate}

\begin{itemize}
  \item You should see the AWS Services search bar and dashboard
\end{itemize}


\begin{enumerate}
  \item Click your \textbf{account name} (top right corner) → Select \textbf{"Account"}
\end{enumerate}

\begin{itemize}
  \item This opens the Account settings page
  \item Alternative path: Navigate via "My Account" link in dropdown
\end{itemize}


\begin{enumerate}
  \item Scroll down to \textbf{"IAM User and Role Access to Billing Information"} section
\end{enumerate}

\begin{itemize}
  \item This section is about halfway down the page
  \item You'll see a description about allowing IAM users to access billing data
  \item \textbf{Why this matters:} By default, only root users can see billing info. This setting allows IAM users with appropriate permissions to view costs.
\end{itemize}


\begin{enumerate}
  \item Click \textbf{"Edit"} button on the right side
\end{enumerate}

\begin{itemize}
  \item A popup or inline editor will appear
\end{itemize}


\begin{enumerate}
  \item Check the box to \textbf{"Activate IAM Access"}
\end{enumerate}

\begin{itemize}
  \item When enabled, IAM users with billing permissions can see costs
  \item \textbf{Best Practice:} This allows you to use IAM users instead of root for daily billing monitoring
\end{itemize}


\begin{enumerate}
  \item Click \textbf{"Update"}
\end{enumerate}

\begin{itemize}
  \item You'll see a success message: "Successfully updated IAM user/role access to billing information"
\end{itemize}


\textbf{Validation Step:}
\begin{itemize}
  \item The setting should now show as "Activated"
  \item This is a one-time setup per AWS account
\end{itemize}


\textbf{Common Issue:} If you don't see this option, verify you're logged in as the root user (account owner), not an IAM user.

\paragraph{Part 2: Create CloudWatch Billing Alarm}


\begin{enumerate}
  \item Navigate to \textbf{CloudWatch} service
  \item Select region: \textbf{N. Virginia (us-east-1)} (billing metrics only in us-east-1)
  \item Go to \textbf{"Alarms"} → \textbf{"Billing"} → \textbf{"Create alarm"}
  \item Click \textbf{"Select metric"}
  \item Select \textbf{"Billing"} → \textbf{"Total Estimated Charge"}
  \item Select \textbf{USD} checkbox
  \item Click \textbf{"Select metric"}
  \item Configure alarm:
\end{enumerate}

\begin{itemize}
  \item \textbf{Threshold type:} Static
  \item \textbf{Whenever:} Greater than
  \item \textbf{Amount:} \$5 (or your preferred threshold)
\end{itemize}

\begin{enumerate}
  \item Click \textbf{"Next"}
\end{enumerate}


\paragraph{Part 3: Configure SNS Notification}


\begin{enumerate}
  \item Under \textbf{"Notification"}:
\end{enumerate}

\begin{itemize}
  \item \textbf{Select an SNS topic:} Create new topic
  \item \textbf{Topic name:} "Billing-Alerts"
  \item \textbf{Email endpoints:} Enter your email address
\end{itemize}

\begin{enumerate}
  \item Click \textbf{"Create topic"}
  \item Click \textbf{"Next"}
  \item \textbf{Alarm name:} "Monthly-Billing-Alert"
  \item \textbf{Alarm description:} "Alert when monthly charges exceed \$5"
  \item Click \textbf{"Next"}
  \item Review and click \textbf{"Create alarm"}
  \item \textbf{Check your email} and click confirmation link
\end{enumerate}


\paragraph{Part 4: Create AWS Budget}


\begin{enumerate}
  \item Navigate to \textbf{"Billing and Cost Management"}
  \item Click \textbf{"Budgets"} in left menu
  \item Click \textbf{"Create budget"}
  \item Select \textbf{"Cost budget"} → \textbf{"Next"}
  \item Configure budget:
\end{enumerate}

\begin{itemize}
  \item \textbf{Budget name:} "Monthly-Cost-Budget"
  \item \textbf{Period:} Monthly
  \item \textbf{Budget amount:} \$10
  \item \textbf{Budget scope:} All AWS services
\end{itemize}

\begin{enumerate}
  \item Click \textbf{"Next"}
  \item Configure alerts:
\end{enumerate}

\begin{itemize}
  \item \textbf{Alert 1:} 80\% of budgeted amount
  \item \textbf{Email recipients:} Your email
  \item \textbf{Alert 2:} 100\% of budgeted amount
\end{itemize}

\begin{enumerate}
  \item Click \textbf{"Next"}
  \item Review and click \textbf{"Create budget"}
\end{enumerate}


\subsubsection{Expected Outcomes}


\begin{itemize}
  \item CloudWatch billing alarm created and active
  \item Email confirmation received for SNS subscription
  \item Budget created with two alert thresholds
  \item Email notifications configured
\end{itemize}


\subsubsection{Verification}


\begin{enumerate}
  \item Go to \textbf{CloudWatch} → \textbf{Alarms} → Verify alarm shows "OK" state
  \item Go to \textbf{Budgets} → Verify budget shows current spend vs. budget
  \item Check email for SNS confirmation
\end{enumerate}


\subsubsection{Troubleshooting}


\textbf{Problem:} Billing metric not showing
\textbf{Solution:} Ensure you're in us-east-1 region; billing metrics only available there

\textbf{Problem:} Email confirmation not received
\textbf{Solution:} Check spam folder; resend confirmation from SNS console

\textbf{Problem:} Can't access billing information
\textbf{Solution:} Enable IAM access to billing in Account settings

\subsubsection{What You Should See at Each Step}


\textbf{After Creating CloudWatch Alarm:}
\begin{itemize}
  \item Alarm appears in CloudWatch → Alarms dashboard
  \item Status shows "Insufficient data" initially (normal - need 24 hours of data)
  \item After confirmation, status changes to "OK" (no alarm condition met)
  \item Graph shows estimated charges over time
\end{itemize}


\textbf{After Confirming SNS Subscription:}
\begin{itemize}
  \item Email from "AWS Notifications" with subject "AWS Notification - Subscription Confirmation"
  \item After clicking link, browser shows "Subscription confirmed!"
  \item SNS console shows subscription status as "Confirmed"
\end{itemize}


\textbf{After Creating Budget:}
\begin{itemize}
  \item Budget appears in Budgets dashboard
  \item Shows current spend vs. budgeted amount (likely \$0.00 of \$10.00)
  \item Alert thresholds displayed at 80\% (\$8) and 100\% (\$10)
  \item Forecast shows projected end-of-month spend
\end{itemize}


\subsubsection{Real-World Tips}


\textbf{Recommended Alert Thresholds:}
\begin{itemize}
  \item For students/learners: \$5 alarm, \$10 budget
  \item For production accounts: Set based on expected monthly spend
  \item Conservative approach: Alert at 50\%, 80\%, and 100\%
\end{itemize}


\textbf{Multiple Budget Strategy:}
\begin{itemize}
  \item Total account budget: \$10
  \item Per-service budgets: EC2 \$3, RDS \$3, S3 \$2, Other \$2
  \item Helps identify which service is driving costs
\end{itemize}


\textbf{Alert Fatigue Prevention:}
\begin{itemize}
  \item Don't set thresholds too low (avoid constant alerts)
  \item Review and adjust monthly based on actual usage
  \item Use forecasted budgets for proactive monitoring
\end{itemize}


\textbf{Best Practice - Multiple Notification Recipients:}
\begin{itemize}
  \item Add team members' emails to SNS topic
  \item Consider SMS notifications for critical budgets (note: SMS costs money)
  \item Set up Slack/Teams integration via Lambda (advanced)
\end{itemize}


\subsubsection{Alternative Approaches}


\textbf{Method 1: AWS Budgets Only}
\begin{itemize}
  \item Skip CloudWatch billing alarm
  \item Use only AWS Budgets (simpler for beginners)
  \item Limitation: Less granular, updated 3x per day vs. CloudWatch every 5 minutes
\end{itemize}


\textbf{Method 2: CloudWatch Only}
\begin{itemize}
  \item Skip AWS Budgets
  \item Create multiple CloudWatch alarms at different thresholds
  \item Limitation: Doesn't show forecasts or budget tracking UI
\end{itemize}


\textbf{Method 3: Cost Anomaly Detection (Advanced)}
\begin{itemize}
  \item AWS provides ML-based anomaly detection
  \item Navigate to Cost Management → Cost Anomaly Detection
  \item Create monitor → Set alert preferences
  \item Automatically detects unusual spending patterns
\end{itemize}


\subsubsection{Cleanup}


\begin{keypoint}
\textbf{Note:} Keep these alerts active for ongoing protection. No cleanup needed.
\end{keypoint}


\textbf{If you need to remove alerts later:}

\begin{enumerate}
  \item \textbf{Delete CloudWatch Alarm:}
\end{enumerate}

\begin{itemize}
  \item CloudWatch → Alarms → Select alarm → Actions → Delete
  \item Confirm deletion by typing alarm name
  \item Alarm deleted immediately
\end{itemize}


\begin{enumerate}
  \item \textbf{Delete Budget:}
\end{enumerate}

\begin{itemize}
  \item Billing → Budgets → Select budget → Actions → Delete budget
  \item Type "delete" to confirm
  \item Budget removed from dashboard
\end{itemize}


\begin{enumerate}
  \item \textbf{Delete SNS Topic:}
\end{enumerate}

\begin{itemize}
  \item SNS → Topics → Select topic → Delete
  \item Type "delete me" to confirm
  \item Associated subscriptions automatically deleted
\end{itemize}


\begin{enumerate}
  \item \textbf{Verify Cleanup:}
\end{enumerate}

\begin{itemize}
  \item Check CloudWatch Alarms list (should be empty)
  \item Check Budgets dashboard (should show no budgets)
  \item Emails stop arriving
\end{itemize}


\textbf{Cost Impact of Deletion:}
\begin{itemize}
  \item No ongoing costs for these free services
  \item Safe to keep active indefinitely
\end{itemize}


---

\subsubsection{Post-Lab Knowledge Check}


Test your understanding of Lab 1 concepts:

\textbf{Question 1:} What's the difference between CloudWatch billing alarms and AWS Budgets?

<details>
<summary>Click to reveal answer</summary>

\textbf{Answer:}
\begin{itemize}
  \item \textbf{CloudWatch Alarms:} Monitor actual charges in near real-time (every 5-10 minutes), trigger SNS notifications when threshold exceeded. Only available in us-east-1 region.
  \item \textbf{AWS Budgets:} Track costs and usage against planned budgets, provide forecasting, update 3 times per day. Support usage-based budgets (not just cost). Available globally.
  \item \textbf{Use both together:} CloudWatch for immediate alerts, Budgets for tracking and forecasting.
\end{itemize}


</details>

\textbf{Question 2:} Why must CloudWatch billing metrics be created in the us-east-1 region?

<details>
<summary>Click to reveal answer</summary>

\textbf{Answer:} Billing is a global service, and AWS consolidates all billing data in the us-east-1 (N. Virginia) region. Billing metrics are only published to CloudWatch in this region. This is an AWS design decision to centralize global billing data.

\textbf{Exam Tip:} Remember this for the exam - it's a common trick question!

</details>

\textbf{Question 3:} If your alarm shows "Insufficient data" status, is something wrong?

<details>
<summary>Click to reveal answer</summary>

\textbf{Answer:} No, this is normal. "Insufficient data" means CloudWatch doesn't have enough data points yet to evaluate the alarm. This typically happens:
\begin{itemize}
  \item Within first 24 hours of alarm creation
  \item For new AWS accounts with minimal usage
  \item After changing alarm parameters
\end{itemize}


Status will change to "OK" once sufficient data is available. If charges exceed threshold, status changes to "In alarm."

</details>

\textbf{Question 4:} You set a \$10 budget but received an alert at \$8. Why?

<details>
<summary>Click to reveal answer</summary>

\textbf{Answer:} You configured an alert threshold at 80\% of your budget. 80\% of \$10 = \$8. This is intentional and a best practice. Getting alerts before hitting 100\% gives you time to investigate and take action before exceeding your budget.

Multiple thresholds (50\%, 80\%, 100\%) provide escalating warnings as you approach your limit.

</details>

\textbf{Question 5:} Can you set up billing alarms for individual services like EC2 or S3?

<details>
<summary>Click to reveal answer</summary>

\textbf{Answer:}
\begin{itemize}
  \item \textbf{CloudWatch Billing Alarms:} Can only monitor total estimated charges for the account, not individual services.
  \item \textbf{AWS Budgets:} YES, can create service-specific budgets (e.g., EC2 only, S3 only).
  \item \textbf{Best Practice:} Use AWS Budgets for service-level cost tracking, CloudWatch alarms for total account spending.
\end{itemize}


</details>

\textbf{Question 6:} What happens if you don't confirm the SNS subscription email?

<details>
<summary>Click to reveal answer</summary>

\textbf{Answer:} The subscription remains in "Pending confirmation" status and you will NOT receive any alarm notifications. The alarm will still evaluate and trigger, but emails won't be sent.

Always check spam folder and confirm subscriptions within 3 days. After 3 days, you may need to recreate the subscription.

</details>

\textbf{Question 7:} Your Free Tier includes 10 CloudWatch alarms. What happens if you create 11?

<details>
<summary>Click to reveal answer</summary>

\textbf{Answer:} You'll be charged \$0.10 per alarm per month for the 11th alarm (and any beyond that). With 11 alarms, cost would be \$0.10/month.

\textbf{Best Practice for Free Tier:}
\begin{itemize}
  \item Stay within 10 alarms
  \item Use AWS Budgets (free) for additional monitoring
  \item Combine multiple thresholds in fewer alarms
\end{itemize}


</details>

\textbf{Question 8:} How often should you check your Free Tier usage dashboard?

<details>
<summary>Click to reveal answer</summary>

\textbf{Answer:}
\begin{itemize}
  \item \textbf{Minimum:} Weekly
  \item \textbf{Recommended:} Every 2-3 days when actively learning
  \item \textbf{Best Practice:} Daily while running labs
  \item \textbf{Set Calendar Reminder:} Add recurring reminder to check dashboard
\end{itemize}


Free Tier dashboard shows current month usage vs. limits with visual progress bars. Catches issues before they become charges.

</details>

\subsubsection{Key Takeaways}


\begin{itemize}
  \item \textbf{Billing alerts are your safety net} - Set them up before doing anything else in AWS
  \item \textbf{Use both CloudWatch alarms and Budgets} - They complement each other
  \item \textbf{Always confirm SNS subscriptions} - Check spam folder if needed
  \item \textbf{Monitor Free Tier usage} - Make it a daily habit during learning phase
  \item \textbf{Be conservative with thresholds} - Better to get warned early than surprised by charges
  \item \textbf{Billing metrics are us-east-1 only} - Remember this for the exam
  \item \textbf{Budgets can track usage, not just costs} - Useful for monitoring Free Tier hours
  \item \textbf{Free Tier is per service} - 750 EC2 hours + 750 RDS hours, not combined
\end{itemize}


\subsubsection{Additional Resources}


\begin{itemize}
  \item \href{https://docs.aws.amazon.com/account-billing/}{AWS Billing and Cost Management Documentation}
  \item \href{https://docs.aws.amazon.com/AmazonCloudWatch/latest/monitoring/monitor\\textit{estimated\}charges\\textit{with\}cloudwatch.html}{CloudWatch Billing Metrics Guide}
  \item \href{https://docs.aws.amazon.com/cost-management/latest/userguide/budgets-best-practices.html}{AWS Budgets Best Practices}
  \item \href{https://aws.amazon.com/free/}{Understanding AWS Free Tier}
\end{itemize}


---

\subsection{Lab 2: IAM Users, Groups, Roles, and MFA}


\textbf{Duration:} 30 minutes
\textbf{Cost:} Free
\textbf{Difficulty:} Beginner

\subsubsection{Objective}


Understand IAM security best practices by creating users, groups, roles, and enabling MFA.

\subsubsection{Prerequisites}


\begin{itemize}
  \item AWS account with root access
  \item Smartphone with authenticator app (Google Authenticator, Authy, etc.)
\end{itemize}


\subsubsection{Step-by-Step Instructions}


\paragraph{Part 1: Secure Root Account with MFA}


\begin{enumerate}
  \item Navigate to \textbf{IAM} service in AWS Console
  \item Click \textbf{"Dashboard"} → Review security recommendations
  \item Click \textbf{"Add MFA"} for root account
  \item \textbf{MFA device name:} "root-mfa-device"
  \item Select \textbf{"Virtual MFA device"} → \textbf{"Next"}
  \item Install authenticator app on your phone:
\end{enumerate}

\begin{itemize}
  \item Google Authenticator (iOS/Android)
  \item Authy (iOS/Android)
  \item Microsoft Authenticator
\end{itemize}

\begin{enumerate}
  \item Click \textbf{"Show QR code"}
  \item Scan QR code with authenticator app
  \item Enter \textbf{two consecutive MFA codes} from app
  \item Click \textbf{"Add MFA"}
  \item \textbf{Verify:} MFA badge appears on dashboard
\end{enumerate}


\begin{important}
\textbf{Important:} Store root account credentials securely and use IAM users for daily tasks.
\end{important}


\paragraph{Part 2: Create IAM Admin User}


\begin{enumerate}
  \item In IAM, click \textbf{"Users"} → \textbf{"Create user"}
  \item \textbf{User name:} "admin-user"
  \item Select \textbf{"Provide user access to the AWS Management Console"}
  \item Choose \textbf{"I want to create an IAM user"}
  \item Password options:
\end{enumerate}

\begin{itemize}
  \item Custom password OR Autogenerated
  \item Uncheck "Users must create a new password at next sign-in"
\end{itemize}

\begin{enumerate}
  \item Click \textbf{"Next"}
  \item \textbf{Permissions options:} "Attach policies directly"
  \item Search and select \textbf{"AdministratorAccess"}
  \item Click \textbf{"Next"}
  \item Review and click \textbf{"Create user"}
  \item \textbf{Download .csv file} (contains credentials)
  \item \textbf{Copy console sign-in URL} (save for later)
\end{enumerate}


\paragraph{Part 3: Create IAM Groups}


\textbf{Create Developers Group:}

\begin{enumerate}
  \item Click \textbf{"User groups"} → \textbf{"Create group"}
  \item \textbf{Group name:} "Developers"
  \item \textbf{Attach permissions policies:}
\end{enumerate}

\begin{itemize}
  \item Search and select \textbf{"AmazonEC2ReadOnlyAccess"}
  \item Search and select \textbf{"AmazonS3FullAccess"}
\end{itemize}

\begin{enumerate}
  \item Click \textbf{"Create group"}
\end{enumerate}


\textbf{Create Administrators Group:}

\begin{enumerate}
  \item Click \textbf{"Create group"} again
  \item \textbf{Group name:} "Administrators"
  \item \textbf{Attach permissions policy:}
\end{enumerate}

\begin{itemize}
  \item Search and select \textbf{"AdministratorAccess"}
\end{itemize}

\begin{enumerate}
  \item Click \textbf{"Create group"}
\end{enumerate}


\paragraph{Part 4: Create Additional IAM Users}


\textbf{Create Developer User 1:}

\begin{enumerate}
  \item Click \textbf{"Users"} → \textbf{"Create user"}
  \item \textbf{User name:} "developer-1"
  \item Enable \textbf{console access}
  \item Set password (custom or autogenerated)
  \item Click \textbf{"Next"}
  \item \textbf{Add user to groups:} Select \textbf{"Developers"} group
  \item Click \textbf{"Next"} → \textbf{"Create user"}
\end{enumerate}


\textbf{Create Developer User 2:}

\begin{enumerate}
  \item Repeat above steps
  \item \textbf{User name:} "developer-2"
  \item Add to \textbf{"Developers"} group
  \item Create user
\end{enumerate}


\paragraph{Part 5: Create IAM Role for EC2}


\begin{enumerate}
  \item Click \textbf{"Roles"} → \textbf{"Create role"}
  \item \textbf{Trusted entity type:} "AWS service"
  \item \textbf{Use case:} Select \textbf{"EC2"}
  \item Click \textbf{"Next"}
  \item \textbf{Attach permissions:}
\end{enumerate}

\begin{itemize}
  \item Search and select \textbf{"AmazonS3ReadOnlyAccess"}
\end{itemize}

\begin{enumerate}
  \item Click \textbf{"Next"}
  \item \textbf{Role name:} "EC2-S3-ReadOnly-Role"
  \item \textbf{Description:} "Allows EC2 instances to read from S3"
  \item Click \textbf{"Create role"}
\end{enumerate}


\paragraph{Part 6: Test IAM Policies}


\begin{enumerate}
  \item \textbf{Sign out} from root account
  \item \textbf{Sign in} as "developer-1" using:
\end{enumerate}

\begin{itemize}
  \item Console sign-in URL (saved earlier)
  \item Username: developer-1
  \item Password: (as set)
\end{itemize}

\begin{enumerate}
  \item Try to access \textbf{S3} service (should work - full access)
  \item Try to create S3 bucket (should work)
  \item Try to access \textbf{IAM} service (should be denied - no permission)
  \item Try to view \textbf{EC2} instances (should work - read-only)
  \item Try to launch EC2 instance (should be denied - read-only)
  \item \textbf{Sign out}
\end{enumerate}


\subsubsection{Expected Outcomes}


\begin{itemize}
  \item Root account secured with MFA
  \item IAM admin user created with full permissions
  \item Two IAM groups created (Developers, Administrators)
  \item Two developer users created and added to Developers group
  \item IAM role created for EC2 to access S3
  \item Successfully tested permissions and restrictions
\end{itemize}


\subsubsection{Verification}


\begin{enumerate}
  \item \textbf{IAM Dashboard} shows:
\end{enumerate}

\begin{itemize}
  \item MFA enabled for root
  \item Multiple users created
  \item Groups with attached policies
  \item Role created
\end{itemize}

\begin{enumerate}
  \item \textbf{Sign-in test} as developer-1 confirms permission boundaries
\end{enumerate}


\subsubsection{Troubleshooting}


\textbf{Problem:} Can't sign in as IAM user
\textbf{Solution:} Use account-specific sign-in URL, not root login page

\textbf{Problem:} MFA setup fails
\textbf{Solution:} Ensure phone time is synchronized; try re-scanning QR code

\textbf{Problem:} Permission denied errors
\textbf{Solution:} Verify user is in correct group; check group policies attached

\subsubsection{Cleanup}


\begin{keypoint}
\textbf{Note:} Keep admin-user for future labs. Can delete developer users and groups if desired.
\end{keypoint}


\textbf{To delete users:}
\begin{enumerate}
  \item Select user → \textbf{"Delete"} → Confirm
\end{enumerate}


\textbf{To delete groups:}
\begin{enumerate}
  \item Remove all users from group first
  \item Select group → \textbf{"Delete"} → Confirm
\end{enumerate}


\textbf{To delete role:}
\begin{enumerate}
  \item Select role → \textbf{"Delete"} → Confirm
\end{enumerate}


---

\subsection{Lab 3: Launch and Configure EC2 Instance}


\textbf{Duration:} 45 minutes
\textbf{Cost:} Free (t2.micro/t3.micro in Free Tier)
\textbf{Difficulty:} Intermediate

\subsubsection{Objective}


Launch a web server on EC2, connect via SSH, create an AMI, and take snapshots.

\subsubsection{Prerequisites}


\begin{itemize}
  \item AWS account
  \item IAM user with EC2 permissions
  \item Basic command line knowledge
\end{itemize}


\subsubsection{Step-by-Step Instructions}


\paragraph{Part 1: Launch EC2 Instance}


\begin{enumerate}
  \item Navigate to \textbf{EC2} service
  \item Select region: \textbf{us-east-1} (or your preferred region)
  \item Click \textbf{"Launch instance"}
  \item \textbf{Name:} "MyWebServer"
  \item \textbf{Application and OS Images (AMI):}
\end{enumerate}

\begin{itemize}
  \item Select \textbf{"Amazon Linux 2023 AMI"}
  \item Verify "Free tier eligible" label
\end{itemize}

\begin{enumerate}
  \item \textbf{Instance type:}
\end{enumerate}

\begin{itemize}
  \item Select \textbf{"t2.micro"} or \textbf{"t3.micro"} (Free tier eligible)
\end{itemize}

\begin{enumerate}
  \item \textbf{Key pair:}
\end{enumerate}

\begin{itemize}
  \item Click \textbf{"Create new key pair"}
  \item \textbf{Key pair name:} "my-key-pair"
  \item \textbf{Key pair type:} RSA
  \item \textbf{Private key file format:}
  \item \textbf{.pem} (for Mac/Linux/Windows OpenSSH)
  \item \textbf{.ppk} (for Windows PuTTY)
  \item Click \textbf{"Create key pair"}
  \item \textbf{Save file securely} (you can't download again)
\end{itemize}


\paragraph{Part 2: Configure Network Settings}


\begin{enumerate}
  \item \textbf{Network settings:}
\end{enumerate}

\begin{itemize}
  \item Click \textbf{"Edit"}
  \item \textbf{VPC:} Default VPC
  \item \textbf{Subnet:} No preference
  \item \textbf{Auto-assign public IP:} Enable
\end{itemize}

\begin{enumerate}
  \item \textbf{Firewall (Security groups):}
\end{enumerate}

\begin{itemize}
  \item Select \textbf{"Create security group"}
  \item \textbf{Security group name:} "web-server-sg"
  \item \textbf{Description:} "Allow SSH and HTTP"
  \item \textbf{Inbound security group rules:}
  \item \textbf{Rule 1:}
  \item Type: SSH
  \item Protocol: TCP
  \item Port: 22
  \item Source: My IP
  \item Click \textbf{"Add security group rule"}
  \item \textbf{Rule 2:}
  \item Type: HTTP
  \item Protocol: TCP
  \item Port: 80
  \item Source: 0.0.0.0/0 (anywhere)
\end{itemize}


\paragraph{Part 3: Configure Storage and User Data}


\begin{enumerate}
  \item \textbf{Configure storage:}
\end{enumerate}

\begin{itemize}
  \item \textbf{Size:} 8 GiB (default)
  \item \textbf{Volume type:} gp3 (default)
  \item Keep other defaults
\end{itemize}

\begin{enumerate}
  \item \textbf{Expand "Advanced details"}
  \item Scroll to \textbf{"User data"}
  \item Paste the following script:
\end{enumerate}


\begin{lstlisting}[language=bash]
\#!/bin/bash
yum update -y
yum install -y httpd
systemctl start httpd
systemctl enable httpd
echo "<h1>Hello from AWS EC2</h1>" > /var/www/html/index.html
\end{lstlisting}

\begin{enumerate}
  \item Click \textbf{"Launch instance"}
  \item Wait for \textbf{"Successfully initiated launch"} message
  \item Click \textbf{"View all instances"}
\end{enumerate}


\paragraph{Part 4: Connect to Instance}


\textbf{Wait for instance to be ready:}
\begin{itemize}
  \item \textbf{Instance state:} Running
  \item \textbf{Status check:} 2/2 checks passed (may take 2-3 minutes)
\end{itemize}


\textbf{Get connection information:}
\begin{enumerate}
  \item Select your instance
  \item Copy \textbf{"Public IPv4 address"}
  \item Test web server: Open browser, navigate to \texttt{http://[public-ip]}
  \item You should see: \textbf{"Hello from AWS EC2"}
\end{enumerate}


\textbf{Connect via SSH (Mac/Linux/Windows OpenSSH):}

\begin{enumerate}
  \item Open terminal
  \item Navigate to directory with key pair:
\end{enumerate}

   \texttt{`}bash
   cd \textasciitilde{}/Downloads
   \texttt{`}
\begin{enumerate}
  \item Set correct permissions:
\end{enumerate}

   \texttt{`}bash
   chmod 400 my-key-pair.pem
   \texttt{`}
\begin{enumerate}
  \item Connect to instance:
\end{enumerate}

   \texttt{`}bash
   ssh -i my-key-pair.pem ec2-user@[public-ip-address]
   \texttt{`}
\begin{enumerate}
  \item Type \textbf{"yes"} to accept fingerprint
  \item You're now connected!
\end{enumerate}


\textbf{Connect via SSH (Windows PuTTY):}

\begin{enumerate}
  \item Open PuTTY
  \item \textbf{Host Name:} ec2-user@[public-ip]
  \item \textbf{Port:} 22
  \item \textbf{Connection → SSH → Auth:}
\end{enumerate}

\begin{itemize}
  \item Browse and select .ppk file
\end{itemize}

\begin{enumerate}
  \item Click \textbf{"Open"}
  \item Accept security alert
  \item You're connected!
\end{enumerate}


\textbf{Verify web server:}
\begin{lstlisting}[language=bash]
sudo systemctl status httpd
\end{lstlisting}

\paragraph{Part 5: Create AMI (Amazon Machine Image)}


\begin{enumerate}
  \item In EC2 Console, select your instance
  \item Click \textbf{"Actions"} → \textbf{"Image and templates"} → \textbf{"Create image"}
  \item \textbf{Image name:} "MyWebServer-AMI"
  \item \textbf{Image description:} "Web server with Apache installed"
  \item Keep other defaults
  \item Click \textbf{"Create image"}
  \item Go to \textbf{"AMIs"} in left navigation menu
  \item Wait for \textbf{Status:} "Available" (takes 2-5 minutes)
\end{enumerate}


\begin{keypoint}
\textbf{Note:} You can now launch new instances from this AMI with Apache pre-installed.
\end{keypoint}


\paragraph{Part 6: Create EBS Snapshot}


\begin{enumerate}
  \item Go to \textbf{"Volumes"} in EC2 left menu
  \item Select volume attached to your instance (check "Attachment information")
  \item Click \textbf{"Actions"} → \textbf{"Create snapshot"}
  \item \textbf{Description:} "WebServer-backup"
  \item \textbf{Tags:}
\end{enumerate}

\begin{itemize}
  \item Key: Name
  \item Value: WebServer-Snapshot
\end{itemize}

\begin{enumerate}
  \item Click \textbf{"Create snapshot"}
  \item Go to \textbf{"Snapshots"} to view status
  \item Wait for \textbf{Status:} "Completed"
\end{enumerate}


\subsubsection{Expected Outcomes}


\begin{itemize}
  \item EC2 instance running Amazon Linux 2023
  \item Apache web server installed and accessible via HTTP
  \item Successfully connected via SSH
  \item AMI created from running instance
  \item EBS snapshot created for backup
\end{itemize}


\subsubsection{Verification}


\begin{enumerate}
  \item \textbf{Web server accessible:} Visit \texttt{http://[public-ip]} → See "Hello from AWS EC2"
  \item \textbf{SSH connection works:} Able to connect and run commands
  \item \textbf{AMI created:} Shows in AMIs list with "Available" status
  \item \textbf{Snapshot created:} Shows in Snapshots list with "Completed" status
\end{enumerate}


\subsubsection{Troubleshooting}


\textbf{Problem:} Can't access web server (timeout)
\textbf{Solution:}
\begin{itemize}
  \item Verify security group allows HTTP (port 80) from 0.0.0.0/0
  \item Ensure instance is in "running" state
  \item Check status checks passed
  \item Verify you're using HTTP, not HTTPS
\end{itemize}


\textbf{Problem:} SSH connection refused
\textbf{Solution:}
\begin{itemize}
  \item Verify security group allows SSH (port 22) from your IP
  \item Check you're using correct key pair file
  \item Ensure using "ec2-user" as username
  \item Verify key file has correct permissions (400)
\end{itemize}


\textbf{Problem:} Permission denied (publickey)
\textbf{Solution:}
\begin{itemize}
  \item Verify using correct .pem file
  \item Check file permissions: \texttt{chmod 400 my-key-pair.pem}
  \item Ensure using correct username (ec2-user for Amazon Linux)
\end{itemize}


\textbf{Problem:} AMI creation fails
\textbf{Solution:}
\begin{itemize}
  \item Ensure instance is in "running" or "stopped" state
  \item Check you have sufficient EBS snapshot quota
\end{itemize}


\subsubsection{Cleanup (Important!)}


\begin{keypoint}
\textbf{Critical:} Always clean up to avoid charges after Free Tier expires.
\end{keypoint}


\textbf{Delete resources in this order:}

\begin{enumerate}
  \item \textbf{Terminate instance:}
\end{enumerate}

\begin{itemize}
  \item Select instance
  \item \textbf{"Instance state"} → \textbf{"Terminate instance"}
  \item Confirm termination
  \item Wait for state: "Terminated"
\end{itemize}


\begin{enumerate}
  \item \textbf{Delete snapshot:}
\end{enumerate}

\begin{itemize}
  \item Go to \textbf{"Snapshots"}
  \item Select your snapshot
  \item \textbf{"Actions"} → \textbf{"Delete snapshot"}
  \item Confirm deletion
\end{itemize}


\begin{enumerate}
  \item \textbf{Deregister AMI:}
\end{enumerate}

\begin{itemize}
  \item Go to \textbf{"AMIs"}
  \item Select your AMI
  \item \textbf{"Actions"} → \textbf{"Deregister AMI"}
  \item Confirm deregistration
\end{itemize}


\begin{enumerate}
  \item \textbf{Delete AMI snapshot:}
\end{enumerate}

\begin{itemize}
  \item Go to \textbf{"Snapshots"}
  \item Find snapshot created by AMI (check description)
  \item \textbf{"Actions"} → \textbf{"Delete snapshot"}
  \item Confirm deletion
\end{itemize}


\begin{enumerate}
  \item \textbf{Delete key pair (optional):}
\end{enumerate}

\begin{itemize}
  \item Go to \textbf{"Key Pairs"}
  \item Select key pair
  \item \textbf{"Actions"} → \textbf{"Delete"}
  \item Confirm deletion
\end{itemize}


---

\subsection{Lab 4: Amazon S3 Storage and Website Hosting}


\textbf{Duration:} 40 minutes
\textbf{Cost:} Free (within 5 GB Free Tier)
\textbf{Difficulty:} Intermediate

\subsubsection{Objective}


Master S3 storage features including bucket creation, static website hosting, versioning, lifecycle policies, and encryption.

\subsubsection{Prerequisites}


\begin{itemize}
  \item AWS account
  \item Basic HTML knowledge
  \item Text editor
\end{itemize}


\subsubsection{Step-by-Step Instructions}


\paragraph{Part 1: Create S3 Bucket}


\begin{enumerate}
  \item Navigate to \textbf{S3} service
  \item Click \textbf{"Create bucket"}
  \item \textbf{Bucket name:} "my-website-[yourname]-[random-numbers]"
\end{enumerate}

\begin{itemize}
  \item Must be globally unique
  \item Example: "my-website-john-12345"
  \item Only lowercase letters, numbers, hyphens
\end{itemize}

\begin{enumerate}
  \item \textbf{AWS Region:} Select your preferred region (us-east-1 recommended)
  \item \textbf{Object Ownership:} ACLs disabled (recommended)
  \item \textbf{Block Public Access settings:}
\end{enumerate}

\begin{itemize}
  \item \textbf{Uncheck} "Block all public access"
  \item \textbf{Check} acknowledgment box (needed for website hosting)
\end{itemize}

\begin{enumerate}
  \item \textbf{Bucket Versioning:} Enable
  \item \textbf{Tags:}
\end{enumerate}

\begin{itemize}
  \item Key: Project
  \item Value: Learning
\end{itemize}

\begin{enumerate}
  \item \textbf{Default encryption:} Server-side encryption with Amazon S3 managed keys (SSE-S3)
  \item Click \textbf{"Create bucket"}
\end{enumerate}


\paragraph{Part 2: Upload HTML Files}


\textbf{Create index.html file:}

\begin{enumerate}
  \item Open text editor
  \item Create file named \texttt{index.html}
  \item Paste the following content:
\end{enumerate}


\begin{lstlisting}[language=html]
<!DOCTYPE html>
<html>
<head>
    <title>My AWS Website</title>
    <style>
        body \{
            font-family: Arial, sans-serif;
            max-width: 800px;
            margin: 50px auto;
            padding: 20px;
        \}
        h1 \{ color: \#FF9900; \}
    </style>
</head>
<body>
    <h1>Welcome to My S3 Website</h1>
    <p>This website is hosted on Amazon S3!</p>
    <p>S3 provides durable, scalable object storage.</p>
</body>
</html>
\end{lstlisting}

\begin{enumerate}
  \item Save file
\end{enumerate}


\textbf{Create error.html file:}

\begin{enumerate}
  \item Create file named \texttt{error.html}
  \item Paste the following content:
\end{enumerate}


\begin{lstlisting}[language=html]
<!DOCTYPE html>
<html>
<head>
    <title>Error</title>
</head>
<body>
    <h1>404 - Page Not Found</h1>
    <p>The requested page doesn't exist.</p>
</body>
</html>
\end{lstlisting}

\begin{enumerate}
  \item Save file
\end{enumerate}


\textbf{Upload files to S3:}

\begin{enumerate}
  \item Click on your bucket name
  \item Click \textbf{"Upload"}
  \item Click \textbf{"Add files"}
  \item Select \texttt{index.html} and \texttt{error.html}
  \item Click \textbf{"Upload"}
  \item Wait for upload to complete
  \item Click \textbf{"Close"}
\end{enumerate}


\paragraph{Part 3: Enable Static Website Hosting}


\begin{enumerate}
  \item In your bucket, go to \textbf{"Properties"} tab
  \item Scroll to \textbf{"Static website hosting"}
  \item Click \textbf{"Edit"}
  \item \textbf{Static website hosting:} Enable
  \item \textbf{Hosting type:} "Host a static website"
  \item \textbf{Index document:} index.html
  \item \textbf{Error document:} error.html
  \item Click \textbf{"Save changes"}
  \item Scroll back to "Static website hosting"
  \item \textbf{Copy the "Bucket website endpoint" URL} (save for later)
\end{enumerate}


\paragraph{Part 4: Configure Bucket Policy for Public Access}


\begin{enumerate}
  \item Go to \textbf{"Permissions"} tab
  \item Scroll to \textbf{"Bucket policy"}
  \item Click \textbf{"Edit"}
  \item Paste the following policy (replace \texttt{YOUR-BUCKET-NAME} with your actual bucket name):
\end{enumerate}


\begin{lstlisting}[language=json]
\{
  "Version": "2012-10-17",
  "Statement": [
    \{
      "Sid": "PublicReadGetObject",
      "Effect": "Allow",
      "Principal": "*",
      "Action": "s3:GetObject",
      "Resource": "arn:aws:s3:::YOUR-BUCKET-NAME/*"
    \}
  ]
\}
\end{lstlisting}

\begin{enumerate}
  \item Click \textbf{"Save changes"}
  \item \textbf{Test website:} Open bucket website endpoint URL in browser
  \item You should see your "Welcome to My S3 Website" page!
\end{enumerate}


\paragraph{Part 5: Test Versioning}


\begin{enumerate}
  \item Edit \texttt{index.html} locally:
\end{enumerate}

\begin{itemize}
  \item Change heading to "Welcome to My Updated S3 Website"
  \item Add a line: \texttt{<p>This is version 2!</p>}
  \item Save file
\end{itemize}

\begin{enumerate}
  \item Upload new version to S3:
\end{enumerate}

\begin{itemize}
  \item Go to bucket → \textbf{"Upload"}
  \item Select modified \texttt{index.html}
  \item \textbf{"Upload"}
\end{itemize}

\begin{enumerate}
  \item View versions:
\end{enumerate}

\begin{itemize}
  \item In bucket, select \texttt{index.html}
  \item Click \textbf{"Versions"} tab
  \item You'll see multiple versions listed
\end{itemize}

\begin{enumerate}
  \item Click on older version to view/download
  \item To restore older version:
\end{enumerate}

\begin{itemize}
  \item Select older version
  \item Click \textbf{"Actions"} → \textbf{"Download"}
  \item Re-upload as new version
\end{itemize}


\paragraph{Part 6: Create Lifecycle Policy}


\begin{enumerate}
  \item Go to \textbf{"Management"} tab
  \item Click \textbf{"Create lifecycle rule"}
  \item \textbf{Lifecycle rule name:} "Archive-Old-Files"
  \item \textbf{Rule scope:} Apply to all objects in the bucket
  \item \textbf{Lifecycle rule actions} (check these):
\end{enumerate}

\begin{itemize}
  \item Transition current versions of objects between storage classes
  \item Expire current versions of objects
\end{itemize}

\begin{enumerate}
  \item \textbf{Transition current versions:}
\end{enumerate}

\begin{itemize}
  \item \textbf{Days after object creation:} 30
  \item \textbf{Storage class:} Standard-IA
  \item Click \textbf{"Add transition"}
  \item \textbf{Days:} 90
  \item \textbf{Storage class:} Glacier Flexible Retrieval
\end{itemize}

\begin{enumerate}
  \item \textbf{Expire current versions of objects:}
\end{enumerate}

\begin{itemize}
  \item \textbf{Days after object creation:} 365
\end{itemize}

\begin{enumerate}
  \item \textbf{Acknowledge warning} about costs
  \item Click \textbf{"Create rule"}
\end{enumerate}


\paragraph{Part 7: Enable Server-Side Encryption}


\begin{enumerate}
  \item Go to \textbf{"Properties"} tab
  \item Scroll to \textbf{"Default encryption"}
  \item Click \textbf{"Edit"}
  \item \textbf{Encryption type:} Server-side encryption with Amazon S3 managed keys (SSE-S3)
  \item \textbf{Bucket Key:} Enabled (reduces encryption costs)
  \item Click \textbf{"Save changes"}
\end{enumerate}


\subsubsection{Expected Outcomes}


\begin{itemize}
  \item S3 bucket created with unique name
  \item Static website hosting enabled and accessible
  \item HTML files uploaded and viewable via HTTP
  \item Versioning enabled and tested
  \item Lifecycle policy created for automatic archival
  \item Encryption enabled for security
\end{itemize}


\subsubsection{Verification}


\begin{enumerate}
  \item \textbf{Website accessible:} Visit bucket endpoint URL → See your website
  \item \textbf{Versioning works:} Multiple versions visible in Versions tab
  \item \textbf{Lifecycle rule created:} Visible in Management tab
  \item \textbf{Encryption enabled:} Shows in Properties tab
\end{enumerate}


\subsubsection{Troubleshooting}


\textbf{Problem:} 403 Forbidden error when accessing website
\textbf{Solution:}
\begin{itemize}
  \item Verify bucket policy allows public read (s3:GetObject)
  \item Check "Block all public access" is OFF
  \item Ensure bucket policy has correct bucket name
  \item Verify ARN includes /* at end
\end{itemize}


\textbf{Problem:} 404 Not Found error
\textbf{Solution:}
\begin{itemize}
  \item Ensure index.html is uploaded to bucket root
  \item Check filename is exactly "index.html" (case-sensitive)
  \item Verify static website hosting is enabled
\end{itemize}


\textbf{Problem:} Can't upload files
\textbf{Solution:}
\begin{itemize}
  \item Check you have s3:PutObject permissions
  \item Verify bucket exists and you're in correct region
  \item Try uploading smaller files first
\end{itemize}


\subsubsection{Cleanup}


\begin{important}
\textbf{Important:} Delete all objects before deleting bucket.
\end{important}


\begin{enumerate}
  \item \textbf{Empty bucket:}
\end{enumerate}

\begin{itemize}
  \item Select your bucket
  \item Click \textbf{"Empty"}
  \item Type \textbf{"permanently delete"}
  \item Click \textbf{"Empty"}
  \item Wait for completion
\end{itemize}


\begin{enumerate}
  \item \textbf{Delete bucket:}
\end{enumerate}

\begin{itemize}
  \item Select your bucket
  \item Click \textbf{"Delete"}
  \item Type bucket name to confirm
  \item Click \textbf{"Delete bucket"}
\end{itemize}


---

\subsection{Lab 5: VPC, Subnets, and Network Configuration}


\textbf{Duration:} 60 minutes
\textbf{Cost:} Free (NAT Gateway excluded to avoid charges)
\textbf{Difficulty:} Advanced

\subsubsection{Objective}


Build a custom VPC with public and private subnets, configure routing, security groups, and NACLs.

\subsubsection{Prerequisites}


\begin{itemize}
  \item Understanding of networking basics (IP addresses, subnets, CIDR)
  \item Completed Lab 3 (EC2 knowledge required)
\end{itemize}


\subsubsection{Step-by-Step Instructions}


\paragraph{Part 1: Create VPC}


\begin{enumerate}
  \item Navigate to \textbf{VPC} service
  \item Click \textbf{"Create VPC"}
  \item \textbf{Resources to create:} "VPC and more" (creates VPC with subnets automatically)
  \item \textbf{Name tag auto-generation:} "MyVPC"
  \item \textbf{IPv4 CIDR block:} 10.0.0.0/16
\end{enumerate}

\begin{itemize}
  \item Provides 65,536 IP addresses
\end{itemize}

\begin{enumerate}
  \item \textbf{IPv6 CIDR block:} No IPv6 CIDR block
  \item \textbf{Tenancy:} Default
  \item \textbf{Number of Availability Zones:} 2
  \item \textbf{Number of public subnets:} 2
  \item \textbf{Number of private subnets:} 2
  \item \textbf{NAT gateways:} None (to stay in Free Tier)
\end{enumerate}

\begin{itemize}
  \item \textbf{Important:} NAT Gateway costs \$0.045/hour
\end{itemize}

\begin{enumerate}
  \item \textbf{VPC endpoints:} None
  \item \textbf{DNS options:}
\end{enumerate}

\begin{itemize}
  \item Enable DNS hostnames: Yes
  \item Enable DNS resolution: Yes
\end{itemize}

\begin{enumerate}
  \item Click \textbf{"Create VPC"}
  \item Wait for creation (takes 1-2 minutes)
  \item Click \textbf{"View VPC"}
\end{enumerate}


\paragraph{Part 2: Review VPC Components}


\textbf{Review VPC:}
\begin{enumerate}
  \item Go to \textbf{"Your VPCs"}
  \item Verify VPC created with CIDR 10.0.0.0/16
  \item Note VPC ID (starts with vpc-)
\end{enumerate}


\textbf{Review Subnets:}
\begin{enumerate}
  \item Go to \textbf{"Subnets"}
  \item You should see 4 subnets:
\end{enumerate}

\begin{itemize}
  \item \textbf{Public subnet 1:} 10.0.0.0/20 (AZ a) - 4096 IPs
  \item \textbf{Public subnet 2:} 10.0.16.0/20 (AZ b) - 4096 IPs
  \item \textbf{Private subnet 1:} 10.0.128.0/20 (AZ a) - 4096 IPs
  \item \textbf{Private subnet 2:} 10.0.144.0/20 (AZ b) - 4096 IPs
\end{itemize}


\textbf{Review Internet Gateway:}
\begin{enumerate}
  \item Go to \textbf{"Internet Gateways"}
  \item Verify IGW created and attached to your VPC
  \item Note IGW ID (starts with igw-)
\end{enumerate}


\textbf{Review Route Tables:}
\begin{enumerate}
  \item Go to \textbf{"Route Tables"}
  \item You should see:
\end{enumerate}

\begin{itemize}
  \item \textbf{Public route table:}
  \item Local route: 10.0.0.0/16 → local
  \item Internet route: 0.0.0.0/0 → igw-xxx
  \item Associated with public subnets
  \item \textbf{Private route tables:}
  \item Local route only: 10.0.0.0/16 → local
  \item Associated with private subnets
\end{itemize}


\paragraph{Part 3: Create Security Groups}


\textbf{Create Web Server Security Group:}

\begin{enumerate}
  \item Go to \textbf{"Security Groups"}
  \item Click \textbf{"Create security group"}
  \item \textbf{Security group name:} "WebServer-SG"
  \item \textbf{Description:} "Allow HTTP and SSH"
  \item \textbf{VPC:} Select your VPC (MyVPC)
  \item \textbf{Inbound rules:}
\end{enumerate}

\begin{itemize}
  \item Click \textbf{"Add rule"}
  \item Type: SSH
  \item Source: My IP
  \item Click \textbf{"Add rule"}
  \item Type: HTTP
  \item Source: 0.0.0.0/0 (anywhere IPv4)
\end{itemize}

\begin{enumerate}
  \item \textbf{Outbound rules:} Leave default (allows all outbound)
  \item \textbf{Tags:}
\end{enumerate}

\begin{itemize}
  \item Key: Name
  \item Value: WebServer-SG
\end{itemize}

\begin{enumerate}
  \item Click \textbf{"Create security group"}
\end{enumerate}


\textbf{Create Database Security Group:}

\begin{enumerate}
  \item Click \textbf{"Create security group"}
  \item \textbf{Security group name:} "Database-SG"
  \item \textbf{Description:} "Allow MySQL from WebServer"
  \item \textbf{VPC:} Select your VPC (MyVPC)
  \item \textbf{Inbound rules:}
\end{enumerate}

\begin{itemize}
  \item Click \textbf{"Add rule"}
  \item Type: MySQL/Aurora
  \item Port: 3306
  \item Source: Custom
  \item Search and select "WebServer-SG"
\end{itemize}

\begin{enumerate}
  \item \textbf{Outbound rules:} Leave default
  \item Click \textbf{"Create security group"}
\end{enumerate}


\begin{keypoint}
\textbf{Explanation:} Database-SG only allows MySQL connections from instances with WebServer-SG, implementing principle of least privilege.
\end{keypoint}


\paragraph{Part 4: Launch EC2 Instance in Custom VPC}


\begin{enumerate}
  \item Go to \textbf{EC2} service
  \item Click \textbf{"Launch instance"}
  \item \textbf{Name:} "VPC-Test-Instance"
  \item \textbf{AMI:} Amazon Linux 2023 AMI (Free Tier)
  \item \textbf{Instance type:} t2.micro
  \item \textbf{Key pair:} Use existing or create new
  \item \textbf{Network settings:}
\end{enumerate}

\begin{itemize}
  \item Click \textbf{"Edit"}
  \item \textbf{VPC:} Select MyVPC
  \item \textbf{Subnet:} Select public subnet (10.0.0.0/20 or 10.0.16.0/20)
  \item \textbf{Auto-assign public IP:} Enable
  \item \textbf{Firewall:} Select existing security group
  \item Select \textbf{WebServer-SG}
\end{itemize}

\begin{enumerate}
  \item Keep other defaults
  \item Click \textbf{"Launch instance"}
  \item Wait for instance to reach "Running" state
  \item \textbf{Verify:} Instance has public IP and you can SSH into it
\end{enumerate}


\paragraph{Part 5: Create Network ACL (NACL)}


\begin{enumerate}
  \item Go to \textbf{"Network ACLs"} in VPC
  \item Click \textbf{"Create network ACL"}
  \item \textbf{Name:} "Custom-NACL"
  \item \textbf{VPC:} Select MyVPC
  \item Click \textbf{"Create network ACL"}
\end{enumerate}


\textbf{Configure Inbound Rules:}

\begin{enumerate}
  \item Select your NACL
  \item Go to \textbf{"Inbound rules"} tab
  \item Click \textbf{"Edit inbound rules"}
  \item Click \textbf{"Add new rule"} and add:
\end{enumerate}

\begin{itemize}
  \item \textbf{Rule 100:}
  \item Type: HTTP (80)
  \item Source: 0.0.0.0/0
  \item Allow
  \item \textbf{Rule 110:}
  \item Type: SSH (22)
  \item Source: 0.0.0.0/0
  \item Allow
  \item \textbf{Rule 120:}
  \item Type: Custom TCP
  \item Port range: 1024-65535 (ephemeral ports)
  \item Source: 0.0.0.0/0
  \item Allow
  \item \textbf{Rule \textbackslash{}* (default):} All traffic, Deny
\end{itemize}

\begin{enumerate}
  \item Click \textbf{"Save changes"}
\end{enumerate}


\textbf{Configure Outbound Rules:}

\begin{enumerate}
  \item Go to \textbf{"Outbound rules"} tab
  \item Click \textbf{"Edit outbound rules"}
  \item Add rules:
\end{enumerate}

\begin{itemize}
  \item \textbf{Rule 100:}
  \item Type: HTTP (80)
  \item Destination: 0.0.0.0/0
  \item Allow
  \item \textbf{Rule 110:}
  \item Type: HTTPS (443)
  \item Destination: 0.0.0.0/0
  \item Allow
  \item \textbf{Rule 120:}
  \item Type: Custom TCP
  \item Port range: 1024-65535
  \item Destination: 0.0.0.0/0
  \item Allow
\end{itemize}

\begin{enumerate}
  \item Click \textbf{"Save changes"}
\end{enumerate}


\textbf{Associate NACL with Subnet (Optional):}

\begin{enumerate}
  \item Go to \textbf{"Subnet associations"} tab
  \item Click \textbf{"Edit subnet associations"}
  \item Select a public subnet
  \item Click \textbf{"Save changes"}
\end{enumerate}


\begin{keypoint}
\textbf{Note:} Default NACL allows all traffic. Custom NACLs deny all traffic by default.
\end{keypoint}


\subsubsection{Expected Outcomes}


\begin{itemize}
  \item Custom VPC created with CIDR 10.0.0.0/16
  \item 2 public subnets and 2 private subnets across 2 AZs
  \item Internet Gateway attached and routes configured
  \item Security groups created for web server and database
  \item EC2 instance launched in public subnet
  \item Custom NACL created and configured
\end{itemize}


\subsubsection{Verification}


\begin{enumerate}
  \item \textbf{VPC exists:} Shows in "Your VPCs" with correct CIDR
  \item \textbf{Subnets created:} 4 subnets visible with correct CIDR blocks
  \item \textbf{Routing works:} EC2 instance in public subnet has internet access
  \item \textbf{Security groups work:} Can SSH to instance, HTTP accessible
  \item \textbf{NACLs configured:} Rules visible in NACL
\end{enumerate}


\subsubsection{Troubleshooting}


\textbf{Problem:} Can't create VPC
\textbf{Solution:}
\begin{itemize}
  \item Check you haven't exceeded VPC limit (5 per region default)
  \item Verify CIDR block doesn't overlap with existing VPCs
\end{itemize}


\textbf{Problem:} EC2 instance has no internet access
\textbf{Solution:}
\begin{itemize}
  \item Verify instance in public subnet
  \item Check route table has route to IGW (0.0.0.0/0 → igw-xxx)
  \item Ensure auto-assign public IP enabled
  \item Verify NACL allows traffic
\end{itemize}


\textbf{Problem:} Can't SSH to instance
\textbf{Solution:}
\begin{itemize}
  \item Check security group allows SSH from your IP
  \item Verify NACL allows SSH and ephemeral ports
  \item Ensure instance has public IP
  \item Check route table configuration
\end{itemize}


\subsubsection{Cleanup}


\textbf{Delete resources in order:}

\begin{enumerate}
  \item \textbf{Terminate EC2 instance:}
\end{enumerate}

\begin{itemize}
  \item Go to EC2 → Instances
  \item Select instance → Terminate
\end{itemize}


\begin{enumerate}
  \item \textbf{Delete custom security groups:}
\end{enumerate}

\begin{itemize}
  \item Go to VPC → Security Groups
  \item Select custom SGs → Delete
  \item Note: Default SG cannot be deleted
\end{itemize}


\begin{enumerate}
  \item \textbf{Delete custom NACLs:}
\end{enumerate}

\begin{itemize}
  \item Go to VPC → Network ACLs
  \item Disassociate from subnets first
  \item Select custom NACL → Delete
  \item Note: Default NACL cannot be deleted
\end{itemize}


\begin{enumerate}
  \item \textbf{Delete VPC:}
\end{enumerate}

\begin{itemize}
  \item Go to VPC → Your VPCs
  \item Select your VPC → Delete VPC
  \item This will delete:
  \item Subnets
  \item Route tables (except default)
  \item Internet Gateway
  \item VPC itself
  \item Confirm deletion
\end{itemize}


---

\subsection{Lab 6: Amazon RDS Database}


\textbf{Duration:} 30 minutes
\textbf{Cost:} Free (db.t3.micro or db.t4g.micro in Free Tier)
\textbf{Difficulty:} Intermediate

\subsubsection{Objective}


Launch a managed MySQL database using Amazon RDS.

\subsubsection{Prerequisites}


\begin{itemize}
  \item Basic understanding of relational databases
  \item Completed Lab 3 (EC2 knowledge) and Lab 5 (VPC knowledge)
\end{itemize}


\subsubsection{Step-by-Step Instructions}


\paragraph{Part 1: Create RDS Database}


\begin{enumerate}
  \item Navigate to \textbf{RDS} service
  \item Click \textbf{"Create database"}
  \item \textbf{Database creation method:} Standard create
  \item \textbf{Engine options:}
\end{enumerate}

\begin{itemize}
  \item Engine type: MySQL
  \item Version: MySQL 8.0.xx (latest)
\end{itemize}

\begin{enumerate}
  \item \textbf{Templates:} \textbf{Free tier}
\end{enumerate}

\begin{itemize}
  \item \textbf{Important:} This automatically configures Free Tier eligible options
\end{itemize}

\begin{enumerate}
  \item \textbf{Settings:}
\end{enumerate}

\begin{itemize}
  \item \textbf{DB instance identifier:} "mydatabase"
  \item \textbf{Master username:} admin
  \item \textbf{Credentials management:} Self managed
  \item \textbf{Master password:} Create secure password (minimum 8 characters)
  \item \textbf{Confirm password:} Re-enter password
  \item \textbf{Save password securely!}
\end{itemize}


\paragraph{Part 2: Configure Instance}


\begin{enumerate}
  \item \textbf{DB instance class:}
\end{enumerate}

\begin{itemize}
  \item Burstable classes (includes t classes)
  \item db.t3.micro or db.t4g.micro (Free Tier eligible)
\end{itemize}

\begin{enumerate}
  \item \textbf{Storage:}
\end{enumerate}

\begin{itemize}
  \item Storage type: General Purpose SSD (gp3)
  \item Allocated storage: 20 GiB
  \item \textbf{Uncheck} "Enable storage autoscaling" (to control costs)
\end{itemize}

\begin{enumerate}
  \item \textbf{Storage autoscaling:} Disabled
\end{enumerate}


\paragraph{Part 3: Configure Connectivity}


\begin{enumerate}
  \item \textbf{Compute resource:}
\end{enumerate}

\begin{itemize}
  \item Don't connect to an EC2 compute resource
\end{itemize}

\begin{enumerate}
  \item \textbf{Network type:} IPv4
  \item \textbf{Virtual private cloud (VPC):}
\end{enumerate}

\begin{itemize}
  \item Select Default VPC (or custom VPC if you have one)
\end{itemize}

\begin{enumerate}
  \item \textbf{DB subnet group:} Default
  \item \textbf{Public access:} \textbf{No} (recommended for security)
\end{enumerate}

\begin{itemize}
  \item Database won't have public IP
  \item Only accessible from EC2 in same VPC
\end{itemize}

\begin{enumerate}
  \item \textbf{VPC security group:}
\end{enumerate}

\begin{itemize}
  \item Choose existing
  \item Create new
  \item \textbf{Name:} "rds-mysql-sg"
\end{itemize}

\begin{enumerate}
  \item \textbf{Availability Zone:} No preference
  \item \textbf{Database port:} 3306 (default)
\end{enumerate}


\paragraph{Part 4: Additional Configuration}


\begin{enumerate}
  \item Expand \textbf{"Additional configuration"}
  \item \textbf{Database options:}
\end{enumerate}

\begin{itemize}
  \item \textbf{Initial database name:} "mydb"
  \item This creates a database automatically
\end{itemize}

\begin{enumerate}
  \item \textbf{Backup:}
\end{enumerate}

\begin{itemize}
  \item \textbf{Uncheck} "Enable automated backups" (to stay in Free Tier)
  \item Free Tier allows backups, but to be safe disable
\end{itemize}

\begin{enumerate}
  \item \textbf{Encryption:}
\end{enumerate}

\begin{itemize}
  \item \textbf{Uncheck} "Enable encryption" (optional, for simplicity)
  \item In production, always enable encryption
\end{itemize}

\begin{enumerate}
  \item \textbf{Monitoring:}
\end{enumerate}

\begin{itemize}
  \item \textbf{Uncheck} "Enable Enhanced monitoring" (to avoid charges)
\end{itemize}

\begin{enumerate}
  \item \textbf{Maintenance:}
\end{enumerate}

\begin{itemize}
  \item Keep defaults
\end{itemize}

\begin{enumerate}
  \item Click \textbf{"Create database"}
  \item Wait 5-10 minutes for database creation
  \item \textbf{Status} will change: Creating → Backing up → Available
\end{enumerate}


\paragraph{Part 5: Review Database Details}


\begin{enumerate}
  \item Once status is \textbf{"Available"}, click on database name
  \item \textbf{Connectivity \& security} tab:
\end{enumerate}

\begin{itemize}
  \item Note \textbf{Endpoint} (e.g., mydatabase.xxxxx.us-east-1.rds.amazonaws.com)
  \item Note \textbf{Port:} 3306
  \item \textbf{Security group:} Click to view rules
\end{itemize}

\begin{enumerate}
  \item \textbf{Configuration} tab:
\end{enumerate}

\begin{itemize}
  \item Verify DB instance class, storage, and version
\end{itemize}


\paragraph{Part 6: Connect to RDS (Requires EC2 in Same VPC)}


\textbf{Launch EC2 instance (if you don't have one):}

\begin{enumerate}
  \item Go to EC2 → Launch instance
  \item Use same VPC as RDS
  \item Select public subnet
  \item Use Amazon Linux 2023 AMI
  \item Launch and SSH into instance
\end{enumerate}


\textbf{Install MySQL client on EC2:}

\begin{lstlisting}[language=bash]
sudo yum update -y
sudo yum install -y mariadb105
\end{lstlisting}

\textbf{Update RDS Security Group:}

\begin{enumerate}
  \item Go to VPC → Security Groups
  \item Select RDS security group (rds-mysql-sg)
  \item Edit inbound rules
  \item Add rule:
\end{enumerate}

\begin{itemize}
  \item Type: MySQL/Aurora
  \item Port: 3306
  \item Source: Security group of your EC2 instance
\end{itemize}

\begin{enumerate}
  \item Save rules
\end{enumerate}


\textbf{Connect to RDS from EC2:}

\begin{lstlisting}[language=bash]
mysql -h mydatabase.xxxxx.us-east-1.rds.amazonaws.com -u admin -p
\end{lstlisting}

Replace with your actual RDS endpoint.

Enter password when prompted.

\textbf{Test database:}

\begin{lstlisting}[language=sql]
SHOW DATABASES;
USE mydb;
CREATE TABLE users (id INT, name VARCHAR(50));
INSERT INTO users VALUES (1, 'John Doe');
INSERT INTO users VALUES (2, 'Jane Smith');
SELECT * FROM users;
\end{lstlisting}

Expected output:
\begin{verbatim}
+------+------------+
| id   | name       |
+------+------------+
|    1 | John Doe   |
|    2 | Jane Smith |
+------+------------+
\end{verbatim}

\textbf{Exit MySQL:}
\begin{lstlisting}[language=sql]
exit;
\end{lstlisting}

\subsubsection{Expected Outcomes}


\begin{itemize}
  \item RDS MySQL database created and running
  \item Database accessible from EC2 instance in same VPC
  \item Successfully connected and executed SQL commands
  \item Test table created with sample data
\end{itemize}


\subsubsection{Verification}


\begin{enumerate}
  \item \textbf{RDS shows "Available" status}
  \item \textbf{Can connect from EC2} using MySQL client
  \item \textbf{SQL commands execute successfully}
  \item \textbf{No public access} (more secure configuration)
\end{enumerate}


\subsubsection{Troubleshooting}


\textbf{Problem:} Can't connect to RDS from EC2
\textbf{Solution:}
\begin{itemize}
  \item Verify both in same VPC
  \item Check RDS security group allows MySQL (3306) from EC2 security group
  \item Ensure using correct endpoint and credentials
  \item Verify RDS status is "Available"
\end{itemize}


\textbf{Problem:} Access denied error
\textbf{Solution:}
\begin{itemize}
  \item Double-check username (admin) and password
  \item Ensure password entered correctly (case-sensitive)
  \item Check user exists in database
\end{itemize}


\textbf{Problem:} Connection timeout
\textbf{Solution:}
\begin{itemize}
  \item Verify security group rules
  \item Check EC2 and RDS in same VPC
  \item Ensure RDS is not publicly accessible (can't connect from outside VPC)
  \item Check network ACLs
\end{itemize}


\subsubsection{Cleanup}


\begin{keypoint}
\textbf{Critical:} RDS instances incur charges if running beyond Free Tier hours (750 hours/month).
\end{keypoint}


\begin{enumerate}
  \item Go to \textbf{RDS} console
  \item Select your database
  \item Click \textbf{"Actions"} → \textbf{"Delete"}
  \item Delete options:
\end{enumerate}

\begin{itemize}
  \item \textbf{Uncheck} "Create final snapshot" (for lab purposes)
  \item \textbf{Uncheck} "Retain automated backups"
  \item \textbf{Check} "I acknowledge that upon instance deletion..."
\end{itemize}

\begin{enumerate}
  \item Type \textbf{"delete me"} to confirm
  \item Click \textbf{"Delete"}
  \item Deletion takes 2-5 minutes
  \item Verify database removed from list
\end{enumerate}


---

\subsection{Lab 7: CloudWatch Monitoring and Alarms}


\textbf{Duration:} 25 minutes
\textbf{Cost:} Free (within Free Tier limits)
\textbf{Difficulty:} Beginner

\subsubsection{Objective}


Monitor EC2 instances using CloudWatch metrics and create alarms for notifications.

\subsubsection{Prerequisites}


\begin{itemize}
  \item Running EC2 instance (from Lab 3 or new instance)
  \item Email address for notifications
\end{itemize}


\subsubsection{Step-by-Step Instructions}


\paragraph{Part 1: Launch EC2 Instance (if needed)}


\begin{enumerate}
  \item Launch t2.micro EC2 instance if you don't have one
  \item Wait for instance to reach "Running" state
  \item Note instance ID
\end{enumerate}


\paragraph{Part 2: Explore CloudWatch Metrics}


\begin{enumerate}
  \item Navigate to \textbf{CloudWatch} service
  \item Click \textbf{"All metrics"} in left menu
  \item Click \textbf{"EC2"}
  \item Click \textbf{"Per-Instance Metrics"}
  \item Search for your instance ID
  \item Select metrics:
\end{enumerate}

\begin{itemize}
  \item \textbf{CPUUtilization}
  \item \textbf{NetworkIn}
  \item \textbf{NetworkOut}
\end{itemize}

\begin{enumerate}
  \item View graphed metrics
  \item Change time range (1 hour, 3 hours, 1 day)
  \item Change period (1 minute, 5 minutes)
\end{enumerate}


\paragraph{Part 3: Create CloudWatch Alarm}


\begin{enumerate}
  \item Select \textbf{"CPUUtilization"} metric (checkbox)
  \item Click \textbf{"Actions"} → \textbf{"Create alarm"}
  \item \textbf{Metric and conditions:}
\end{enumerate}

\begin{itemize}
  \item Metric name: CPUUtilization
  \item Statistic: Average
  \item Period: 5 minutes
\end{itemize}

\begin{enumerate}
  \item \textbf{Conditions:}
\end{enumerate}

\begin{itemize}
  \item Threshold type: Static
  \item Whenever CPUUtilization is: Greater
  \item than: \textbf{70}
\end{itemize}

\begin{enumerate}
  \item Click \textbf{"Next"}
\end{enumerate}


\paragraph{Part 4: Configure SNS Notification}


\begin{enumerate}
  \item \textbf{Alarm state trigger:} In alarm
  \item \textbf{SNS topic:}
\end{enumerate}

\begin{itemize}
  \item Create new topic
  \item \textbf{Topic name:} "EC2-Alerts"
  \item \textbf{Email endpoints:} Enter your email address
\end{itemize}

\begin{enumerate}
  \item Click \textbf{"Create topic"}
  \item Click \textbf{"Next"}
  \item \textbf{Alarm name:} "High-CPU-Alert"
  \item \textbf{Alarm description:} "Alert when EC2 CPU exceeds 70\%"
  \item Click \textbf{"Next"}
  \item Review settings
  \item Click \textbf{"Create alarm"}
  \item \textbf{Check email} and click confirmation link in SNS subscription email
\end{enumerate}


\paragraph{Part 5: Test Alarm (Optional)}


\begin{keypoint}
\textbf{Warning:} This will stress your CPU. Only do if you want to test.
\end{keypoint}


\textbf{SSH into EC2 instance:}

\begin{lstlisting}[language=bash]
ssh -i your-key.pem ec2-user@[public-ip]
\end{lstlisting}

\textbf{Install stress tool:}

\begin{lstlisting}[language=bash]
sudo yum install -y stress
\end{lstlisting}

\textbf{Run CPU stress test:}

\begin{lstlisting}[language=bash]
stress --cpu 2 --timeout 300
\end{lstlisting}

This runs for 5 minutes (300 seconds).

\textbf{Monitor alarm:}

\begin{enumerate}
  \item Go to CloudWatch → Alarms
  \item Wait 5-10 minutes
  \item Alarm state will change: OK → In alarm
  \item You'll receive email notification
  \item After stress test completes, alarm returns to OK state
\end{enumerate}


\paragraph{Part 6: View CloudWatch Logs}


\begin{enumerate}
  \item In CloudWatch, go to \textbf{"Logs"} → \textbf{"Log groups"}
  \item Click \textbf{"Create log group"}
  \item \textbf{Log group name:} "/aws/my-application"
  \item Click \textbf{"Create"}
  \item Click on log group to explore
  \item Note: No logs yet (need to configure application to send logs)
\end{enumerate}


\textbf{Explore existing log groups:}
\begin{itemize}
  \item Look for /aws/lambda/, /aws/rds/, etc.
  \item Click on log group → Log streams → View logs
\end{itemize}


\subsubsection{Expected Outcomes}


\begin{itemize}
  \item CloudWatch metrics visible for EC2 instance
  \item CPU alarm created with 70\% threshold
  \item SNS topic created and email subscription confirmed
  \item Email notification received when alarm triggered (if tested)
  \item Log group created
\end{itemize}


\subsubsection{Verification}


\begin{enumerate}
  \item \textbf{Alarm visible} in CloudWatch → Alarms
  \item \textbf{SNS subscription confirmed} (check email)
  \item \textbf{Metrics displaying} in graphs
  \item \textbf{Alarm triggers correctly} (if stress test performed)
\end{enumerate}


\subsubsection{Troubleshooting}


\textbf{Problem:} No metrics showing for EC2
\textbf{Solution:}
\begin{itemize}
  \item Wait 5-10 minutes after instance launch
  \item Verify instance is running
  \item Check correct region selected
  \item Refresh page
\end{itemize}


\textbf{Problem:} Email notification not received
\textbf{Solution:}
\begin{itemize}
  \item Check spam/junk folder
  \item Verify email address entered correctly
  \item Resend confirmation from SNS console
  \item Check SNS subscription status (should be "Confirmed")
\end{itemize}


\textbf{Problem:} Alarm not triggering
\textbf{Solution:}
\begin{itemize}
  \item Verify CPU is actually exceeding 70\%
  \item Check alarm configuration and threshold
  \item Wait for evaluation period (5 minutes)
  \item Review alarm history in details
\end{itemize}


\subsubsection{Cleanup}


\begin{enumerate}
  \item \textbf{Delete CloudWatch alarm:}
\end{enumerate}

\begin{itemize}
  \item Go to CloudWatch → Alarms
  \item Select alarm → Actions → Delete
  \item Confirm deletion
\end{itemize}


\begin{enumerate}
  \item \textbf{Delete SNS topic:}
\end{enumerate}

\begin{itemize}
  \item Go to SNS → Topics
  \item Select topic → Delete
  \item Type "delete me"
  \item Confirm deletion
\end{itemize}


\begin{enumerate}
  \item \textbf{Delete log group:}
\end{enumerate}

\begin{itemize}
  \item Go to CloudWatch → Log groups
  \item Select log group → Actions → Delete
  \item Confirm deletion
\end{itemize}


\begin{enumerate}
  \item \textbf{Terminate EC2 instance:}
\end{enumerate}

\begin{itemize}
  \item Go to EC2 → Instances
  \item Select instance → Instance state → Terminate
\end{itemize}


---

\subsection{Lab 8: AWS Cost Management Tools}


\textbf{Duration:} 30 minutes
\textbf{Cost:} Free
\textbf{Difficulty:} Beginner

\subsubsection{Objective}


Explore AWS billing and cost management tools including Pricing Calculator, Cost Explorer, Budgets, and Trusted Advisor.

\subsubsection{Prerequisites}


\begin{itemize}
  \item AWS account with some usage (even minimal)
  \item Billing access enabled
\end{itemize}


\subsubsection{Step-by-Step Instructions}


\paragraph{Part 1: AWS Pricing Calculator}


\begin{enumerate}
  \item Open browser and visit \href{https://calculator.aws}{https://calculator.aws}
  \item Click \textbf{"Create estimate"}
  \item \textbf{Add EC2:}
\end{enumerate}

\begin{itemize}
  \item Search for "EC2"
  \item Click \textbf{"Configure"}
  \item \textbf{Region:} Select us-east-1
  \item \textbf{Quick estimate:}
  \item Number of instances: 10
  \item Instance type: t3.medium
  \item \textbf{Pricing model:} On-Demand
  \item Review monthly cost estimate
  \item Click \textbf{"Add to my estimate"}
\end{itemize}


\begin{enumerate}
  \item \textbf{Add S3:}
\end{enumerate}

\begin{itemize}
  \item Search for "S3"
  \item Click \textbf{"Configure"}
  \item \textbf{S3 Standard storage:} 1000 GB
  \item \textbf{PUT/COPY/POST requests:} 100,000
  \item \textbf{GET requests:} 1,000,000
  \item Review cost
  \item Click \textbf{"Add to my estimate"}
\end{itemize}


\begin{enumerate}
  \item \textbf{Add RDS:}
\end{enumerate}

\begin{itemize}
  \item Search for "RDS"
  \item Click \textbf{"Configure"}
  \item \textbf{Database engine:} MySQL
  \item \textbf{Instance type:} db.t3.medium
  \item \textbf{Deployment:} Single-AZ
  \item \textbf{Storage:} 100 GB
  \item \textbf{Pricing model:} On-Demand
  \item Click \textbf{"Add to my estimate"}
\end{itemize}


\begin{enumerate}
  \item \textbf{Review total:}
\end{enumerate}

\begin{itemize}
  \item See estimated monthly cost
  \item Compare different pricing models:
  \item On-Demand
  \item Reserved Instances (1-year, 3-year)
  \item Savings Plans
  \item Note potential savings
\end{itemize}


\begin{enumerate}
  \item \textbf{Export estimate:}
\end{enumerate}

\begin{itemize}
  \item Click \textbf{"Export"} → \textbf{"PDF"} or \textbf{"CSV"}
  \item Save for reference
\end{itemize}


\begin{enumerate}
  \item \textbf{Share estimate:}
\end{enumerate}

\begin{itemize}
  \item Click \textbf{"Share"}
  \item Copy shareable link
  \item Can send to colleagues or save for later
\end{itemize}


\paragraph{Part 2: AWS Cost Explorer}


\begin{keypoint}
\textbf{Note:} Cost Explorer takes 24 hours to populate for new accounts.
\end{keypoint}


\begin{enumerate}
  \item Go to \textbf{Billing and Cost Management} console
  \item Click \textbf{"Cost Explorer"} in left menu
  \item Click \textbf{"Launch Cost Explorer"} (if first time)
  \item Wait for initialization (if new account, data appears in 24 hours)
\end{enumerate}


\textbf{Explore costs (if data available):}

\begin{enumerate}
  \item \textbf{View monthly costs:}
\end{enumerate}

\begin{itemize}
  \item Default view shows last 6 months
  \item View costs by service
  \item Identify top services
\end{itemize}


\begin{enumerate}
  \item \textbf{Filter by service:}
\end{enumerate}

\begin{itemize}
  \item Click filter dropdown
  \item Select specific service (EC2, S3, etc.)
  \item View service-specific costs
\end{itemize}


\begin{enumerate}
  \item \textbf{Group by:}
\end{enumerate}

\begin{itemize}
  \item Service
  \item Region
  \item Tag
  \item Instance type
\end{itemize}


\begin{enumerate}
  \item \textbf{View forecast:}
\end{enumerate}

\begin{itemize}
  \item See projected costs for next month
  \item Based on current usage patterns
\end{itemize}


\begin{enumerate}
  \item \textbf{Create custom report:}
\end{enumerate}

\begin{itemize}
  \item Select date range
  \item Choose groupings and filters
  \item Click \textbf{"Save to report library"}
  \item Name: "Monthly Service Breakdown"
  \item Save report for future use
\end{itemize}


\begin{enumerate}
  \item \textbf{Download CSV:}
\end{enumerate}

\begin{itemize}
  \item Click \textbf{"Download CSV"}
  \item Open in spreadsheet for analysis
\end{itemize}


\paragraph{Part 3: Review Bills}


\begin{enumerate}
  \item Go to \textbf{"Bills"} in Billing console
  \item \textbf{View current month charges:}
\end{enumerate}

\begin{itemize}
  \item Expand services to see breakdown
  \item View charges by region
  \item Check data transfer costs
\end{itemize}

\begin{enumerate}
  \item \textbf{Check Free Tier usage:}
\end{enumerate}

\begin{itemize}
  \item Click \textbf{"Free Tier"} in left menu
  \item View current month usage vs. Free Tier limits
  \item \textbf{Important:} Monitor to avoid charges
  \item See warnings for services approaching limits
\end{itemize}

\begin{enumerate}
  \item \textbf{Download bill:}
\end{enumerate}

\begin{itemize}
  \item Click \textbf{"Download CSV"}
  \item Save for records
\end{itemize}


\paragraph{Part 4: AWS Budgets}


\begin{enumerate}
  \item Go to \textbf{"Budgets"} in Billing console
  \item Review budget created in Lab 1 (if you did it)
  \item \textbf{Create additional budget:}
\end{enumerate}

\begin{itemize}
  \item Click \textbf{"Create budget"}
  \item \textbf{Budget type:} Usage budget
  \item \textbf{Service:} Amazon Elastic Compute Cloud
  \item Click \textbf{"Next"}
\end{itemize}


\begin{enumerate}
  \item \textbf{Set budget details:}
\end{enumerate}

\begin{itemize}
  \item \textbf{Budget name:} "EC2-Usage-Budget"
  \item \textbf{Period:} Monthly
  \item \textbf{Usage type:} Running Hours
  \item \textbf{Unit:} Hrs
  \item \textbf{Amount:} 750 (Free Tier limit)
  \item Click \textbf{"Next"}
\end{itemize}


\begin{enumerate}
  \item \textbf{Configure alert:}
\end{enumerate}

\begin{itemize}
  \item \textbf{Threshold:} 80\% of budgeted amount
  \item \textbf{Email recipients:} Your email
  \item Click \textbf{"Add alert threshold"}
  \item \textbf{Second threshold:} 100\%
  \item Click \textbf{"Next"}
\end{itemize}


\begin{enumerate}
  \item Review and click \textbf{"Create budget"}
  \item \textbf{View budgets:}
\end{enumerate}

\begin{itemize}
  \item See all budgets in dashboard
  \item Monitor current usage vs. budget
  \item View alerts history
\end{itemize}


\paragraph{Part 5: AWS Trusted Advisor}


\begin{enumerate}
  \item Navigate to \textbf{Trusted Advisor} service
  \item \textbf{Dashboard overview:}
\end{enumerate}

\begin{itemize}
  \item View checks by category:
  \item Cost Optimization
  \item Performance
  \item Security
  \item Fault Tolerance
  \item Service Limits
\end{itemize}


\begin{enumerate}
  \item \textbf{Review 7 core checks} (available on Basic support):
\end{enumerate}

\begin{itemize}
  \item \textbf{S3 Bucket Permissions}
  \item Checks for publicly accessible buckets
  \item Click to view details
  \item \textbf{Security Groups - Specific Ports Unrestricted}
  \item Identifies overly permissive rules
  \item \textbf{IAM Use}
  \item Checks if you're using IAM
  \item \textbf{MFA on Root Account}
  \item Verifies MFA enabled
  \item \textbf{EBS Public Snapshots}
  \item Checks for public snapshots
  \item \textbf{RDS Public Snapshots}
  \item Checks for public snapshots
  \item \textbf{Service Limits}
  \item Shows usage vs. limits
\end{itemize}


\begin{enumerate}
  \item \textbf{Click on each check:}
\end{enumerate}

\begin{itemize}
  \item View details and recommendations
  \item Take action on warnings
  \item Green = good, Yellow = investigate, Red = action needed
\end{itemize}


\begin{enumerate}
  \item \textbf{Refresh checks:}
\end{enumerate}

\begin{itemize}
  \item Click \textbf{"Refresh all"}
  \item Checks update (may take few minutes)
\end{itemize}


\begin{keypoint}
\textbf{Note:} Full Trusted Advisor features require Business or Enterprise support plan.
\end{keypoint}


\subsubsection{Expected Outcomes}


\begin{itemize}
  \item Created cost estimate in Pricing Calculator
  \item Explored Cost Explorer (if data available)
  \item Reviewed current billing and Free Tier usage
  \item Created usage budget for EC2
  \item Reviewed Trusted Advisor recommendations
\end{itemize}


\subsubsection{Verification}


\begin{enumerate}
  \item \textbf{Pricing estimate created} and can be shared
  \item \textbf{Cost Explorer launched} (data may take 24 hours)
  \item \textbf{Current bill viewable} with service breakdown
  \item \textbf{Budgets configured} with email alerts
  \item \textbf{Trusted Advisor checks reviewed}
\end{enumerate}


\subsubsection{Troubleshooting}


\textbf{Problem:} Can't access billing information
\textbf{Solution:}
\begin{itemize}
  \item Enable IAM access to billing in Account settings
  \item Sign in as root user or IAM user with billing permissions
\end{itemize}


\textbf{Problem:} Cost Explorer shows no data
\textbf{Solution:}
\begin{itemize}
  \item Wait 24 hours after account creation
  \item Ensure you have some usage (launch services)
  \item Refresh page
\end{itemize}


\textbf{Problem:} Trusted Advisor shows limited checks
\textbf{Solution:}
\begin{itemize}
  \item Basic support only includes 7 core checks
  \item Upgrade to Business/Enterprise for full checks
  \item This is expected behavior
\end{itemize}


\subsubsection{Cleanup}


\begin{keypoint}
\textbf{Note:} Keep budgets and Trusted Advisor checks active for ongoing protection. No cleanup needed.
\end{keypoint}


---

\subsection{Lab 9: Lambda Serverless Function}


\textbf{Duration:} 25 minutes
\textbf{Cost:} Free (1 million requests/month in Free Tier)
\textbf{Difficulty:} Intermediate

\subsubsection{Objective}


Create a serverless Lambda function with API Gateway trigger.

\subsubsection{Prerequisites}


\begin{itemize}
  \item Basic programming knowledge (Python helpful but not required)
  \item Understanding of API concepts
\end{itemize}


\subsubsection{Step-by-Step Instructions}


\paragraph{Part 1: Create Lambda Function}


\begin{enumerate}
  \item Navigate to \textbf{Lambda} service
  \item Click \textbf{"Create function"}
  \item \textbf{Function option:} Author from scratch
  \item \textbf{Function name:} "HelloWorldFunction"
  \item \textbf{Runtime:} Python 3.12 (or latest available)
  \item \textbf{Architecture:} x86\_64
  \item \textbf{Permissions:}
\end{enumerate}

\begin{itemize}
  \item Execution role: Create a new role with basic Lambda permissions
  \item Role name: (auto-generated)
\end{itemize}

\begin{enumerate}
  \item Click \textbf{"Create function"}
  \item Wait for function creation
\end{enumerate}


\paragraph{Part 2: Write Function Code}


\begin{enumerate}
  \item In \textbf{Code source} section:
  \item Delete existing code in \texttt{lambda\_function.py}
  \item Paste the following code:
\end{enumerate}


\begin{lstlisting}[language=python]
import json

def lambda\_handler(event, context):
    \# Get name from event, default to 'World'
    name = event.get('name', 'World')

    \# Create response
    message = f'Hello, \{name\}!'

    return \{
        'statusCode': 200,
        'body': json.dumps(message),
        'headers': \{
            'Content-Type': 'application/json'
        \}
    \}
\end{lstlisting}

\begin{enumerate}
  \item Click \textbf{"Deploy"} (important!)
  \item Wait for "Successfully deployed" message
\end{enumerate}


\paragraph{Part 3: Test Function}


\begin{enumerate}
  \item Click \textbf{"Test"} button
  \item \textbf{Configure test event:}
\end{enumerate}

\begin{itemize}
  \item \textbf{Event name:} "TestEvent"
  \item \textbf{Event JSON:}
\end{itemize}

   \texttt{`}json
   {
     "name": "AWS Student"
   }
   \texttt{`}
\begin{enumerate}
  \item Click \textbf{"Save"}
  \item Click \textbf{"Test"} again
  \item \textbf{View execution results:}
\end{enumerate}

\begin{itemize}
  \item \textbf{Status:} Succeeded
  \item \textbf{Response:}
\end{itemize}

   \texttt{`}json
   {
     "statusCode": 200,
     "body": "\textbackslash{}"Hello, AWS Student!\textbackslash{}"",
     "headers": {
       "Content-Type": "application/json"
     }
   }
   \texttt{`}
\begin{enumerate}
  \item View \textbf{logs} in output
  \item Note execution time and memory used
\end{enumerate}


\paragraph{Part 4: View CloudWatch Logs}


\begin{enumerate}
  \item Click \textbf{"Monitor"} tab
  \item Click \textbf{"View CloudWatch logs"}
  \item Click on latest log stream
  \item View log details:
\end{enumerate}

\begin{itemize}
  \item START RequestId
  \item Function output
  \item END RequestId
  \item REPORT (duration, memory)
\end{itemize}


\paragraph{Part 5: Configure API Gateway Trigger}


\begin{enumerate}
  \item Go back to \textbf{Lambda function} (Code tab)
  \item Click \textbf{"Add trigger"}
  \item \textbf{Select a trigger:} API Gateway
  \item \textbf{API type:} HTTP API
  \item \textbf{Security:} Open
\end{enumerate}

\begin{itemize}
  \item \textbf{Warning:} This makes API publicly accessible
  \item For production, use authentication
\end{itemize}

\begin{enumerate}
  \item Click \textbf{"Add"}
  \item Wait for trigger creation
\end{enumerate}


\paragraph{Part 6: Test API Endpoint}


\begin{enumerate}
  \item In \textbf{Configuration} → \textbf{Triggers}, click on API Gateway
  \item \textbf{Copy API endpoint URL}
\end{enumerate}

\begin{itemize}
  \item Example: \texttt{https://abc123.execute-api.us-east-1.amazonaws.com/default/HelloWorldFunction}
\end{itemize}

\begin{enumerate}
  \item \textbf{Test in browser:}
\end{enumerate}

\begin{itemize}
  \item Paste URL in browser
  \item Add query parameter: \texttt{?name=YourName}
  \item Full URL: \texttt{https://abc123.execute-api.us-east-1.amazonaws.com/default/HelloWorldFunction?name=John}
  \item \textbf{Result:} You should see: \texttt{"Hello, John!"}
\end{itemize}


\begin{enumerate}
  \item \textbf{Test with curl (terminal):}
\end{enumerate}

   \texttt{`}bash
   curl "https://your-api-url.execute-api.us-east-1.amazonaws.com/default/HelloWorldFunction?name=John"
   \texttt{`}

\begin{enumerate}
  \item \textbf{Test with different names:}
\end{enumerate}

\begin{itemize}
  \item Try \texttt{?name=AWS}
  \item Try without parameter (should return "Hello, World!")
\end{itemize}


\paragraph{Part 7: Modify Function}


\begin{enumerate}
  \item Go back to \textbf{Code} tab
  \item Modify code to add more functionality:
\end{enumerate}


\begin{lstlisting}[language=python]
import json
from datetime import datetime

def lambda\_handler(event, context):
    \# Get name from event or query parameters
    name = event.get('name')
    if not name and 'queryStringParameters' in event:
        name = event['queryStringParameters'].get('name', 'World')
    else:
        name = name or 'World'

    \# Get current time
    current\_time = datetime.now().strftime('\%Y-\%m-\%d \%H:\%M:\%S')

    \# Create response
    message = \{
        'greeting': f'Hello, \{name\}!',
        'timestamp': current\_time,
        'requestId': context.request\_id
    \}

    return \{
        'statusCode': 200,
        'body': json.dumps(message),
        'headers': \{
            'Content-Type': 'application/json'
        \}
    \}
\end{lstlisting}

\begin{enumerate}
  \item Click \textbf{"Deploy"}
  \item \textbf{Test again} with API endpoint
  \item Now response includes timestamp and request ID
\end{enumerate}


\subsubsection{Expected Outcomes}


\begin{itemize}
  \item Lambda function created and deployed
  \item Function executes successfully with test events
  \item CloudWatch logs capture function output
  \item API Gateway trigger configured
  \item Function accessible via public HTTPS endpoint
  \item Modified function with enhanced functionality
\end{itemize}


\subsubsection{Verification}


\begin{enumerate}
  \item \textbf{Test event executes successfully}
  \item \textbf{API endpoint returns correct response}
  \item \textbf{CloudWatch logs show execution details}
  \item \textbf{Different inputs produce different outputs}
\end{enumerate}


\subsubsection{Troubleshooting}


\textbf{Problem:} Function fails with syntax error
\textbf{Solution:}
\begin{itemize}
  \item Check Python indentation (use spaces, not tabs)
  \item Verify all quotes and brackets match
  \item Review error in CloudWatch logs
\end{itemize}


\textbf{Problem:} API returns "Internal Server Error"
\textbf{Solution:}
\begin{itemize}
  \item Check CloudWatch logs for error details
  \item Verify function deployed after code changes
  \item Ensure JSON formatting correct in response
\end{itemize}


\textbf{Problem:} Can't access API endpoint
\textbf{Solution:}
\begin{itemize}
  \item Verify API Gateway trigger added
  \item Check security set to "Open"
  \item Ensure using correct HTTP method (GET)
  \item Try in different browser or incognito mode
\end{itemize}


\textbf{Problem:} Query parameters not working
\textbf{Solution:}
\begin{itemize}
  \item Use modified code that checks queryStringParameters
  \item Format URL correctly: \texttt{?name=Value}
  \item Check API Gateway integration settings
\end{itemize}


\subsubsection{Cleanup}


\begin{enumerate}
  \item \textbf{Delete Lambda function:}
\end{enumerate}

\begin{itemize}
  \item Select function
  \item \textbf{Actions} → \textbf{Delete}
  \item Type "delete"
  \item Confirm deletion
\end{itemize}


\begin{enumerate}
  \item \textbf{Delete API Gateway:}
\end{enumerate}

\begin{itemize}
  \item Go to \textbf{API Gateway} service
  \item Select your API
  \item \textbf{Actions} → \textbf{Delete}
  \item Confirm deletion
\end{itemize}


\begin{keypoint}
\textbf{Note:} CloudWatch logs persist after function deletion. Delete log group if desired:
- CloudWatch → Log groups → Select /aws/lambda/HelloWorldFunction → Delete
\end{keypoint}


---

\subsection{Lab 10: CloudFormation Infrastructure as Code}


\textbf{Duration:} 20 minutes
\textbf{Cost:} Free (resources created are Free Tier eligible)
\textbf{Difficulty:} Intermediate

\subsubsection{Objective}


Deploy AWS infrastructure using CloudFormation templates (Infrastructure as Code).

\subsubsection{Prerequisites}


\begin{itemize}
  \item Understanding of YAML or JSON
  \item Text editor
  \item Familiarity with S3 (from Lab 4)
\end{itemize}


\subsubsection{Step-by-Step Instructions}


\paragraph{Part 1: Create CloudFormation Template}


\begin{enumerate}
  \item Open text editor
  \item Create file named \texttt{simple-stack.yaml}
  \item Paste the following template:
\end{enumerate}


\begin{lstlisting}[language=yaml]
AWSTemplateFormatVersion: '2010-09-09'
Description: Simple S3 bucket stack for learning CloudFormation

Resources:
  MyS3Bucket:
    Type: AWS::S3::Bucket
    Properties:
      BucketName: !Sub 'cf-bucket-\$\{AWS::AccountId\}'
      VersioningConfiguration:
        Status: Enabled
      Tags:
        - Key: Environment
          Value: Learning
        - Key: ManagedBy
          Value: CloudFormation

Outputs:
  BucketName:
    Description: Name of the S3 bucket
    Value: !Ref MyS3Bucket
  BucketArn:
    Description: ARN of the S3 bucket
    Value: !GetAtt MyS3Bucket.Arn
\end{lstlisting}

\begin{enumerate}
  \item \textbf{Save file} to your computer
\end{enumerate}


\begin{keypoint}
\textbf{Explanation:}
- \textbf{Resources:} Defines S3 bucket with versioning
- \textbf{!Sub:} Substitutes account ID to make bucket name unique
- \textbf{Outputs:} Returns bucket name and ARN after creation
\end{keypoint}


\paragraph{Part 2: Create CloudFormation Stack}


\begin{enumerate}
  \item Navigate to \textbf{CloudFormation} service
  \item Click \textbf{"Create stack"} → \textbf{"With new resources (standard)"}
  \item \textbf{Prepare template:} Template is ready
  \item \textbf{Template source:} Upload a template file
  \item Click \textbf{"Choose file"} and select \texttt{simple-stack.yaml}
  \item Click \textbf{"Next"}
  \item \textbf{Stack name:} "MyFirstStack"
  \item Click \textbf{"Next"}
  \item \textbf{Configure stack options:}
\end{enumerate}

\begin{itemize}
  \item \textbf{Tags (optional):}
  \item Key: Project
  \item Value: CloudFormation-Lab
\end{itemize}

\begin{enumerate}
  \item Click \textbf{"Next"}
  \item \textbf{Review:}
\end{enumerate}

\begin{itemize}
  \item Verify all settings
  \item Review template in JSON/YAML view
\end{itemize}

\begin{enumerate}
  \item Click \textbf{"Submit"}
\end{enumerate}


\paragraph{Part 3: Monitor Stack Creation}


\begin{enumerate}
  \item \textbf{Stack status:} CREATE\\textit{IN\}PROGRESS
  \item Click \textbf{"Events"} tab
\end{enumerate}

\begin{itemize}
  \item Watch real-time creation events
  \item See each resource being created
\end{itemize}

\begin{enumerate}
  \item Click \textbf{"Resources"} tab
\end{enumerate}

\begin{itemize}
  \item View logical ID and physical ID
  \item See S3 bucket being created
\end{itemize}

\begin{enumerate}
  \item Wait for \textbf{Status:} CREATE\_COMPLETE (takes 1-2 minutes)
  \item Click \textbf{"Outputs"} tab
\end{enumerate}

\begin{itemize}
  \item View BucketName and BucketArn
  \item Copy bucket name
\end{itemize}


\paragraph{Part 4: Verify Resource Creation}


\begin{enumerate}
  \item Open new tab → Navigate to \textbf{S3} service
  \item \textbf{Verify bucket exists:}
\end{enumerate}

\begin{itemize}
  \item Find bucket: \texttt{cf-bucket-[your-account-id]}
  \item Click on bucket
  \item Verify versioning enabled (Properties → Bucket Versioning)
  \item Check tags (Properties → Tags)
\end{itemize}

\begin{enumerate}
  \item \textbf{Note:} This bucket was created entirely by CloudFormation template
\end{enumerate}


\paragraph{Part 5: Update Stack}


\textbf{Create updated template:}

\begin{enumerate}
  \item Open \texttt{simple-stack.yaml}
  \item Add encryption configuration:
\end{enumerate}


\begin{lstlisting}[language=yaml]
AWSTemplateFormatVersion: '2010-09-09'
Description: Simple S3 bucket stack for learning CloudFormation

Resources:
  MyS3Bucket:
    Type: AWS::S3::Bucket
    Properties:
      BucketName: !Sub 'cf-bucket-\$\{AWS::AccountId\}'
      VersioningConfiguration:
        Status: Enabled
      BucketEncryption:
        ServerSideEncryptionConfiguration:
          - ServerSideEncryptionByDefault:
              SSEAlgorithm: AES256
      PublicAccessBlockConfiguration:
        BlockPublicAcls: true
        BlockPublicPolicy: true
        IgnorePublicAcls: true
        RestrictPublicBuckets: true
      Tags:
        - Key: Environment
          Value: Learning
        - Key: ManagedBy
          Value: CloudFormation

Outputs:
  BucketName:
    Description: Name of the S3 bucket
    Value: !Ref MyS3Bucket
  BucketArn:
    Description: ARN of the S3 bucket
    Value: !GetAtt MyS3Bucket.Arn
\end{lstlisting}

\begin{enumerate}
  \item Save file
\end{enumerate}


\textbf{Update the stack:}

\begin{enumerate}
  \item Go back to \textbf{CloudFormation} console
  \item Select \textbf{MyFirstStack}
  \item Click \textbf{"Update"}
  \item \textbf{Replace current template:} Upload a template file
  \item Choose updated \texttt{simple-stack.yaml}
  \item Click \textbf{"Next"}
  \item \textbf{Parameters:} (none to change)
  \item Click \textbf{"Next"}
  \item \textbf{Review change set:}
\end{enumerate}

\begin{itemize}
  \item CloudFormation shows what will change
  \item Added: Encryption configuration
  \item Added: Public access block
  \item \textbf{Important:} Shows BEFORE making changes
\end{itemize}

\begin{enumerate}
  \item Click \textbf{"Next"}
  \item Review and click \textbf{"Submit"}
  \item \textbf{Watch update:}
\end{enumerate}

\begin{itemize}
  \item Status: UPDATE\\textit{IN\}PROGRESS
  \item Events show modifications
  \item Status: UPDATE\_COMPLETE
\end{itemize}


\textbf{Verify update:}

\begin{enumerate}
  \item Go to S3 → Your bucket
  \item Properties → Default encryption → Verify enabled
  \item Properties → Block public access → Verify all enabled
\end{enumerate}


\paragraph{Part 6: View Stack Template}


\begin{enumerate}
  \item In CloudFormation, select stack
  \item Click \textbf{"Template"} tab
  \item \textbf{View in Designer:}
\end{enumerate}

\begin{itemize}
  \item Click \textbf{"View in Application Composer"} or \textbf{"View in Designer"}
  \item See visual representation of resources
  \item Shows relationships between resources
\end{itemize}

\begin{enumerate}
  \item \textbf{Download template:}
\end{enumerate}

\begin{itemize}
  \item View in JSON or YAML format
  \item Click "Copy to clipboard" if needed
\end{itemize}


\subsubsection{Expected Outcomes}


\begin{itemize}
  \item CloudFormation stack created successfully
  \item S3 bucket deployed with versioning enabled
  \item Stack updated to add encryption
  \item Template viewable in designer
  \item Infrastructure defined as code (repeatable, version-controlled)
\end{itemize}


\subsubsection{Verification}


\begin{enumerate}
  \item \textbf{Stack shows CREATE\_COMPLETE status}
  \item \textbf{S3 bucket exists} with correct configuration
  \item \textbf{Outputs display} bucket name and ARN
  \item \textbf{Update successful} with encryption enabled
  \item \textbf{All resources tagged} with CloudFormation info
\end{enumerate}


\subsubsection{Troubleshooting}


\textbf{Problem:} Stack creation fails with "Bucket already exists"
\textbf{Solution:}
\begin{itemize}
  \item Bucket names must be globally unique
  \item Change bucket name in template
  \item Or delete existing bucket first
\end{itemize}


\textbf{Problem:} Template validation error
\textbf{Solution:}
\begin{itemize}
  \item Check YAML syntax (indentation critical)
  \item Verify all keys spelled correctly
  \item Use online YAML validator
  \item Check CloudFormation documentation for resource properties
\end{itemize}


\textbf{Problem:} Update fails
\textbf{Solution:}
\begin{itemize}
  \item Review change set before confirming
  \item Some properties can't be updated (require replacement)
  \item Check Events tab for specific error
  \item May need to delete and recreate stack
\end{itemize}


\textbf{Problem:} Can't delete stack
\textbf{Solution:}
\begin{itemize}
  \item Ensure S3 bucket is empty first
  \item CloudFormation can't delete non-empty buckets
  \item Manually empty bucket, then retry delete
\end{itemize}


\subsubsection{Cleanup}


\begin{important}
\textbf{Important:} CloudFormation makes cleanup easy - deletes all resources automatically.
\end{important}


\begin{enumerate}
  \item Go to \textbf{CloudFormation} console
  \item Select \textbf{MyFirstStack}
  \item Click \textbf{"Delete"}
  \item \textbf{Confirm deletion}
  \item \textbf{Monitor deletion:}
\end{enumerate}

\begin{itemize}
  \item Status: DELETE\\textit{IN\}PROGRESS
  \item Events show resources being deleted
  \item S3 bucket deleted (if empty)
  \item Status: DELETE\_COMPLETE (or stack disappears)
\end{itemize}

\begin{enumerate}
  \item \textbf{Verify in S3:}
\end{enumerate}

\begin{itemize}
  \item Go to S3 console
  \item Bucket should be gone
\end{itemize}


\begin{keypoint}
\textbf{Note:} If deletion fails, it's usually because S3 bucket not empty. Empty bucket manually and retry.
\end{keypoint}


---

\subsection{Lab 11: Auto Scaling and Load Balancing}


\textbf{Duration:} 45 minutes
\textbf{Cost:} Free (within Free Tier limits)
\textbf{Difficulty:} Advanced

\subsubsection{Learning Objectives}


By the end of this lab, you will be able to:

\begin{enumerate}
  \item Create a Launch Template for EC2 instances
  \item Configure an Application Load Balancer (ALB)
  \item Set up an Auto Scaling Group with scaling policies
  \item Understand target tracking and step scaling
  \item Test automatic scale-out and scale-in behaviors
  \item Monitor Auto Scaling activities in CloudWatch
\end{enumerate}


\subsubsection{Why This Lab Matters}


\textbf{Real-World Scenario:} An e-commerce website experiences 10x traffic during Black Friday sales. Auto Scaling automatically adds servers during peak hours and removes them when traffic decreases, optimizing both performance and cost.

\textbf{Exam Relevance:} Auto Scaling and ELB are heavily tested topics. Know:
\begin{itemize}
  \item Types of load balancers (ALB, NLB, CLB)
  \item Auto Scaling components (launch templates, groups, policies)
  \item Scaling policies (target tracking, step, scheduled)
  \item Health checks and high availability
\end{itemize}


\subsubsection{Prerequisites}


\begin{itemize}
  \item Completed Lab 3 (EC2) and Lab 5 (VPC)
  \item Understanding of load balancing concepts
  \item Basic knowledge of web servers
\end{itemize}


\subsubsection{Step-by-Step Instructions}


\paragraph{Part 1: Create Launch Template}


\begin{keypoint}
\textbf{What You'll See:} Launch template configuration wizard with pre-defined settings for EC2 instances.
\end{keypoint}


\begin{enumerate}
  \item Navigate to \textbf{EC2} service
  \item In left menu, click \textbf{"Launch Templates"}
  \item Click \textbf{"Create launch template"}
  \item \textbf{Launch template name:} "WebServer-Template"
  \item \textbf{Template version description:} "Initial version with Apache"
  \item \textbf{Auto Scaling guidance:} Check "Provide guidance to help me set up a template that I can use with EC2 Auto Scaling"
  \item \textbf{Application and OS Images (AMI):}
\end{enumerate}

\begin{itemize}
  \item Click \textbf{"Quick Start"}
  \item Select \textbf{"Amazon Linux 2023 AMI"}
  \item Verify "Free tier eligible" label
\end{itemize}


\begin{enumerate}
  \item \textbf{Instance type:} t2.micro (or t3.micro)
  \item \textbf{Key pair:} Select existing key pair or create new one
\end{enumerate}

\begin{itemize}
  \item \textbf{Best Practice:} Use existing key from Lab 3 if available
\end{itemize}


\begin{enumerate}
  \item \textbf{Network settings:}
\end{enumerate}

\begin{itemize}
  \item \textbf{Subnet:} Don't include in launch template (let Auto Scaling choose)
  \item \textbf{Firewall (security groups):}
  \item Click \textbf{"Create security group"}
  \item \textbf{Name:} "ALB-WebServer-SG"
  \item \textbf{Description:} "Allow HTTP from Load Balancer"
  \item \textbf{VPC:} Default VPC
  \item \textbf{Inbound rules:}
  \item Rule 1: HTTP (80), Source: 0.0.0.0/0
  \item Rule 2: SSH (22), Source: My IP
\end{itemize}


\begin{enumerate}
  \item \textbf{Advanced details:}
\end{enumerate}

\begin{itemize}
  \item Scroll to \textbf{"User data"} section
  \item Paste the following script:
\end{itemize}


\begin{lstlisting}[language=bash]
\#!/bin/bash
yum update -y
yum install -y httpd
systemctl start httpd
systemctl enable httpd

\# Create unique web page showing instance ID
INSTANCE\_ID=\$(ec2-metadata --instance-id | cut -d " " -f 2)
AZ=\$(ec2-metadata --availability-zone | cut -d " " -f 2)
cat > /var/www/html/index.html <<EOF
<!DOCTYPE html>
<html>
<head>
    <title>Auto Scaling Demo</title>
    <style>
        body \{ font-family: Arial; text-align: center; margin-top: 50px; \}
        .box \{ background: linear-gradient(135deg, \#667eea 0\%, \#764ba2 100\%);
               color: white; padding: 40px; border-radius: 10px;
               display: inline-block; \}
        h1 \{ margin: 0; \}
        p \{ font-size: 18px; \}
    </style>
</head>
<body>
    <div class="box">
        <h1>Auto Scaling is Working!</h1>
        <p><strong>Instance ID:</strong> \$INSTANCE\_ID</p>
        <p><strong>Availability Zone:</strong> \$AZ</p>
        <p>Refresh to see different instances</p>
    </div>
</body>
</html>
EOF
\end{lstlisting}

\begin{enumerate}
  \item Click \textbf{"Create launch template"}
  \item You'll see success message: "Successfully created WebServer-Template"
  \item Click \textbf{"View launch template"} to verify
\end{enumerate}


\textbf{Validation:}
\begin{itemize}
  \item Template shows in list with version 1
  \item All configuration visible in template details
\end{itemize}


\paragraph{Part 2: Create Application Load Balancer}


\begin{enumerate}
  \item In EC2 console, scroll down left menu to \textbf{"Load Balancers"}
  \item Click \textbf{"Create load balancer"}
  \item \textbf{Load balancer types:} Select \textbf{"Application Load Balancer"}
  \item Click \textbf{"Create"}
  \item \textbf{Basic configuration:}
\end{enumerate}

\begin{itemize}
  \item \textbf{Name:} "WebApp-ALB"
  \item \textbf{Scheme:} Internet-facing
  \item \textbf{IP address type:} IPv4
\end{itemize}


\begin{enumerate}
  \item \textbf{Network mapping:}
\end{enumerate}

\begin{itemize}
  \item \textbf{VPC:} Default VPC
  \item \textbf{Mappings:} Select at least 2 Availability Zones
  \item Check boxes for us-east-1a, us-east-1b (or your region's AZs)
  \item Select public subnets for each
\end{itemize}


\begin{enumerate}
  \item \textbf{Security groups:}
\end{enumerate}

\begin{itemize}
  \item Click \textbf{"Create new security group"} (opens new tab)
  \item \textbf{Name:} "ALB-SG"
  \item \textbf{Description:} "Allow HTTP from internet"
  \item \textbf{VPC:} Default
  \item \textbf{Inbound rules:}
  \item Type: HTTP, Port: 80, Source: 0.0.0.0/0
  \item \textbf{Outbound rules:} Keep default (all traffic)
  \item Click \textbf{"Create security group"}
  \item Return to ALB tab, refresh security groups list
  \item Select \textbf{"ALB-SG"}
  \item \textbf{Remove default security group}
\end{itemize}


\begin{enumerate}
  \item \textbf{Listeners and routing:}
\end{enumerate}

\begin{itemize}
  \item Protocol: HTTP, Port: 80 (default)
  \item \textbf{Default action:} Create target group
  \item Click \textbf{"Create target group"} (opens new tab)
\end{itemize}


\paragraph{Part 3: Create Target Group}


\begin{enumerate}
  \item \textbf{Target type:} Instances
  \item \textbf{Target group name:} "WebApp-TG"
  \item \textbf{Protocol:} HTTP, Port: 80
  \item \textbf{VPC:} Default VPC
  \item \textbf{Health checks:}
\end{enumerate}

\begin{itemize}
  \item \textbf{Protocol:} HTTP
  \item \textbf{Path:} / (root path)
  \item \textbf{Advanced health check settings:}
  \item Healthy threshold: 2
  \item Unhealthy threshold: 2
  \item Timeout: 5 seconds
  \item Interval: 30 seconds
  \item Success codes: 200
  \item \textbf{Why these settings:} Fast health checks (30s interval) with quick failover (2 failed checks = unhealthy)
\end{itemize}


\begin{enumerate}
  \item Click \textbf{"Next"}
  \item \textbf{Register targets:} Skip (Auto Scaling will register instances automatically)
  \item Click \textbf{"Create target group"}
  \item Return to ALB tab, refresh target groups
  \item Select \textbf{"WebApp-TG"} from dropdown
  \item \textbf{Tags (optional):}
\end{enumerate}

\begin{itemize}
  \item Key: Project, Value: AutoScaling-Lab
\end{itemize}


\begin{enumerate}
  \item \textbf{Review} all settings
  \item Click \textbf{"Create load balancer"}
  \item Wait 2-3 minutes for \textbf{State:} Active
\end{enumerate}


\textbf{Validation:}
\begin{itemize}
  \item ALB shows "Active" state
  \item Copy DNS name (e.g., WebApp-ALB-1234567890.us-east-1.elb.amazonaws.com)
  \item Try accessing in browser (will show 503 error - no targets yet, this is expected)
\end{itemize}


\paragraph{Part 4: Create Auto Scaling Group}


\begin{enumerate}
  \item In EC2 left menu, click \textbf{"Auto Scaling Groups"}
  \item Click \textbf{"Create Auto Scaling group"}
  \item \textbf{Step 1: Choose launch template}
\end{enumerate}

\begin{itemize}
  \item \textbf{Name:} "WebApp-ASG"
  \item \textbf{Launch template:} Select "WebServer-Template"
  \item \textbf{Version:} Latest (1)
  \item Click \textbf{"Next"}
\end{itemize}


\begin{enumerate}
  \item \textbf{Step 2: Choose instance launch options}
\end{enumerate}

\begin{itemize}
  \item \textbf{VPC:} Default VPC
  \item \textbf{Availability Zones and subnets:} Select 2 or more AZs
  \item Choose public subnets in each AZ
  \item Click \textbf{"Next"}
\end{itemize}


\begin{enumerate}
  \item \textbf{Step 3: Configure advanced options}
\end{enumerate}

\begin{itemize}
  \item \textbf{Load balancing:} Attach to an existing load balancer
  \item \textbf{Choose from your load balancer target groups}
  \item Select \textbf{"WebApp-TG"}
  \item \textbf{Health checks:}
  \item Check \textbf{"Turn on Elastic Load Balancing health checks"}
  \item Health check grace period: 300 seconds
  \item \textbf{Why:} Gives instances time to fully start before health checks
  \item \textbf{Monitoring:}
  \item Check \textbf{"Enable group metrics collection within CloudWatch"}
  \item Click \textbf{"Next"}
\end{itemize}


\begin{enumerate}
  \item \textbf{Step 4: Configure group size and scaling}
\end{enumerate}

\begin{itemize}
  \item \textbf{Group size:}
  \item Desired capacity: 2
  \item Minimum capacity: 1
  \item Maximum capacity: 4
  \item \textbf{Scaling policies:}
  \item Select \textbf{"Target tracking scaling policy"}
  \item \textbf{Scaling policy name:} "Target-Tracking-CPU"
  \item \textbf{Metric type:} Average CPU utilization
  \item \textbf{Target value:} 50
  \item \textbf{Instance warmup:} 300 seconds
  \item \textbf{Why this works:} When average CPU across all instances exceeds 50\%, add instances. When below 50\%, remove instances.
  \item Click \textbf{"Next"}
\end{itemize}


\begin{enumerate}
  \item \textbf{Step 5: Add notifications (optional)}
\end{enumerate}

\begin{itemize}
  \item Skip or add SNS topic for scaling events
  \item Click \textbf{"Next"}
\end{itemize}


\begin{enumerate}
  \item \textbf{Step 6: Add tags}
\end{enumerate}

\begin{itemize}
  \item \textbf{Key:} Name, \textbf{Value:} AutoScaled-WebServer
  \item Check \textbf{"Tag new instances"}
  \item Click \textbf{"Next"}
\end{itemize}


\begin{enumerate}
  \item \textbf{Step 7: Review}
\end{enumerate}

\begin{itemize}
  \item Verify all settings
  \item Click \textbf{"Create Auto Scaling group"}
\end{itemize}


\begin{enumerate}
  \item \textbf{Monitor creation:}
\end{enumerate}

\begin{itemize}
  \item ASG created immediately
  \item Watch "Activity" tab for instance launches
  \item Wait 3-5 minutes for 2 instances to launch and become healthy
\end{itemize}


\textbf{Validation:}
\begin{itemize}
  \item Auto Scaling group shows "2 instances" desired/running
  \item Activity history shows successful launches
  \item Go to Target Group → Targets tab → Both instances show "healthy"
  \item Access ALB DNS name → See web page with instance ID
\end{itemize}


\paragraph{Part 5: Test Load Balancing}


\begin{enumerate}
  \item \textbf{Copy ALB DNS name} from Load Balancers page
  \item \textbf{Open in browser:} \texttt{http://WebApp-ALB-1234567890.us-east-1.elb.amazonaws.com}
  \item You should see: "Auto Scaling is Working!" with an instance ID
  \item \textbf{Refresh page multiple times} (F5 or Cmd+R)
  \item Notice instance ID changes between refreshes
\end{enumerate}

\begin{itemize}
  \item \textbf{What's happening:} ALB distributes requests across instances (round-robin by default)
\end{itemize}


\begin{enumerate}
  \item \textbf{Test from command line (optional):}
\end{enumerate}

\begin{lstlisting}[language=bash]
for i in \{1..10\}; do curl http://your-alb-dns-name.elb.amazonaws.com | grep "Instance ID"; done
\end{lstlisting}
\begin{itemize}
  \item Shows distribution across instances
\end{itemize}


\textbf{Expected Behavior:}
\begin{itemize}
  \item Requests alternate between 2 different instance IDs
  \item Both instances serve traffic
  \item Response time is fast (<100ms)
\end{itemize}


\paragraph{Part 6: Test Auto Scaling (Scale Out)}


\begin{keypoint}
\textbf{Warning:} This generates CPU load. Monitor closely and stop if needed.
\end{keypoint}


\begin{enumerate}
  \item \textbf{SSH into one instance:}
\end{enumerate}

\begin{itemize}
  \item Go to EC2 → Instances
  \item Find instances tagged "AutoScaled-WebServer"
  \item Connect via SSH:
\end{itemize}

   \texttt{`}bash
   ssh -i your-key.pem ec2-user@[public-ip]
   \texttt{`}

\begin{enumerate}
  \item \textbf{Install stress tool:}
\end{enumerate}

\begin{lstlisting}[language=bash]
sudo yum install -y stress
\end{lstlisting}

\begin{enumerate}
  \item \textbf{Generate CPU load:}
\end{enumerate}

\begin{lstlisting}[language=bash]
stress --cpu 2 --timeout 600
\end{lstlisting}
\begin{itemize}
  \item Runs for 10 minutes (600 seconds)
  \item Pushes CPU to \textasciitilde{}100\%
\end{itemize}


\begin{enumerate}
  \item \textbf{Monitor Auto Scaling:}
\end{enumerate}

\begin{itemize}
  \item Go to Auto Scaling Groups → WebApp-ASG
  \item Click \textbf{"Activity"} tab
  \item Watch for new scaling activities
  \item \textbf{Time to scale:} 5-10 minutes typically
  \item 5 minutes of high CPU (CloudWatch evaluation)
  \item 2-3 minutes to launch new instance
  \item 5 minutes warmup period
\end{itemize}


\begin{enumerate}
  \item \textbf{Watch CloudWatch metrics:}
\end{enumerate}

\begin{itemize}
  \item Go to CloudWatch → Metrics → EC2 → By Auto Scaling Group
  \item Select CPUUtilization for WebApp-ASG
  \item \textbf{Graph settings:} 1-minute period
  \item You'll see CPU spike above 50\% target
\end{itemize}


\begin{enumerate}
  \item \textbf{Verify scale-out:}
\end{enumerate}

\begin{itemize}
  \item Auto Scaling group capacity increases: 2 → 3 or 3 → 4
  \item Activity history shows: "Launching a new EC2 instance"
  \item New instance appears in Instances list
  \item Target group shows 3-4 healthy targets
\end{itemize}


\textbf{What You Should See:}
\begin{itemize}
  \item CPU metric crosses 50\% threshold
  \item After \textasciitilde{}5 minutes, scaling activity triggers
  \item New instance launches automatically
  \item Total capacity increases
  \item Load distributes across more instances
\end{itemize}


\paragraph{Part 7: Test Auto Scaling (Scale In)}


\begin{enumerate}
  \item \textbf{Stop stress test:} Press Ctrl+C in SSH session (or wait for timeout)
  \item \textbf{Exit SSH:} Type \texttt{exit}
  \item \textbf{Monitor CPU decrease:}
\end{enumerate}

\begin{itemize}
  \item CloudWatch shows CPU dropping below 50\%
  \item Wait 15-20 minutes for scale-in
  \item \textbf{Why longer:} AWS conservatively waits before removing capacity
\end{itemize}


\begin{enumerate}
  \item \textbf{Watch Auto Scaling Activity:}
\end{enumerate}

\begin{itemize}
  \item Activity tab shows: "Terminating EC2 instance"
  \item Capacity decreases back to 2 (desired capacity)
  \item Extra instance terminates automatically
\end{itemize}


\textbf{Expected Timeline:}
\begin{itemize}
  \item CPU drops: Immediate
  \item Scale-in evaluation: 15 minutes (default cooldown)
  \item Instance termination: 2-3 minutes
  \item Total time to scale in: \textasciitilde{}20 minutes
\end{itemize}


\subsubsection{Expected Outcomes}


\begin{itemize}
  \item Launch template created with user data script
  \item Application Load Balancer distributing traffic across AZs
  \item Target group with health checks configured
  \item Auto Scaling group maintaining 2 instances normally
  \item Automatic scale-out when CPU exceeds 50\%
  \item Automatic scale-in when CPU returns to normal
  \item All instances serving traffic through ALB
\end{itemize}


\subsubsection{Verification Checklist}


\begin{itemize}
  \item [ ] Launch template shows in EC2 templates list
  \item [ ] ALB status is "Active"
  \item [ ] Target group shows all instances "healthy"
  \item [ ] Accessing ALB URL displays web page
  \item [ ] Refreshing shows different instance IDs (load balancing)
  \item [ ] Auto Scaling group maintains desired capacity
  \item [ ] Scale-out occurred during stress test
  \item [ ] Scale-in occurred after CPU normalized
\end{itemize}


\subsubsection{Real-World Tips}


\textbf{Launch Template Best Practices:}
\begin{itemize}
  \item Version templates for rollback capability
  \item Use latest Amazon Linux AMI for security patches
  \item Include monitoring agents in user data
  \item Test user data scripts before using in templates
\end{itemize}


\textbf{Load Balancer Configuration:}
\begin{itemize}
  \item Always use at least 2 AZs for high availability
  \item Configure appropriate health check paths (not just /)
  \item Set reasonable timeout values (5-10 seconds)
  \item Use HTTPS in production (requires SSL certificate)
  \item Enable access logs for troubleshooting
\end{itemize}


\textbf{Auto Scaling Tuning:}
\begin{itemize}
  \item \textbf{Conservative scaling:} Lower thresholds (40\% CPU) with longer cooldowns
  \item \textbf{Aggressive scaling:} Higher thresholds (70\% CPU) with shorter cooldowns
  \item \textbf{Production recommendation:} Start conservative, tune based on metrics
  \item \textbf{Cost optimization:} Use scheduled scaling for predictable patterns
\end{itemize}


\textbf{Common Scaling Metrics:}
\begin{itemize}
  \item CPU utilization: Most common, good for compute-bound apps
  \item Request count per target: Good for web apps with uniform requests
  \item Network throughput: For network-intensive applications
  \item Custom CloudWatch metrics: Application-specific (queue length, etc.)
\end{itemize}


\subsubsection{Troubleshooting}


\textbf{Problem:} Instances launch but stay "unhealthy" in target group

\textbf{Solution:}
\begin{itemize}
  \item Check security group allows HTTP (port 80) from ALB
  \item Verify health check path is correct (/)
  \item Ensure web server started (check user data logs: /var/log/cloud-init-output.log)
  \item Increase health check grace period to 400-500 seconds
  \item SSH to instance and test: \texttt{curl localhost}
\end{itemize}


\textbf{Problem:} Auto Scaling doesn't scale out despite high CPU

\textbf{Solution:}
\begin{itemize}
  \item Verify CPU metric is publishing to CloudWatch (EC2 → Instances → Monitoring)
  \item Check Auto Scaling group already at maximum capacity (4 instances)
  \item Wait full evaluation period (5 minutes of high CPU)
  \item Review scaling policy configuration and thresholds
  \item Check CloudWatch Alarms for scaling policy
\end{itemize}


\textbf{Problem:} Cannot access load balancer URL (timeout)

\textbf{Solution:}
\begin{itemize}
  \item Verify ALB security group allows HTTP from 0.0.0.0/0
  \item Ensure ALB is in public subnets with internet gateway
  \item Check ALB state is "Active" not "Provisioning"
  \item Verify at least one target is healthy
  \item Check route tables have internet gateway route
\end{itemize}


\textbf{Problem:} Scale-in never occurs

\textbf{Solution:}
\begin{itemize}
  \item Default scale-in protection may be enabled (check ASG settings)
  \item Wait longer (scale-in takes 15-20 minutes)
  \item Verify CPU actually dropped below threshold
  \item Check instance protection settings on individual instances
  \item Review scale-in policies (may have different thresholds)
\end{itemize}


\textbf{Problem:} Web page not showing instance ID

\textbf{Solution:}
\begin{itemize}
  \item User data script may have failed
  \item SSH to instance: \texttt{sudo cat /var/log/cloud-init-output.log}
  \item Check for script errors
  \item Verify httpd is running: \texttt{sudo systemctl status httpd}
  \item Test HTML file: \texttt{cat /var/www/html/index.html}
\end{itemize}


\subsubsection{Cleanup}


\begin{keypoint}
\textbf{Critical:} Load Balancers and running EC2 instances incur charges. Clean up immediately after lab.
\end{keypoint}


\textbf{Cleanup Order (Important - follow sequence):}

\begin{enumerate}
  \item \textbf{Delete Auto Scaling Group:}
\end{enumerate}

\begin{itemize}
  \item Go to Auto Scaling Groups
  \item Select "WebApp-ASG"
  \item \textbf{Actions} → \textbf{Delete}
  \item Type "delete" to confirm
  \item \textbf{This terminates all instances in the group}
  \item Wait for instances to terminate (2-3 minutes)
\end{itemize}


\begin{enumerate}
  \item \textbf{Verify instances terminated:}
\end{enumerate}

\begin{itemize}
  \item Go to EC2 → Instances
  \item Ensure all "AutoScaled-WebServer" instances show "Terminated"
\end{itemize}


\begin{enumerate}
  \item \textbf{Delete Load Balancer:}
\end{enumerate}

\begin{itemize}
  \item Go to Load Balancers
  \item Select "WebApp-ALB"
  \item \textbf{Actions} → \textbf{Delete load balancer}
  \item Type "confirm" to delete
  \item Wait for deletion (1-2 minutes)
\end{itemize}


\begin{enumerate}
  \item \textbf{Delete Target Group:}
\end{enumerate}

\begin{itemize}
  \item Go to Target Groups
  \item Select "WebApp-TG"
  \item \textbf{Actions} → \textbf{Delete}
  \item Confirm deletion
\end{itemize}


\begin{enumerate}
  \item \textbf{Delete Launch Template:}
\end{enumerate}

\begin{itemize}
  \item Go to Launch Templates
  \item Select "WebServer-Template"
  \item \textbf{Actions} → \textbf{Delete template}
  \item Confirm deletion
\end{itemize}


\begin{enumerate}
  \item \textbf{Delete Security Groups:}
\end{enumerate}

\begin{itemize}
  \item Go to Security Groups
  \item Select "ALB-SG" → \textbf{Actions} → \textbf{Delete security groups}
  \item Select "ALB-WebServer-SG" → \textbf{Actions} → \textbf{Delete security groups}
  \item \textbf{Note:} May need to wait if dependencies exist
\end{itemize}


\begin{enumerate}
  \item \textbf{Verify cleanup:}
\end{enumerate}

\begin{itemize}
  \item No running or pending EC2 instances from this lab
  \item No load balancers in list
  \item No Auto Scaling groups in list
  \item Target group deleted
\end{itemize}


\textbf{Cost Warning:}
\begin{itemize}
  \item Load Balancers: \$0.0225/hour (\textasciitilde{}\$16/month) - NOT free tier eligible
  \item EC2 instances: Free if within 750 hours/month on t2.micro
  \item \textbf{Leaving ALB running overnight = \textasciitilde{}\$0.54 wasted}
\end{itemize}


\textbf{Verification Commands (optional):}
\begin{lstlisting}[language=bash]
aws elbv2 describe-load-balancers --region us-east-1
aws autoscaling describe-auto-scaling-groups --region us-east-1
aws ec2 describe-instances --filters "Name=instance-state-name,Values=running" --region us-east-1
\end{lstlisting}

\subsubsection{Post-Lab Knowledge Check}


\textbf{Question 1:} What's the difference between desired, minimum, and maximum capacity?

<details>
<summary>Click to reveal answer</summary>

\textbf{Answer:}
\begin{itemize}
  \item \textbf{Desired capacity:} Current target number of instances Auto Scaling maintains
  \item \textbf{Minimum capacity:} Lowest number of instances (never goes below this)
  \item \textbf{Maximum capacity:} Highest number of instances (never exceeds this)
\end{itemize}


Example: Min=1, Desired=2, Max=4
\begin{itemize}
  \item Normal operation: 2 instances running
  \item During scale-out: Can add up to 2 more (total 4)
  \item During scale-in: Can remove 1 (minimum is 1)
  \item Desired capacity changes dynamically, min/max are boundaries
\end{itemize}


</details>

\textbf{Question 2:} Why use target tracking instead of step scaling?

<details>
<summary>Click to reveal answer</summary>

\textbf{Answer:}
\begin{itemize}
  \item \textbf{Target tracking:} Simpler to configure, automatically calculates scaling adjustments to maintain target (like a thermostat). Best for most use cases.
  \item \textbf{Step scaling:} More control, define specific scaling amounts for different threshold ranges. Use for complex scaling patterns.
  \item \textbf{Exam tip:} Target tracking is recommended by AWS for most scenarios and is the default option.
\end{itemize}


</details>

\textbf{Question 3:} What happens if an instance fails health checks?

<details>
<summary>Click to reveal answer</summary>

\textbf{Answer:}
\begin{enumerate}
  \item Target group marks instance "unhealthy" after 2 failed checks (configurable)
  \item Load balancer stops sending traffic to that instance
  \item Auto Scaling detects unhealthy instance
  \item After grace period, Auto Scaling terminates unhealthy instance
  \item Auto Scaling launches replacement instance to maintain desired capacity
  \item New instance goes through health checks before receiving traffic
\end{enumerate}


This provides self-healing infrastructure!

</details>

\textbf{Question 4:} Can Auto Scaling work without a load balancer?

<details>
<summary>Click to reveal answer</summary>

\textbf{Answer:} Yes! Auto Scaling works independently of load balancers.

\textbf{Without ELB:}
\begin{itemize}
  \item Instances still launch/terminate based on policies
  \item Use cases: Batch processing, worker nodes, background jobs
  \item Health checks based on EC2 status only
\end{itemize}


\textbf{With ELB:}
\begin{itemize}
  \item Better for web apps serving user traffic
  \item Health checks from both ELB and EC2
  \item Traffic distributed automatically
  \item More resilient architecture
\end{itemize}


\textbf{Exam tip:} Know that Auto Scaling and ELB are separate services that work well together but aren't required together.

</details>

\textbf{Question 5:} What's the purpose of the warmup period?

<details>
<summary>Click to reveal answer</summary>

\textbf{Answer:} Warmup period (300 seconds recommended) prevents new instances from being evaluated for scaling before they're ready.

\textbf{Without warmup:}
\begin{itemize}
  \item New instance launched (CPU low while starting)
  \item Average CPU of group drops
  \item Might trigger premature scale-in
  \item Instability and thrashing
\end{itemize}


\textbf{With warmup:}
\begin{itemize}
  \item New instance launches
  \item Metrics ignored for 300 seconds
  \item Instance has time to initialize and receive traffic
  \item Stable scaling behavior
\end{itemize}


</details>

\subsubsection{Key Takeaways}


\begin{itemize}
  \item \textbf{Auto Scaling provides elasticity} - capacity matches demand automatically
  \item \textbf{ELB distributes traffic} - no single point of failure
  \item \textbf{Target tracking is simplest} - set target, AWS handles the rest
  \item \textbf{Health checks are critical} - determines which instances receive traffic
  \item \textbf{Multi-AZ deployment} - provides high availability
  \item \textbf{Launch templates} - define instance configuration once, reuse many times
  \item \textbf{Cooldown periods prevent thrashing} - avoids rapid scale in/out cycles
  \item \textbf{Cost optimization} - only pay for what you need, when you need it
\end{itemize}


---

\subsection{Lab 12: DynamoDB Hands-On}


\textbf{Duration:} 35 minutes
\textbf{Cost:} Free (25 GB storage always free)
\textbf{Difficulty:} Intermediate

\subsubsection{Learning Objectives}


By the end of this lab, you will be able to:

\begin{enumerate}
  \item Create a DynamoDB table with partition and sort keys
  \item Understand DynamoDB data types and attributes
  \item Add, query, and scan items using the AWS Console
  \item Create and use Global Secondary Indexes (GSI)
  \item Configure DynamoDB auto-scaling
  \item Understand read/write capacity modes
  \item Export table data and enable point-in-time recovery
\end{enumerate}


\subsubsection{Why This Lab Matters}


\textbf{Real-World Scenario:} A mobile gaming company uses DynamoDB to store player profiles, game scores, and real-time leaderboards. DynamoDB handles millions of requests per day with single-digit millisecond latency, automatically scaling without server management.

\textbf{Exam Relevance:} DynamoDB is a key AWS service. Know:
\begin{itemize}
  \item NoSQL vs. relational databases
  \item Primary keys (partition key, sort key)
  \item Indexes (LSI, GSI)
  \item Capacity modes (on-demand vs. provisioned)
  \item DynamoDB Accelerator (DAX) for caching
\end{itemize}


\subsubsection{Prerequisites}


\begin{itemize}
  \item Basic understanding of databases (SQL or NoSQL)
  \item Knowledge of data structures (key-value pairs)
  \item Completed Lab 1 (billing alerts recommended)
\end{itemize}


\subsubsection{Step-by-Step Instructions}


\paragraph{Part 1: Create DynamoDB Table}


\begin{enumerate}
  \item Navigate to \textbf{DynamoDB} service
  \item Click \textbf{"Create table"}
  \item \textbf{Table details:}
\end{enumerate}

\begin{itemize}
  \item \textbf{Table name:} "GameScores"
  \item \textbf{Partition key:} "UserId" (String)
  \item \textbf{Sort key:} "GameTitle" (String)
  \item \textbf{Why these keys:}
  \item Partition key distributes data across partitions
  \item Sort key orders items within each partition
  \item Together form unique composite key
  \item Example: User123 + Minecraft, User123 + Fortnite
\end{itemize}


\begin{enumerate}
  \item \textbf{Table settings:}
\end{enumerate}

\begin{itemize}
  \item Select \textbf{"Customize settings"} (not Default settings)
\end{itemize}


\begin{enumerate}
  \item \textbf{Table class:}
\end{enumerate}

\begin{itemize}
  \item Select \textbf{"DynamoDB Standard"}
  \item (Standard-IA is for infrequently accessed data)
\end{itemize}


\begin{enumerate}
  \item \textbf{Read/write capacity settings:}
\end{enumerate}

\begin{itemize}
  \item \textbf{Capacity mode:} On-demand
  \item \textbf{Why on-demand:} Automatically scales, no capacity planning, pay per request
  \item \textbf{Alternative:} Provisioned mode (Free Tier: 25 WCU + 25 RCU)
  \item \textbf{Best for lab:} On-demand (simpler, less chance of throttling)
\end{itemize}


\begin{enumerate}
  \item \textbf{Secondary indexes:}
\end{enumerate}

\begin{itemize}
  \item Skip for now (add later in lab)
\end{itemize}


\begin{enumerate}
  \item \textbf{Encryption at rest:}
\end{enumerate}

\begin{itemize}
  \item \textbf{Encryption type:} Owned by Amazon DynamoDB
  \item (Default, no additional cost)
\end{itemize}


\begin{enumerate}
  \item \textbf{Tags (optional):}
\end{enumerate}

\begin{itemize}
  \item Key: Project, Value: DynamoDB-Lab
\end{itemize}


\begin{enumerate}
  \item Click \textbf{"Create table"}
  \item Wait 10-20 seconds for status: "Active"
\end{enumerate}


\textbf{Validation:}
\begin{itemize}
  \item Table appears in Tables list
  \item Status shows "Active"
  \item Table details show partition and sort keys
\end{itemize}


\paragraph{Part 2: Add Items to Table}


\begin{enumerate}
  \item Click on \textbf{"GameScores"} table name
  \item Click \textbf{"Explore table items"} button
  \item Click \textbf{"Create item"}
\end{enumerate}


\textbf{Item 1:}
\begin{enumerate}
  \item \textbf{Add attributes:}
\end{enumerate}

\begin{itemize}
  \item UserId (String): "User001"
  \item GameTitle (String): "Minecraft"
\end{itemize}

\begin{enumerate}
  \item Click \textbf{"Add new attribute"} → \textbf{Number}
\end{enumerate}

\begin{itemize}
  \item Attribute name: "Score"
  \item Value: 1250
\end{itemize}

\begin{enumerate}
  \item Click \textbf{"Add new attribute"} → \textbf{Number}
\end{enumerate}

\begin{itemize}
  \item Attribute name: "Level"
  \item Value: 15
\end{itemize}

\begin{enumerate}
  \item Click \textbf{"Add new attribute"} → \textbf{String}
\end{enumerate}

\begin{itemize}
  \item Attribute name: "PlayerName"
  \item Value: "Steve"
\end{itemize}

\begin{enumerate}
  \item Click \textbf{"Add new attribute"} → \textbf{Number}
\end{enumerate}

\begin{itemize}
  \item Attribute name: "Timestamp"
  \item Value: 1699564800 (Unix timestamp)
\end{itemize}

\begin{enumerate}
  \item Click \textbf{"Create item"}
\end{enumerate}


\textbf{Item 2:}
\begin{enumerate}
  \item Click \textbf{"Create item"} again
  \item Add attributes:
\end{enumerate}

\begin{itemize}
  \item UserId: "User001"
  \item GameTitle: "Fortnite"
  \item Score: 2400
  \item Level: 22
  \item PlayerName: "Steve"
  \item Timestamp: 1699651200
\end{itemize}


\textbf{Item 3:}
\begin{enumerate}
  \item Create third item:
\end{enumerate}

\begin{itemize}
  \item UserId: "User002"
  \item GameTitle: "Minecraft"
  \item Score: 980
  \item Level: 12
  \item PlayerName: "Alex"
  \item Timestamp: 1699737600
\end{itemize}


\textbf{Item 4:}
\begin{enumerate}
  \item Create fourth item:
\end{enumerate}

\begin{itemize}
  \item UserId: "User002"
  \item GameTitle: "Fortnite"
  \item Score: 3100
  \item Level: 28
  \item PlayerName: "Alex"
  \item Timestamp: 1699824000
\end{itemize}


\textbf{Item 5:}
\begin{enumerate}
  \item Create fifth item:
\end{enumerate}

\begin{itemize}
  \item UserId: "User003"
  \item GameTitle: "Minecraft"
  \item Score: 1800
  \item Level: 18
  \item PlayerName: "Herobrine"
  \item Timestamp: 1699910400
\end{itemize}


\textbf{What You Should See:}
\begin{itemize}
  \item Items appear in table immediately
  \item Each item has UserId + GameTitle (keys) plus additional attributes
  \item Items can have different attributes (schema-less)
  \item Scan shows all items
\end{itemize}


\paragraph{Part 3: Query Items}


\begin{keypoint}
\textbf{Query vs. Scan:} Query is efficient (uses keys), Scan reads entire table (slow, expensive).
\end{keypoint}


\begin{enumerate}
  \item Click \textbf{"Query"} (default view after adding items)
  \item \textbf{Query items where:}
\end{enumerate}

\begin{itemize}
  \item Partition key: UserId
  \item \textbf{Enter:} "User001"
\end{itemize}

\begin{enumerate}
  \item Click \textbf{"Run"}
\end{enumerate}


\textbf{Results:}
\begin{itemize}
  \item Shows 2 items: Minecraft and Fortnite for User001
  \item Sorted by GameTitle (sort key)
  \item Fast query using primary key
\end{itemize}


\begin{enumerate}
  \item \textbf{Add sort key condition:}
\end{enumerate}

\begin{itemize}
  \item Partition key: "User001"
  \item \textbf{Sort key condition:} GameTitle = "Minecraft"
\end{itemize}

\begin{enumerate}
  \item Click \textbf{"Run"}
\end{enumerate}


\textbf{Results:}
\begin{itemize}
  \item Shows only 1 item: User001's Minecraft score
  \item Even more specific query
\end{itemize}


\begin{enumerate}
  \item \textbf{Try another query:}
\end{enumerate}

\begin{itemize}
  \item Partition key: "User002"
  \item Leave sort key empty
\end{itemize}

\begin{enumerate}
  \item Click \textbf{"Run"}
\end{enumerate}


\textbf{Results:}
\begin{itemize}
  \item Shows both games for User002
\end{itemize}


\textbf{Real-World Use:} Query player's specific game score or all games for a player

\paragraph{Part 4: Scan Items}


\begin{enumerate}
  \item Click \textbf{"Scan"} tab (next to Query)
  \item Click \textbf{"Run"}
\end{enumerate}


\textbf{Results:}
\begin{itemize}
  \item Shows all 5 items from table
  \item No filtering applied
  \item \textbf{Warning:} Scans are expensive for large tables (read all data)
\end{itemize}


\begin{enumerate}
  \item \textbf{Add scan filter:}
\end{enumerate}

\begin{itemize}
  \item Click \textbf{"Filters"}
  \item \textbf{Add filter:}
  \item Attribute name: Score
  \item Condition: Greater than or equal to
  \item Value: 2000
  \item Click \textbf{"Run"}
\end{itemize}


\textbf{Results:}
\begin{itemize}
  \item Shows only 2 items: User001 Fortnite (2400), User002 Fortnite (3100)
  \item Filtered after scanning entire table
  \item \textbf{Note:} Still scans all items then filters (not as efficient as query)
\end{itemize}


\textbf{Best Practice:} Use queries with keys whenever possible; scans for analytics only

\paragraph{Part 5: Create Global Secondary Index (GSI)}


\textbf{Problem:} What if we want to find all players with Score > 2000 efficiently?
\textbf{Solution:} Create GSI with Score as partition key

\begin{enumerate}
  \item Go to \textbf{"Indexes"} tab
  \item Click \textbf{"Create index"}
  \item \textbf{Index details:}
\end{enumerate}

\begin{itemize}
  \item \textbf{Partition key:} Score (Number)
  \item \textbf{Sort key:} Timestamp (Number) (optional, but useful for ordering)
  \item \textbf{Index name:} "ScoreIndex"
  \item \textbf{Attribute projections:} All
  \item Projects all table attributes into index
  \item Alternative: Keys only (smaller, cheaper) or Include (specify attributes)
\end{itemize}


\begin{enumerate}
  \item Click \textbf{"Create index"}
  \item Wait 20-30 seconds for status: "Active"
\end{enumerate}


\textbf{Validation:}
\begin{itemize}
  \item Index shows in Indexes tab
  \item Status: Active
  \item Can now query by Score efficiently
\end{itemize}


\paragraph{Part 6: Query Using GSI}


\begin{enumerate}
  \item Go back to \textbf{"Explore table items"}
  \item Click \textbf{"Query"} tab
  \item \textbf{Index:} Select "ScoreIndex" from dropdown
  \item \textbf{Query items:}
\end{enumerate}

\begin{itemize}
  \item Partition key (Score): 1250
\end{itemize}

\begin{enumerate}
  \item Click \textbf{"Run"}
\end{enumerate}


\textbf{Results:}
\begin{itemize}
  \item Shows User001's Minecraft score
  \item Queried by Score instead of UserId
\end{itemize}


\begin{enumerate}
  \item \textbf{Try range query on GSI:}
\end{enumerate}

\begin{itemize}
  \item Unfortunately, DynamoDB Query requires exact partition key value
  \item For range queries on Score, must use Scan with filter (limitation of DynamoDB)
  \item \textbf{Alternative approach:} Create items with score ranges as partition keys (advanced)
\end{itemize}


\textbf{Real-World Use:}
\begin{itemize}
  \item GSI for querying by non-key attributes
  \item Common pattern: UserId as partition key, GSI on Email for login lookups
  \item Up to 20 GSIs per table
\end{itemize}


\paragraph{Part 7: Update Item}


\begin{enumerate}
  \item In "Explore table items" view, select an item (click radio button)
  \item Click \textbf{"Actions"} → \textbf{"Edit item"}
  \item Change Score from 1250 to 1300
  \item Click \textbf{"Add new attribute"} → \textbf{String}
\end{enumerate}

\begin{itemize}
  \item Attribute name: "Achievement"
  \item Value: "Master Builder"
\end{itemize}

\begin{enumerate}
  \item Click \textbf{"Save changes"}
\end{enumerate}


\textbf{What You Should See:}
\begin{itemize}
  \item Item updated immediately
  \item New attribute added
  \item Other items not affected (schema-less flexibility)
\end{itemize}


\paragraph{Part 8: Delete Item}


\begin{enumerate}
  \item Select an item (checkbox)
  \item Click \textbf{"Actions"} → \textbf{"Delete items"}
  \item Confirm deletion
  \item Item removed immediately
\end{enumerate}


\textbf{Restore item (practice adding):}
\begin{itemize}
  \item Click "Create item" and re-add the deleted item
\end{itemize}


\paragraph{Part 9: Configure Table Settings}


\begin{enumerate}
  \item Go to \textbf{"Additional settings"} tab
\end{enumerate}


\textbf{Point-in-time recovery (PITR):}
\begin{enumerate}
  \item Scroll to \textbf{"Point-in-time recovery"} section
  \item Click \textbf{"Edit"}
  \item Select \textbf{"Turn on"}
  \item Click \textbf{"Save changes"}
\end{enumerate}

\begin{itemize}
  \item \textbf{What it does:} Continuous backups, restore to any point in last 35 days
  \item \textbf{Cost:} Additional charge based on table size
  \item \textbf{Exam tip:} PITR protects against accidental deletes/updates
\end{itemize}


\textbf{Time to Live (TTL):}
\begin{enumerate}
  \item Scroll to \textbf{"Time to Live (TTL)"} section
  \item Click \textbf{"Edit"}
  \item \textbf{Turn on} TTL
  \item \textbf{TTL attribute:} "ExpiresAt"
  \item Click \textbf{"Save changes"}
\end{enumerate}

\begin{itemize}
  \item \textbf{What it does:} Automatically deletes items after expiration time
  \item \textbf{Use case:} Session data, temporary records
  \item \textbf{Cost:} Free (deletion doesn't consume write capacity)
\end{itemize}


\textbf{Note:} We didn't add ExpiresAt to our items, so TTL won't affect them

\paragraph{Part 10: Export Table Data}


\begin{enumerate}
  \item Go to \textbf{"Exports and streams"} tab
  \item Click \textbf{"Export to S3"}
  \item \textbf{Destination S3 bucket:}
\end{enumerate}

\begin{itemize}
  \item Click \textbf{"Browse S3"}
  \item Select existing bucket or create new one: "dynamodb-exports-[yourname]"
\end{itemize}

\begin{enumerate}
  \item \textbf{Export format:} DynamoDB JSON
  \item Click \textbf{"Export"}
  \item Export status: "In progress" → "Completed" (2-3 minutes)
  \item \textbf{View exported data:}
\end{enumerate}

\begin{itemize}
  \item Go to S3 → Your export bucket
  \item Navigate through folders to find data file
  \item Download and view JSON export
\end{itemize}


\textbf{Use cases:}
\begin{itemize}
  \item Data analysis with Athena
  \item Backup and archival
  \item Data migration
  \item Compliance requirements
\end{itemize}


\subsubsection{Expected Outcomes}


\begin{itemize}
  \item DynamoDB table created with composite primary key
  \item Multiple items added with various attributes
  \item Successfully queried items using partition key
  \item Scanned table with filters applied
  \item Created Global Secondary Index for alternative queries
  \item Updated and deleted items
  \item Configured Point-in-time Recovery and TTL
  \item Exported table data to S3
\end{itemize}


\subsubsection{Verification Checklist}


\begin{itemize}
  \item [ ] Table "GameScores" shows "Active" status
  \item [ ] At least 5 items in table
  \item [ ] Query by UserId returns correct items
  \item [ ] Scan with filter shows expected results
  \item [ ] Global Secondary Index "ScoreIndex" is active
  \item [ ] Can query using GSI
  \item [ ] Point-in-time recovery enabled
  \item [ ] Successfully exported to S3
\end{itemize}


\subsubsection{Real-World Tips}


\textbf{Partition Key Design:}
\begin{itemize}
  \item High cardinality (many unique values) for even distribution
  \item Avoid "hot partitions" (one key getting most traffic)
  \item Bad example: Date as partition key (all today's data on one partition)
  \item Good example: CustomerId (distributed across customers)
\end{itemize}


\textbf{When to Use DynamoDB:}
\begin{itemize}
  \item \textbf{Yes:} High-scale applications, gaming, IoT, mobile backends, real-time bidding
  \item \textbf{Yes:} Serverless applications (pairs well with Lambda)
  \item \textbf{Yes:} Key-value access patterns
  \item \textbf{No:} Complex joins (use RDS instead)
  \item \textbf{No:} Ad-hoc queries (use Athena + S3 or RDS)
  \item \textbf{No:} ACID transactions across tables (use RDS)
\end{itemize}


\textbf{Capacity Mode Selection:}
\begin{itemize}
  \item \textbf{On-demand:} Unpredictable workloads, new applications, pay-per-request
  \item \textbf{Provisioned:} Predictable traffic, steady state, cost optimization (up to 60\% savings)
  \item \textbf{Can switch:} Once per 24 hours between modes
\end{itemize}


\textbf{Cost Optimization:}
\begin{itemize}
  \item Use on-demand for development, provisioned for production
  \item Enable auto-scaling for provisioned mode
  \item Use S3 + Athena for analytics instead of scans
  \item Archive old data to S3 using TTL + streams
  \item Standard-IA class for infrequently accessed data
\end{itemize}


\subsubsection{Troubleshooting}


\textbf{Problem:} "Validation Exception" when creating item

\textbf{Solution:}
\begin{itemize}
  \item Verify partition key and sort key values provided
  \item Check attribute names don't have typos
  \item Ensure data types match (String vs. Number)
  \item Partition and sort keys are required fields
\end{itemize}


\textbf{Problem:} Query returns no results

\textbf{Solution:}
\begin{itemize}
  \item Verify exact partition key value (case-sensitive)
  \item Check you're using correct index (table vs. GSI)
  \item Confirm items exist with that partition key
  \item Try scan to see all items first
\end{itemize}


\textbf{Problem:} Cannot create GSI - limit exceeded

\textbf{Solution:}
\begin{itemize}
  \item Free Tier allows up to 20 GSIs per table
  \item Delete unused indexes before creating new ones
  \item Consider if you really need GSI (scan might be acceptable)
\end{itemize}


\textbf{Problem:} Export to S3 fails

\textbf{Solution:}
\begin{itemize}
  \item Verify S3 bucket exists and in same region
  \item Check DynamoDB has permissions to write to bucket
  \item Ensure bucket name is globally unique
  \item Review export status details for specific error
\end{itemize}


\textbf{Problem:} High read/write costs

\textbf{Solution:}
\begin{itemize}
  \item Check for scan operations (use queries instead)
  \item Review on-demand vs. provisioned pricing
  \item Implement caching layer (DAX or ElastiCache)
  \item Use eventually consistent reads (50\% cheaper)
\end{itemize}


\subsubsection{Cleanup}


\begin{keypoint}
\textbf{Good News:} DynamoDB charges only for storage and requests. With 25 GB always free, small tables are essentially free.
\end{keypoint}


\textbf{Option 1: Keep Table (Recommended for Learning)}
\begin{itemize}
  \item Minimal cost with 5 items (<1 KB storage)
  \item Good for practicing queries
  \item Can experiment further
\end{itemize}


\textbf{Option 2: Delete Table}

\begin{enumerate}
  \item Go to \textbf{DynamoDB} → \textbf{Tables}
  \item Select \textbf{"GameScores"} table
  \item Click \textbf{"Delete"}
  \item \textbf{Delete all CloudWatch alarms for this table:} Check box
  \item \textbf{Create a backup before deleting:} Uncheck (for lab purposes)
  \item Type \textbf{"delete"} to confirm
  \item Click \textbf{"Delete table"}
  \item \textbf{Delete S3 export bucket (if created):}
\end{enumerate}

\begin{itemize}
  \item Go to S3
  \item Select export bucket
  \item Click \textbf{"Empty"} → Type "permanently delete" → Empty
  \item Click \textbf{"Delete"} → Type bucket name → Delete
\end{itemize}


\textbf{Verification:}
\begin{itemize}
  \item Table no longer in DynamoDB tables list
  \item Export bucket deleted from S3
  \item No ongoing charges
\end{itemize}


\textbf{Cost Note:}
\begin{itemize}
  \item On-demand mode: \$0 with no traffic
  \item 5 items < 1 KB: \textasciitilde{}\$0.00025/month storage
  \item Effectively free to keep for practice
\end{itemize}


\subsubsection{Post-Lab Knowledge Check}


\textbf{Question 1:} What's the difference between partition key and sort key?

<details>
<summary>Click to reveal answer</summary>

\textbf{Answer:}
\begin{itemize}
  \item \textbf{Partition key (Hash key):} Required, determines which partition stores the item, must be unique if used alone
  \item \textbf{Sort key (Range key):} Optional, orders items within a partition, enables range queries
  \item \textbf{Together:} Form composite primary key (partition + sort), partition key groups items, sort key orders within group
\end{itemize}


Example: UserId (partition) + Timestamp (sort) allows querying all actions for a user, ordered by time

</details>

\textbf{Question 2:} When should you use a Global Secondary Index?

<details>
<summary>Click to reveal answer</summary>

\textbf{Answer:}
Use GSI when you need to query by attributes other than the primary key.

\textbf{Example scenarios:}
\begin{itemize}
  \item Table has UserId as partition key, need to query by Email → Create GSI with Email as partition key
  \item Table has OrderId as partition key, need to find all orders for a CustomerId → GSI on CustomerId
  \item Need different sort order → GSI with different sort key
\end{itemize}


\textbf{Limitations:}
\begin{itemize}
  \item Eventually consistent (slight delay)
  \item Consumes additional write capacity
  \item Costs extra storage (projects attributes)
  \item Cannot be changed after creation (must delete and recreate)
\end{itemize}


</details>

\textbf{Question 3:} What's the difference between Query and Scan?

<details>
<summary>Click to reveal answer</summary>

\textbf{Answer:}
\begin{itemize}
  \item \textbf{Query:} Efficient, uses partition key (and optionally sort key), returns only matching items, low latency, predictable cost
  \item \textbf{Scan:} Inefficient, reads entire table, filters after reading, high latency for large tables, expensive
\end{itemize}


\textbf{Example:}
\begin{itemize}
  \item Query for "UserId=User001": Reads only User001's items
  \item Scan with filter "UserId=User001": Reads ALL items, then filters
\end{itemize}


\textbf{Exam tip:} Always prefer Query over Scan. Use Scan only for analytics or one-time operations.

</details>

\textbf{Question 4:} What is DynamoDB Accelerator (DAX)?

<details>
<summary>Click to reveal answer</summary>

\textbf{Answer:}
DAX is an in-memory cache for DynamoDB providing microsecond latency.

\textbf{Features:}
\begin{itemize}
  \item Fully managed, highly available cache
  \item Reduces read latency from milliseconds to microseconds
  \item No application code changes (drop-in compatible)
  \item Supports eventually consistent and strongly consistent reads
\end{itemize}


\textbf{Use cases:}
\begin{itemize}
  \item Read-heavy workloads
  \item Gaming leaderboards
  \item Real-time bidding
  \item Applications requiring <1ms response
\end{itemize}


\textbf{Note:} Not included in Free Tier, costs apply

</details>

\textbf{Question 5:} How does DynamoDB pricing work?

<details>
<summary>Click to reveal answer</summary>

\textbf{Answer:}
\textbf{On-Demand Mode:}
\begin{itemize}
  \item Pay per request: \$1.25 per million write requests, \$0.25 per million read requests
  \item Storage: \$0.25 per GB/month
  \item Best for unpredictable workloads
\end{itemize}


\textbf{Provisioned Mode:}
\begin{itemize}
  \item Reserve capacity: \$0.00065 per WCU-hour, \$0.00013 per RCU-hour
  \item Auto-scaling available
  \item Best for predictable, steady workloads
  \item Free Tier: 25 WCU + 25 RCU + 25 GB storage
\end{itemize}


\textbf{Additional costs:}
\begin{itemize}
  \item Backups, restores, global tables, streams
  \item Data transfer out
\end{itemize}


\textbf{Exam tip:} Know the difference between capacity modes and when to use each

</details>

\textbf{Question 6:} Can DynamoDB handle relational data like SQL databases?

<details>
<summary>Click to reveal answer</summary>

\textbf{Answer:}
DynamoDB is NoSQL - not designed for relational patterns like JOINs.

\textbf{What DynamoDB CAN'T do well:}
\begin{itemize}
  \item Multi-table joins
  \item Complex aggregations
  \item Ad-hoc queries
  \item Referential integrity constraints
\end{itemize}


\textbf{What DynamoDB DOES well:}
\begin{itemize}
  \item Key-value lookups
  \item Single-table design patterns
  \item High-scale applications
  \item Low-latency requirements
\end{itemize}


\textbf{Best Practice:}
\begin{itemize}
  \item Denormalize data (duplicate information across items)
  \item Single-table design (advanced pattern)
  \item Use RDS/Aurora for complex relational needs
\end{itemize}


\textbf{Exam tip:} Know when to use DynamoDB vs. RDS

</details>

\subsubsection{Key Takeaways}


\begin{itemize}
  \item \textbf{DynamoDB is NoSQL} - schema-less, scalable, managed service
  \item \textbf{Primary key is critical} - determines data distribution and access patterns
  \item \textbf{Query > Scan} - always design for query access patterns
  \item \textbf{GSIs enable flexibility} - query by non-key attributes
  \item \textbf{25 GB storage always free} - great for learning and small apps
  \item \textbf{On-demand mode} - simplest for beginners, no capacity planning
  \item \textbf{Point-in-time recovery} - protection against mistakes
  \item \textbf{Use with Lambda} - perfect for serverless architectures
  \item \textbf{Not a replacement for RDS} - choose right database for use case
\end{itemize}


\subsubsection{Additional Resources}


\begin{itemize}
  \item \href{https://docs.aws.amazon.com/dynamodb/}{DynamoDB Developer Guide}
  \item \href{https://docs.aws.amazon.com/amazondynamodb/latest/developerguide/best-practices.html}{Best Practices for DynamoDB}
  \item \href{https://docs.aws.amazon.com/amazondynamodb/latest/developerguide/data-modeling.html}{DynamoDB Data Modeling}
  \item \href{https://www.youtube.com/results?search\_query=aws+reinvent+dynamodb}{AWS re:Invent DynamoDB Sessions}
\end{itemize}


---

---

\subsection{Troubleshooting FAQ}


This section addresses common issues encountered across all labs.

\subsubsection{General AWS Console Issues}


\textbf{Q: I can't find a service in the AWS Console}

A:
\begin{itemize}
  \item Use the search bar at the top (type service name)
  \item Check if you're in the correct region (some services are region-specific)
  \item Verify your IAM user has permissions to access that service
  \item Some services have different names (e.g., "Billing and Cost Management" vs. "Billing")
\end{itemize}


\textbf{Q: AWS Console is very slow or timing out}

A:
\begin{itemize}
  \item Clear browser cache and cookies
  \item Try incognito/private browsing mode
  \item Switch to a different browser (Chrome, Firefox, Edge)
  \item Check your internet connection (minimum 5 Mbps recommended)
  \item Try a different region (some regions may have connectivity issues)
  \item Check AWS Service Health Dashboard: https://status.aws.amazon.com
\end{itemize}


\textbf{Q: I'm getting "You are not authorized to perform this operation" errors}

A:
\begin{itemize}
  \item Check IAM user has appropriate permissions (policies attached)
  \item If using IAM user, ensure root account enabled IAM billing access (for billing operations)
  \item Wait 5-10 minutes after creating IAM user (policy propagation delay)
  \item Try logging out and back in
  \item Verify you're in correct account (check account ID)
\end{itemize}


\textbf{Q: Resources I created don't appear in the console}

A:
\begin{itemize}
  \item \textbf{Most common:} Wrong region selected (check region dropdown in top right)
  \item Wait 30-60 seconds and refresh (eventual consistency)
  \item Check filters applied in console (clear all filters)
  \item Verify resource actually created (check for error messages)
  \item Check CloudTrail for creation events
\end{itemize}


\subsubsection{Billing and Cost Issues}


\textbf{Q: I'm being charged even though I'm using Free Tier}

A:
\begin{itemize}
  \item Check Free Tier Dashboard for usage vs. limits
  \item Verify instance types are Free Tier eligible (t2.micro, t3.micro)
  \item Check for resources in multiple regions (Free Tier is per-account, not per-region)
  \item Look for non-Free Tier services (NAT Gateway, Load Balancers)
  \item Data transfer charges (out to internet)
  \item EBS storage beyond 30 GB
  \item RDS Multi-AZ (not Free Tier eligible)
\end{itemize}


\textbf{Q: My billing alarm isn't working}

A:
\begin{itemize}
  \item Verify billing alarm created in us-east-1 region only
  \item Check SNS subscription confirmed (look for confirmation email)
  \item Wait 24 hours for initial data (billing metrics take time to populate)
  \item Ensure charges actually exceed threshold
  \item Verify alarm state is "OK" not "Insufficient data"
\end{itemize}


\textbf{Q: I can't access billing dashboard}

A:
\begin{itemize}
  \item Must be logged in as root user OR
  \item IAM user with billing permissions AND root enabled IAM billing access
  \item Go to Account settings → Enable "Activate IAM Access" to billing
\end{itemize}


\textbf{Q: Unexpected charges after Free Tier ended}

A:
\begin{itemize}
  \item Free Tier expires 12 months after account creation (check start date)
  \item Some services are "always free" (Lambda 1M requests, DynamoDB 25GB)
  \item Set up billing alerts for post-Free Tier monitoring
  \item Review bill in detail to identify costly services
  \item Consider shutting down non-essential resources
\end{itemize}


\subsubsection{EC2 Issues}


\textbf{Q: Cannot connect to EC2 instance via SSH}

A:
\begin{enumerate}
  \item \textbf{Connection timeout:}
\end{enumerate}

\begin{itemize}
  \item Security group allows SSH (port 22) from your IP
  \item Instance in public subnet with public IP
  \item Route table has internet gateway route (0.0.0.0/0 → igw-xxx)
  \item Network ACL allows SSH traffic
  \item Instance is in "running" state
\end{itemize}


\begin{enumerate}
  \item \textbf{Permission denied (publickey):}
\end{enumerate}

\begin{itemize}
  \item Using correct .pem key file
  \item Key file has proper permissions: \texttt{chmod 400 key.pem}
  \item Using correct username (ec2-user for Amazon Linux, ubuntu for Ubuntu)
  \item Key pair matches instance
\end{itemize}


\begin{enumerate}
  \item \textbf{Host key verification failed:}
\end{enumerate}

\begin{itemize}
  \item Type "yes" to accept fingerprint
  \item Or use: \texttt{ssh -o StrictHostKeyChecking=no -i key.pem ec2-user@IP}
\end{itemize}


\textbf{Q: Instance status checks failing}

A:
\begin{itemize}
  \item \textbf{1/2 checks passed:} System reachability failed (AWS hardware issue - stop/start instance)
  \item \textbf{0/2 checks passed:} Both system and instance checks failing (check logs, bad user data script)
  \item Wait 2-3 minutes after launch (checks take time)
  \item View System Log and Instance Screenshot from Actions menu
\end{itemize}


\textbf{Q: User data script didn't run}

A:
\begin{itemize}
  \item SSH to instance: \texttt{cat /var/log/cloud-init-output.log}
  \item Look for errors in script execution
  \item Check syntax (bash scripts need \#!/bin/bash)
  \item Ensure script has proper permissions
  \item User data runs only on first boot (unless configured otherwise)
\end{itemize}


\textbf{Q: Can't access web server on EC2}

A:
\begin{itemize}
  \item Security group allows HTTP (port 80) from 0.0.0.0/0
  \item Web server actually running: \texttt{sudo systemctl status httpd} or \texttt{nginx}
  \item Check from inside: \texttt{curl localhost} (should work)
  \item Using HTTP not HTTPS (http:// not https://)
  \item Check firewall on instance: \texttt{sudo iptables -L}
\end{itemize}


\textbf{Q: Instance type not available / capacity error}

A:
\begin{itemize}
  \item Try different Availability Zone within same region
  \item Try slightly different instance type (t2.micro vs t3.micro)
  \item Wait 30 minutes and retry (capacity fluctuates)
  \item Consider different region
  \item For persistent issues, contact AWS Support
\end{itemize}


\subsubsection{S3 Issues}


\textbf{Q: 403 Forbidden error when accessing S3 website}

A:
\begin{itemize}
  \item Bucket policy allows public read (\texttt{s3:GetObject} for Principal: *)
  \item "Block all public access" is OFF
  \item Static website hosting enabled
  \item Objects actually uploaded to bucket
  \item Bucket policy ARN includes \texttt{/\textit{} at end (\texttt{arn:aws:s3:::bucket-name/}})
  \item Correct bucket website endpoint URL (not regular S3 URL)
\end{itemize}


\textbf{Q: 404 Not Found error on S3 website}

A:
\begin{itemize}
  \item index.html file exists in bucket root (case-sensitive)
  \item File name exactly "index.html" (not Index.html or index.HTML)
  \item Static website hosting enabled
  \item Check if using correct endpoint (website endpoint, not REST endpoint)
  \item Clear browser cache
\end{itemize}


\textbf{Q: Cannot create bucket - name already exists}

A:
\begin{itemize}
  \item S3 bucket names are globally unique across all AWS accounts
  \item Try different name with random numbers: \texttt{my-bucket-12345678}
  \item Bucket names must be DNS-compliant (lowercase, no underscores)
  \item Someone else may have that name (even if deleted <24 hours ago)
\end{itemize}


\textbf{Q: Cannot delete bucket - "Bucket not empty"}

A:
\begin{enumerate}
  \item Go to bucket
  \item Click "Empty" button
  \item Type "permanently delete"
  \item Wait for emptying to complete
  \item Then click "Delete bucket"
\end{enumerate}

\begin{itemize}
  \item Alternative: Enable versioning → Delete all versions → Then delete bucket
  \item Check for incomplete multipart uploads
\end{itemize}


\textbf{Q: S3 uploads are very slow}

A:
\begin{itemize}
  \item Check internet connection speed
  \item Try uploading smaller files first
  \item Use multipart upload for files >100 MB
  \item Consider using AWS CLI or SDKs (faster than console)
  \item Check if browser extensions interfering
\end{itemize}


\subsubsection{VPC and Networking Issues}


\textbf{Q: EC2 instance has no internet access}

A:
\begin{itemize}
  \item Instance in public subnet (check subnet settings)
  \item Instance has public IP or Elastic IP
  \item Subnet's route table has route to Internet Gateway (0.0.0.0/0 → igw-xxx)
  \item Security group allows outbound traffic (default allows all)
  \item Network ACL allows outbound traffic (default allows all)
  \item DNS resolution enabled for VPC
\end{itemize}


\textbf{Q: Cannot SSH between EC2 instances in same VPC}

A:
\begin{itemize}
  \item Security groups allow traffic between instances
  \item Use private IPs (not public IPs) for same-VPC communication
  \item Both instances in subnets with proper routing
  \item Network ACLs allow traffic (if custom NACLs)
\end{itemize}


\textbf{Q: VPC creation fails}

A:
\begin{itemize}
  \item Check VPC limit not exceeded (5 per region by default)
  \item CIDR block doesn't overlap with existing VPCs (if VPC peering planned)
  \item CIDR block valid (/16 to /28 for VPC)
  \item Try different region if persistent issues
\end{itemize}


\textbf{Q: Security group changes don't take effect}

A:
\begin{itemize}
  \item Wait 30-60 seconds (eventual consistency)
  \item Refresh console page
  \item Check correct security group attached to instance
  \item Verify rule syntax (port ranges, protocols, sources)
  \item Security groups are stateful (don't need outbound rules for responses)
\end{itemize}


\subsubsection{RDS Issues}


\textbf{Q: Cannot connect to RDS from EC2}

A:
\begin{itemize}
  \item RDS and EC2 in same VPC
  \item RDS security group allows MySQL/PostgreSQL port from EC2 security group
  \item Using RDS endpoint hostname (not IP address)
  \item Correct port (3306 for MySQL, 5432 for PostgreSQL)
  \item RDS status is "Available"
  \item Credentials correct (case-sensitive)
  \item Not trying to connect from public internet (RDS public access = No)
\end{itemize}


\textbf{Q: RDS creation is slow}

A:
\begin{itemize}
  \item Normal: RDS takes 5-15 minutes to create
  \item Multi-AZ takes longer (15-20 minutes)
  \item Watch status: Creating → Backing up → Available
  \item If stuck for >30 minutes, check CloudTrail for errors
\end{itemize}


\textbf{Q: RDS costs more than expected}

A:
\begin{itemize}
  \item Check instance class (db.t3.micro is Free Tier)
  \item Multi-AZ doubles costs (not Free Tier eligible)
  \item Backup storage over 20 GB
  \item Provisioned IOPS costs extra
  \item Enabled Enhanced Monitoring (\$)
  \item Verify "Free Tier" template used during creation
\end{itemize}


\subsubsection{IAM Issues}


\textbf{Q: IAM user can't sign in}

A:
\begin{itemize}
  \item Using IAM user sign-in URL (not root sign-in page)
  \item Sign-in URL format: \texttt{https://account-id.signin.aws.amazon.com/console}
  \item Or: \texttt{https://account-alias.signin.aws.amazon.com/console}
  \item Username and password correct (case-sensitive)
  \item User has console access enabled (not just programmatic)
  \item Account not locked after multiple failed attempts (wait 15 minutes)
\end{itemize}


\textbf{Q: IAM user has no permissions despite policy attached}

A:
\begin{itemize}
  \item Wait 5-10 minutes for policy propagation
  \item Check policy attached to user OR group user belongs to
  \item Policy JSON syntax correct (use policy validator)
  \item No explicit Deny statements (Deny overrides Allow)
  \item Check if SCPs (Service Control Policies) limiting access (AWS Organizations)
\end{itemize}


\textbf{Q: Cannot enable MFA}

A:
\begin{itemize}
  \item Phone time synchronized with internet time
  \item Scan QR code successfully with authenticator app
  \item Enter two consecutive codes (not same code twice)
  \item Codes entered quickly (expire every 30 seconds)
  \item Try different authenticator app if persistent issues
\end{itemize}


\subsubsection{Lambda Issues}


\textbf{Q: Lambda function fails with timeout}

A:
\begin{itemize}
  \item Increase timeout setting (default 3 seconds, max 15 minutes)
  \item Check function actually completes within timeout
  \item Look for infinite loops or blocking operations
  \item Review CloudWatch logs for execution time
\end{itemize}


\textbf{Q: Lambda function fails with "Permission denied"}

A:
\begin{itemize}
  \item Lambda execution role has required permissions
  \item If accessing S3: Role needs \texttt{s3:GetObject}, \texttt{s3:PutObject}
  \item If accessing DynamoDB: Role needs appropriate DynamoDB permissions
  \item Check CloudWatch Logs for specific permission errors
\end{itemize}


\textbf{Q: API Gateway returns "Internal Server Error"}

A:
\begin{itemize}
  \item Check Lambda function CloudWatch logs for actual error
  \item Function returning proper response format (statusCode, body, headers)
  \item Integration configured correctly (Lambda proxy integration recommended)
  \item API deployed (must redeploy after changes)
\end{itemize}


\textbf{Q: Lambda function can't connect to RDS/internet}

A:
\begin{itemize}
  \item If Lambda in VPC: VPC needs NAT Gateway for internet access
  \item Security groups allow traffic between Lambda and RDS
  \item Lambda execution role has \texttt{ec2:CreateNetworkInterface} permission
  \item Timeout increased (VPC adds latency)
\end{itemize}


\subsubsection{CloudFormation Issues}


\textbf{Q: Stack creation failed - rollback initiated}

A:
\begin{itemize}
  \item Check "Events" tab for specific error message
  \item Common: Resource name conflict (already exists)
  \item Common: Insufficient permissions
  \item Common: Invalid parameter values
  \item Fix template and create new stack (or update if supported)
\end{itemize}


\textbf{Q: Stack stuck in CREATE\\textit{IN\}PROGRESS}

A:
\begin{itemize}
  \item Check Events for last completed action
  \item Some resources take time (RDS 10-15 min, NAT Gateway 3-5 min)
  \item If truly stuck >30 min, delete stack and recreate
  \item Check service limits (might be at capacity)
\end{itemize}


\textbf{Q: Can't delete stack - resource dependencies}

A:
\begin{itemize}
  \item Empty S3 buckets before deleting stack
  \item Remove ENIs (Elastic Network Interfaces) attached to Lambda/RDS
  \item Delete dependencies manually then retry
  \item Check "Retain" resource policy (some resources protected)
\end{itemize}


\subsubsection{Auto Scaling and Load Balancer Issues}


\textbf{Q: Auto Scaling not launching instances}

A:
\begin{itemize}
  \item Check Auto Scaling group Activity tab for errors
  \item Verify launch template valid (AMI exists, instance type available)
  \item Not at maximum capacity limit
  \item Subnet has available IP addresses
  \item Service limits not exceeded (EC2 instance limit)
\end{itemize}


\textbf{Q: Load Balancer shows all targets unhealthy}

A:
\begin{itemize}
  \item Health check path correct (must return 200 status)
  \item Security group allows health check traffic from load balancer
  \item Application actually running on instances
  \item Health check interval/threshold appropriate (increase grace period)
  \item Check target group health check settings
\end{itemize}


\textbf{Q: Load Balancer returns 503 Service Unavailable}

A:
\begin{itemize}
  \item No healthy targets available
  \item All targets failing health checks
  \item Target group has no registered targets
  \item Instances in correct subnets
  \item Wait for instances to pass health checks (2 consecutive successes)
\end{itemize}


\subsubsection{DynamoDB Issues}


\textbf{Q: Query returns no results}

A:
\begin{itemize}
  \item Partition key value exact match (case-sensitive)
  \item Using correct index (base table vs GSI)
  \item Items actually exist with that key
  \item Try Scan to verify items in table
\end{itemize}


\textbf{Q: DynamoDB throttling errors (ProvisionedThroughputExceededException)}

A:
\begin{itemize}
  \item Switch to On-Demand capacity mode
  \item Or increase provisioned capacity (WCU/RCU)
  \item Enable Auto Scaling for provisioned mode
  \item Implement exponential backoff in application
  \item Check for hot partition (one key getting all traffic)
\end{itemize}


\textbf{Q: Cannot create GSI - limit exceeded}

A:
\begin{itemize}
  \item Maximum 20 GSIs per table
  \item Delete unused indexes
  \item Consider if Scan with filter acceptable
\end{itemize}


\subsubsection{Common Error Messages Decoded}


\textbf{"InvalidParameterValue":}
\begin{itemize}
  \item A parameter you provided has invalid value
  \item Check AWS documentation for valid values
  \item Common: Wrong availability zone, invalid CIDR block, bad instance type
\end{itemize}


\textbf{"UnauthorizedOperation":}
\begin{itemize}
  \item IAM user/role lacks required permission
  \item Add necessary policy to user/role
  \item Check typos in action names
\end{itemize}


\textbf{"ResourceNotFoundException":}
\begin{itemize}
  \item Resource you're trying to access doesn't exist
  \item Verify resource ID correct
  \item Check correct region
  \item Resource may have been deleted
\end{itemize}


\textbf{"LimitExceeded":}
\begin{itemize}
  \item Hit AWS service limit (EC2 instances, VPCs, security groups)
  \item Request limit increase via Service Quotas console
  \item Or clean up unused resources
\end{itemize}


\textbf{"DependencyViolation":}
\begin{itemize}
  \item Trying to delete resource with dependencies
  \item Example: Can't delete security group attached to running instance
  \item Remove dependencies first
\end{itemize}


\subsubsection{Getting Help}


\textbf{When to ask for help:}
\begin{itemize}
  \item Tried all troubleshooting steps
  \item Issue persists for >30 minutes
  \item Billing concern (unexpected charges)
  \item Service outage suspected
\end{itemize}


\textbf{Where to get help:}
\begin{enumerate}
  \item \textbf{AWS Documentation:} Most comprehensive - https://docs.aws.amazon.com
  \item \textbf{AWS re:Post:} Community forum - https://repost.aws
  \item \textbf{AWS Support:} If you have paid support plan
  \item \textbf{Stack Overflow:} Tag questions with [amazon-web-services]
  \item \textbf{AWS Service Health Dashboard:} Check for outages - https://status.aws.amazon.com
\end{enumerate}


\textbf{Information to provide when asking for help:}
\begin{itemize}
  \item AWS service name
  \item Region
  \item Exact error message
  \item Steps taken so far
  \item Screenshots (redact sensitive info)
  \item Resource IDs
  \item Timeline (when did issue start)
\end{itemize}


\textbf{What NOT to share:}
\begin{itemize}
  \item AWS Access Keys or Secret Keys
  \item Passwords
  \item Full ARNs with account IDs (can be partially redacted)
  \item Credit card information
\end{itemize}


---

\subsection{Additional Practice Recommendations}


\subsubsection{Explore More AWS Services}


\begin{enumerate}
  \item \textbf{DynamoDB}
\end{enumerate}

\begin{itemize}
  \item Create NoSQL table
  \item Add items using console
  \item Query and scan data
  \item Explore indexes
\end{itemize}


\begin{enumerate}
  \item \textbf{CloudTrail}
\end{enumerate}

\begin{itemize}
  \item Enable trail
  \item View API call history
  \item Search for specific events
  \item Understand audit logging
\end{itemize}


\begin{enumerate}
  \item \textbf{AWS Config}
\end{enumerate}

\begin{itemize}
  \item Set up Config
  \item Track resource configurations
  \item View configuration timeline
  \item Create compliance rules
\end{itemize}


\begin{enumerate}
  \item \textbf{Auto Scaling}
\end{enumerate}

\begin{itemize}
  \item Create launch template
  \item Set up Auto Scaling group
  \item Configure scaling policies
  \item Test scale-out/scale-in
\end{itemize}


\begin{enumerate}
  \item \textbf{Elastic Beanstalk}
\end{enumerate}

\begin{itemize}
  \item Deploy sample application
  \item Explore managed environment
  \item View logs and monitoring
  \item Update application
\end{itemize}


\subsubsection{AWS CLI Practice}


\textbf{Install AWS CLI:}

\begin{enumerate}
  \item Follow instructions: \href{https://aws.amazon.com/cli/}{https://aws.amazon.com/cli/}
  \item Configure credentials: \texttt{aws configure}
  \item Run basic commands:
\end{enumerate}

   \texttt{`}bash
   aws s3 ls
   aws ec2 describe-instances
   aws iam list-users
   \texttt{`}

\subsubsection{Multi-Region Exploration}


\begin{enumerate}
  \item \textbf{Compare regions:}
\end{enumerate}

\begin{itemize}
  \item Note service availability differences
  \item Check pricing variations
  \item Test latency from your location
\end{itemize}


\begin{enumerate}
  \item \textbf{Practice disaster recovery:}
\end{enumerate}

\begin{itemize}
  \item Create resources in multiple regions
  \item Understand cross-region replication
  \item Learn about global services vs. regional
\end{itemize}


\subsubsection{Cost Management}


\begin{enumerate}
  \item \textbf{Monitor Free Tier Dashboard daily}
  \item \textbf{Set up multiple budgets} for different services
  \item \textbf{Review Cost Explorer weekly}
  \item \textbf{Practice using Pricing Calculator} for different scenarios
  \item \textbf{Understand billing cycle} and payment methods
\end{enumerate}


\subsubsection{Documentation Habits}


\begin{enumerate}
  \item \textbf{Take screenshots} of each step
  \item \textbf{Create diagrams} of architectures built
  \item \textbf{Write notes} on lessons learned
  \item \textbf{Document errors} and solutions
  \item \textbf{Build your own cheat sheet}
\end{enumerate}


\subsubsection{Advanced Labs (After Basics)}


\begin{enumerate}
  \item \textbf{VPC Peering}
  \item \textbf{Load Balancer with Auto Scaling}
  \item \textbf{RDS Multi-AZ Deployment}
  \item \textbf{CloudFront with S3 Origin}
  \item \textbf{Lambda with SQS and DynamoDB}
  \item \textbf{CodePipeline for CI/CD}
\end{enumerate}


---

\subsection{Final Important Reminders}


\subsubsection{Always Clean Up Resources}


\begin{keypoint}
\textbf{Critical:} Leaving resources running can result in unexpected charges after Free Tier expires.
\end{keypoint}


\textbf{Daily checklist:}
\begin{itemize}
  \item [ ] Terminate all EC2 instances
  \item [ ] Delete RDS databases
  \item [ ] Empty and delete S3 buckets
  \item [ ] Delete CloudFormation stacks
  \item [ ] Remove unused EBS volumes and snapshots
  \item [ ] Check billing dashboard
\end{itemize}


\subsubsection{Monitor Costs}


\begin{itemize}
  \item Check \textbf{Free Tier usage} daily
  \item Review \textbf{billing alerts}
  \item Set up \textbf{budgets} for each service
  \item Download \textbf{monthly bills} for records
  \item Use \textbf{Cost Explorer} to track trends
\end{itemize}


\subsubsection{Security Best Practices}


\begin{itemize}
  \item \textbf{Never share} root account credentials
  \item \textbf{Enable MFA} on all accounts
  \item \textbf{Use IAM roles} instead of access keys when possible
  \item \textbf{Follow principle of least privilege}
  \item \textbf{Regularly rotate} credentials
  \item \textbf{Review} security group rules frequently
\end{itemize}


\subsubsection{Learn More}


\begin{itemize}
  \item \textbf{AWS Documentation:} \href{https://docs.aws.amazon.com}{https://docs.aws.amazon.com}
  \item \textbf{AWS Skill Builder:} Free training courses
  \item \textbf{AWS Workshops:} \href{https://workshops.aws}{https://workshops.aws}
  \item \textbf{AWS YouTube:} Official tutorials and demos
  \item \textbf{AWS re:Post:} Community Q\&A forum
\end{itemize}


---

\subsection{Congratulations!}


You've completed all 10 hands-on labs! You now have practical experience with:

\begin{itemize}
  \item Billing and cost management
  \item IAM security
  \item EC2 compute instances
  \item S3 object storage
  \item VPC networking
  \item RDS managed databases
  \item CloudWatch monitoring
  \item Lambda serverless functions
  \item CloudFormation infrastructure as code
\end{itemize}


This hands-on knowledge will significantly help you on the AWS Cloud Practitioner exam and in real-world AWS usage.

\textbf{Next steps:}
\begin{itemize}
  \item Review weak areas
  \item Take practice exams
  \item Study theory in conjunction with practical experience
  \item Schedule your certification exam
\end{itemize}


Good luck on your AWS Cloud Practitioner journey!

---

\href{06-study-plan.md}{← Back to Study Plan} | \href{README.md}{Return to Main Guide →}
