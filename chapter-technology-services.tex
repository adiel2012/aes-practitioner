\chapter{Domain 3: Cloud Technology and Services}




This is the largest domain of the AWS Certified Cloud Practitioner exam, covering core AWS services across compute, storage, networking, databases, and more.

\subsection{Table of Contents}


\begin{itemize}
  \item \href{\#aws-global-infrastructure}{AWS Global Infrastructure}
  \item \href{\#compute-services}{Compute Services}
  \item \href{\#storage-services}{Storage Services}
  \item \href{\#database-services}{Database Services}
  \item \href{\#networking-and-content-delivery}{Networking and Content Delivery}
  \item \href{\#management-and-governance}{Management and Governance}
  \item \href{\#additional-services}{Additional Services}
  \item \href{\#review-questions}{Review Questions}
\end{itemize}


---

\subsection{AWS Global Infrastructure}


\subsubsection{Regions}


\textbf{Geographic areas with multiple Availability Zones}

\begin{itemize}
  \item \textbf{Current Count}: 33 Regions worldwide (and growing)
  \item \textbf{Isolation}: Each Region is completely isolated from other Regions
  \item \textbf{Selection Criteria}:
  \item \textbf{Compliance requirements}: Data sovereignty and regulatory requirements
  \item \textbf{Proximity to users}: Lower latency for end users
  \item \textbf{Available services}: Not all services are available in all Regions
  \item \textbf{Pricing}: Costs vary by Region
\end{itemize}


\begin{examtip}
\textbf{Exam Tip}: Choose the right Region based on compliance, latency, service availability, and cost considerations.
\end{examtip}


\subsubsection{Availability Zones (AZs)}


\textbf{Discrete data centers within a Region}

\begin{itemize}
  \item \textbf{Composition}: One or more discrete data centers per AZ
  \item \textbf{Redundancy}: Each AZ has redundant power, networking, and connectivity
  \item \textbf{Physical Separation}: AZs are physically separated within a Region
  \item \textbf{Connectivity}: Connected with high-bandwidth, low-latency networking
  \item \textbf{Minimum Count}: At least 3 AZs per Region (most have more)
  \item \textbf{High Availability}: Deploy resources across multiple AZs for fault tolerance
\end{itemize}


\textbf{Key Benefits}:
\begin{itemize}
  \item Fault isolation
  \item High availability through redundancy
  \item Disaster recovery within a Region
\end{itemize}


\textbf{Real-World Example}: Netflix deploys its content delivery infrastructure across multiple AZs within each Region. If one AZ experiences issues, their application automatically routes traffic to healthy AZs, ensuring uninterrupted streaming for millions of users.

\textbf{Common Configuration Mistakes}:
\begin{enumerate}
  \item Deploying all resources in a single AZ (no fault tolerance)
  \item Not considering cross-AZ data transfer costs
  \item Assuming AZ names (us-east-1a) are the same across AWS accounts (they're randomized)
  \item Not testing failover between AZs before production deployment
\end{enumerate}


\textbf{Service Limits by Infrastructure}:
\begin{itemize}
  \item Maximum VPCs per Region: 5 (soft limit, can be increased)
  \item Maximum subnets per VPC: 200
  \item Elastic IPs per Region: 5 (soft limit)
  \item VPC Peering connections per VPC: 125
\end{itemize}


\subsubsection{Edge Locations}


\textbf{Content delivery endpoints worldwide}

\begin{itemize}
  \item \textbf{Count}: 400+ Edge Locations globally
  \item \textbf{Primary Use}: CloudFront content caching
  \item \textbf{Performance}: Lower latency for end users
  \item \textbf{Services}: Also used by Route 53, AWS Shield, and AWS WAF
  \item \textbf{Coverage}: More Edge Locations than Regions
\end{itemize}


\subsubsection{AWS Local Zones}


\textbf{Region extensions for ultra-low latency}

\begin{itemize}
  \item Extension of a Region placed closer to specific geographic areas
  \item Single-digit millisecond latency to end users
  \item Ideal for latency-sensitive applications (gaming, live video, AR/VR)
  \item Not available in all locations
\end{itemize}


\subsubsection{AWS Wavelength}


\textbf{5G edge computing}

\begin{itemize}
  \item Embeds AWS compute and storage services within 5G networks
  \item Ultra-low latency applications
  \item Mobile edge computing use cases
  \item Reduces data routing to application servers
\end{itemize}


\subsubsection{AWS Outposts}


\textbf{On-premises AWS infrastructure}

\begin{itemize}
  \item Fully managed service extending AWS infrastructure to on-premises facilities
  \item Same AWS APIs, tools, and hardware available on-premises
  \item Enables true hybrid cloud deployments
  \item AWS manages and maintains the infrastructure
  \item Run AWS services locally with consistent experience
\end{itemize}


\begin{keypoint}
\textbf{Key Point}: Remember the hierarchy: \textbf{Regions} contain \textbf{Availability Zones}. \textbf{Edge Locations} are separate and used primarily for content delivery.
\end{keypoint}


\subsubsection{Global Infrastructure Comparison Table}


\begin{longtable}{lllll}
\toprule
\textbf{Component} & \textbf{Count} & \textbf{Primary Use} & \textbf{Redundancy Level} & \textbf{Latency} \\
\midrule
\textbf{Regions} & 33+ & Full AWS service deployment & Isolated from each other & Variable (based on distance) \\
\textbf{Availability Zones} & 100+ (3+ per Region) & Fault-tolerant deployments & Within Region & Low (single-digit ms) \\
\textbf{Edge Locations} & 400+ & Content caching & N/A & Minimal to end users \\
\textbf{Local Zones} & 16+ & Ultra-low latency compute & Extension of parent Region & <10ms \\
\textbf{Wavelength Zones} & 20+ & 5G edge computing & Within telecom networks & <1ms \\
\bottomrule
\end{longtable}

\subsubsection{Infrastructure Selection Flowchart}


\begin{verbatim}
START: Where should I deploy my resources?
│
├─> Need global distribution?
│   ├─> YES: Deploy in multiple Regions
│   │   └─> Consider: CloudFront for content, Route 53 for DNS routing
│   │
│   └─> NO: Single Region deployment
│       │
│       ├─> Need high availability?
│       │   ├─> YES: Deploy across multiple AZs
│       │   │   └─> Use: Multi-AZ RDS, ALB across AZs, Auto Scaling
│       │   │
│       │   └─> NO: Single AZ (only for dev/test)
│       │
│       └─> Need ultra-low latency?
│           ├─> For specific metro areas: Use Local Zones
│           ├─> For mobile 5G apps: Use Wavelength
│           └─> For on-premises: Use Outposts
\end{verbatim}

\subsubsection{Migration Scenario: Data Center to AWS}


\textbf{Scenario}: Company with on-premises data center in Chicago needs to migrate to AWS.

\textbf{Current Setup}:
\begin{itemize}
  \item 100 servers across 2 data centers
  \item Customers primarily in US and Europe
  \item Compliance requires data residency in US
\end{itemize}


\textbf{AWS Migration Strategy}:

\begin{enumerate}
  \item \textbf{Region Selection}: Choose us-east-1 (N. Virginia) for primary and us-west-2 (Oregon) for DR
\end{enumerate}

\begin{itemize}
  \item Reason: Established Regions with all services available
  \item Cost: Lower pricing than newer Regions
  \item Latency: Good for US customers
\end{itemize}


\begin{enumerate}
  \item \textbf{Network Connectivity}:
\end{enumerate}

\begin{itemize}
  \item Phase 1: Site-to-Site VPN (immediate, low cost)
  \item Phase 2: Direct Connect (after 3 months, for production traffic)
  \item Cost: VPN \textasciitilde{}\$0.05/hour, Direct Connect \textasciitilde{}\$0.30/hour (1 Gbps)
\end{itemize}


\begin{enumerate}
  \item \textbf{High Availability Architecture}:
\end{enumerate}

\begin{itemize}
  \item Deploy across 3 AZs in us-east-1
  \item Application Load Balancer across all AZs
  \item RDS Multi-AZ for databases
  \item S3 for object storage (automatically multi-AZ)
\end{itemize}


\begin{enumerate}
  \item \textbf{European Customers}:
\end{enumerate}

\begin{itemize}
  \item CloudFront distribution with origin in us-east-1
  \item Edge locations in Europe cache content
  \item Route 53 latency-based routing
\end{itemize}


\textbf{Cost Comparison}:
\begin{itemize}
  \item On-premises: \$50,000/month (hardware, power, cooling, staff)
  \item AWS initial: \$35,000/month (no hardware investment)
  \item AWS optimized (after 6 months with RIs): \$22,000/month
  \item Savings: 56\% reduction in monthly costs
\end{itemize}


---

\subsection{Compute Services}


\subsubsection{Amazon EC2 (Elastic Compute Cloud)}


\textbf{Virtual servers in the cloud}

Amazon EC2 provides resizable compute capacity in the cloud, allowing you to launch virtual servers (instances) on demand.

\paragraph{Instance Types}


\begin{longtable}{lll}
\toprule
\textbf{Category} & \textbf{Use Case} & \textbf{Examples} \\
\midrule
\textbf{General Purpose} & Balanced compute, memory, networking & T3, M5 \\
\textbf{Compute Optimized} & High-performance processors & C5, C6 \\
\textbf{Memory Optimized} & Large datasets in memory & R5, X1 \\
\textbf{Storage Optimized} & High sequential read/write access & I3, D2 \\
\textbf{Accelerated Computing} & GPU, FPGA workloads & P3, G4 \\
\bottomrule
\end{longtable}

\textbf{Detailed Instance Type Comparison}

\begin{longtable}{llllll}
\toprule
\textbf{Instance Family} & \textbf{vCPU Range} & \textbf{Memory Range} & \textbf{Network Performance} & \textbf{Best Use Cases} & \textbf{Example Workloads} \\
\midrule
\textbf{T3/T4g} & 2-8 & 0.5-32 GB & Up to 5 Gbps & Burstable, web servers & Small websites, dev environments \\
\textbf{M6i/M7g} & 2-128 & 8-512 GB & Up to 50 Gbps & Balanced applications & App servers, backend systems \\
\textbf{C6i/C7g} & 2-128 & 4-256 GB & Up to 50 Gbps & Compute-intensive & Video encoding, gaming servers, HPC \\
\textbf{R6i/R7g} & 2-128 & 16-1024 GB & Up to 50 Gbps & Memory-intensive & In-memory databases, big data analytics \\
\textbf{X2iedn} & 16-128 & 256-4096 GB & Up to 100 Gbps & Extreme memory & SAP HANA, real-time analytics \\
\textbf{I4i} & 2-128 & 16-1024 GB & Up to 75 Gbps & Storage-intensive & NoSQL databases, data warehousing \\
\textbf{P4d} & 96 & 1152 GB & 400 Gbps & ML training & Deep learning, GPU clusters \\
\textbf{G5} & 4-96 & 16-768 GB & Up to 100 Gbps & Graphics-intensive & ML inference, graphics rendering \\
\bottomrule
\end{longtable}

\textbf{Real-World Use Case: E-Commerce Website}

\textbf{Scenario}: Online retailer with variable traffic patterns, peak during holidays.

\textbf{Solution Architecture}:
\begin{itemize}
  \item \textbf{Frontend Web Servers}: T3.medium instances (burstable for normal traffic)
  \item Cost: \$0.0416/hour = \textasciitilde{}\$30/month each
  \item Auto Scaling: 2-20 instances based on demand
  \item Normal: 4 instances = \$120/month
  \item Peak: 15 instances = \$450/month
  \item \textbf{Application Servers}: M5.large instances (consistent performance)
  \item Cost: \$0.096/hour = \textasciitilde{}\$70/month each
  \item Reserved Instances (3-year): 75\% discount = \$17.50/month each
  \item Deploy: 8 instances = \$140/month with RIs
  \item \textbf{Database}: R5.xlarge (memory-optimized for database)
  \item Cost: \$0.252/hour = \textasciitilde{}\$184/month
  \item Reserved Instance: \$46/month
\end{itemize}


\textbf{Total Monthly Cost}:
\begin{itemize}
  \item Normal traffic: \$306/month
  \item Peak traffic: \$636/month
  \item Average: \textasciitilde{}\$400/month vs \$2,000+ on-premises
\end{itemize}


\textbf{Performance Metrics}:
\begin{itemize}
  \item Page load time: <2 seconds
  \item Database query response: <100ms
  \item Handles 10,000 concurrent users during peak
  \item 99.99\% uptime with Multi-AZ deployment
\end{itemize}


\paragraph{EC2 Pricing Models}


\#\#\#\#\# 1. On-Demand Instances

\begin{itemize}
  \item \textbf{Billing}: Pay per hour or per second (minimum 60 seconds)
  \item \textbf{Commitment}: No upfront commitment or long-term contract
  \item \textbf{Cost}: Highest per-hour cost
  \item \textbf{Best For}:
  \item Unpredictable workloads
  \item Short-term, spiky workloads
  \item Testing and development
  \item Applications with flexible start/stop times
\end{itemize}


\#\#\#\#\# 2. Reserved Instances (RI)

\begin{itemize}
  \item \textbf{Commitment}: 1 or 3 year terms
  \item \textbf{Discount}: Up to 75\% compared to On-Demand pricing
  \item \textbf{Types}:
  \item \textbf{Standard RI}: Maximum discount, cannot change instance type
  \item \textbf{Convertible RI}: Lower discount, can change instance type/family
  \item \textbf{Payment Options}: All upfront, partial upfront, or no upfront
  \item \textbf{Best For}: Steady-state workloads with predictable usage
\end{itemize}


\#\#\#\#\# 3. Savings Plans

\begin{itemize}
  \item \textbf{Commitment}: Consistent compute usage measured in \$/hour
  \item \textbf{Term}: 1 or 3 year commitment
  \item \textbf{Discount}: Up to 72\% compared to On-Demand
  \item \textbf{Flexibility}: More flexible than Reserved Instances
  \item \textbf{Coverage}: Applies to EC2, Lambda, and Fargate
  \item \textbf{Best For}: Flexible compute usage with commitment to consistent spend
\end{itemize}


\#\#\#\#\# 4. Spot Instances

\begin{itemize}
  \item \textbf{Mechanism}: Bid on unused EC2 capacity
  \item \textbf{Discount}: Up to 90\% compared to On-Demand pricing
  \item \textbf{Interruption}: Can be terminated by AWS with 2-minute warning
  \item \textbf{Best For}:
  \item Fault-tolerant applications
  \item Flexible workloads
  \item Batch processing
  \item Big data analysis
  \item \textbf{Not Suitable For}: Critical workloads or databases
\end{itemize}


\#\#\#\#\# 5. Dedicated Hosts

\begin{itemize}
  \item \textbf{Definition}: Physical EC2 server dedicated exclusively to your use
  \item \textbf{Cost}: Most expensive option
  \item \textbf{Use Cases}:
  \item Regulatory compliance requirements
  \item Server-bound software licenses
  \item Socket/core visibility needed for licensing
  \item \textbf{Control}: Full control over instance placement
\end{itemize}


\#\#\#\#\# 6. Dedicated Instances

\begin{itemize}
  \item \textbf{Definition}: Instances run on hardware dedicated to a single customer
  \item \textbf{Sharing}: May share hardware with other instances in your account
  \item \textbf{Cost}: Less expensive than Dedicated Hosts
  \item \textbf{Isolation}: Hardware-level isolation from other AWS accounts
\end{itemize}


\textbf{EC2 Pricing Model Comparison Table}

\begin{longtable}{llllll}
\toprule
\textbf{Pricing Model} & \textbf{Discount vs On-Demand} & \textbf{Commitment} & \textbf{Interruption Risk} & \textbf{Best For} & \textbf{Payment Options} \\
\midrule
\textbf{On-Demand} & 0\% (baseline) & None & None & Short-term, unpredictable & Per hour/second \\
\textbf{Reserved (Standard)} & Up to 75\% & 1 or 3 years & None & Steady-state workloads & All/Partial/No upfront \\
\textbf{Reserved (Convertible)} & Up to 66\% & 1 or 3 years & None & Flexible steady workloads & All/Partial/No upfront \\
\textbf{Savings Plans} & Up to 72\% & 1 or 3 years & None & Flexible compute usage & All/Partial/No upfront \\
\textbf{Spot Instances} & Up to 90\% & None & Can be interrupted & Fault-tolerant, flexible & Per hour \\
\textbf{Dedicated Hosts} & Varies & 1 or 3 years & None & Licensing, compliance & On-Demand or Reserved \\
\bottomrule
\end{longtable}

\textbf{Cost Comparison Example: m5.xlarge in us-east-1}

\begin{longtable}{llllll}
\toprule
\textbf{Scenario} & \textbf{Pricing Model} & \textbf{Hourly Cost} & \textbf{Monthly Cost (730 hrs)} & \textbf{Annual Cost} & \textbf{3-Year Total} \\
\midrule
\textbf{Dev/Test (8hrs/day, 22 days/month)} & On-Demand & \$0.192 & \$337 & \$4,147 & \$12,441 \\
\textbf{Production (24/7)} & On-Demand & \$0.192 & \$140 & \$1,681 & \$5,043 \\
\textbf{Production (24/7)} & Standard RI (3yr, all upfront) & \$0.112 & \$82 & \$981 & \$2,943 \\
\textbf{Production (24/7)} & Savings Plan (3yr) & \$0.118 & \$86 & \$1,032 & \$3,096 \\
\textbf{Batch Processing (avg 50\% uptime)} & Spot & \$0.038 & \$14 & \$168 & \$504 \\
\bottomrule
\end{longtable}

\textbf{Cost Optimization Strategy}:
\begin{enumerate}
  \item \textbf{Baseline workload}: Use Standard RIs or Savings Plans (75\% savings)
  \item \textbf{Variable workload}: Use Savings Plans for flexibility
  \item \textbf{Peak capacity}: Auto Scale with On-Demand
  \item \textbf{Batch/fault-tolerant}: Use Spot Instances (90\% savings)
\end{enumerate}


\textbf{Common Configuration Mistakes}:
\begin{enumerate}
  \item Running On-Demand for steady-state workloads (missing 75\% savings)
  \item Using Spot Instances for databases or critical workloads
  \item Over-provisioning instance size (wasting resources)
  \item Not enabling detailed monitoring for performance optimization
  \item Forgetting to delete stopped instances (still incurs EBS charges)
  \item Not using Auto Scaling (manually managing capacity)
  \item Storing data on instance store for persistent data (data loss on stop)
  \item Not tagging instances (difficult cost allocation)
\end{enumerate}


\textbf{Service Limits and Quotas}:
\begin{itemize}
  \item On-Demand instance limit: Varies by instance family (typically 20-1280 vCPUs)
  \item Spot instance limit: 20 Spot instances per Region (soft limit)
  \item Reserved Instances: 20 per month (soft limit)
  \item Elastic IPs: 5 per Region (soft limit)
  \item EBS volumes: 5,000 per Region (soft limit)
  \item EBS snapshots: 10,000 per Region (soft limit)
\end{itemize}


\textbf{Monitoring and Troubleshooting}:

\begin{enumerate}
  \item \textbf{CloudWatch Metrics} (5-minute intervals, free):
\end{enumerate}

\begin{itemize}
  \item CPUUtilization
  \item NetworkIn/NetworkOut
  \item DiskReadOps/DiskWriteOps
  \item StatusCheckFailed
\end{itemize}


\begin{enumerate}
  \item \textbf{Detailed Monitoring} (1-minute intervals, paid):
\end{enumerate}

\begin{itemize}
  \item Enable for production workloads
  \item Cost: \$0.14 per instance per month
  \item Better for Auto Scaling responsiveness
\end{itemize}


\begin{enumerate}
  \item \textbf{Common Issues}:
\end{enumerate}

\begin{itemize}
  \item High CPU: Resize instance or optimize application
  \item High memory: Use memory-optimized instance type
  \item Network bottleneck: Use enhanced networking or larger instance
  \item Disk I/O bottleneck: Use Provisioned IOPS EBS volumes
\end{itemize}


\begin{enumerate}
  \item \textbf{Troubleshooting Commands}:
\end{enumerate}

   \texttt{`}
   \# Check system status
   aws ec2 describe-instance-status --instance-ids i-1234567890abcdef0

   \# View CloudWatch metrics
   aws cloudwatch get-metric-statistics --namespace AWS/EC2 \textbackslash{}
     --metric-name CPUUtilization --dimensions Name=InstanceId,Value=i-xxx

   \# Check security group rules
   aws ec2 describe-security-groups --group-ids sg-xxx
   \texttt{`}

\paragraph{Auto Scaling}


\textbf{Automatically adjust capacity based on demand}

\begin{itemize}
  \item \textbf{Scaling Actions}:
  \item \textbf{Scale Out}: Add instances when demand increases
  \item \textbf{Scale In}: Remove instances when demand decreases
  \item \textbf{Scaling Policies}:
  \item \textbf{Target Tracking}: Maintain a specific metric (e.g., 50\% CPU utilization)
  \item \textbf{Step Scaling}: Scale based on CloudWatch alarm thresholds
  \item \textbf{Scheduled Scaling}: Scale based on predictable time-based patterns
  \item \textbf{Benefits}:
  \item Improved availability
  \item Cost optimization
  \item Fault tolerance
  \item Works seamlessly with Elastic Load Balancing
\end{itemize}


\subsubsection{AWS Lambda}


\textbf{Serverless compute service}

Run code without provisioning or managing servers.

\textbf{Key Features}:
\begin{itemize}
  \item \textbf{Zero Server Management}: No infrastructure to provision or manage
  \item \textbf{Automatic Scaling}: Scales automatically from a few requests to thousands
  \item \textbf{Subsecond Metering}: Pay only for compute time consumed (billed per 100ms)
  \item \textbf{Language Support}: Python, Node.js, Java, Go, C\#, Ruby, PowerShell
  \item \textbf{Execution Limit}: Maximum 15 minutes per execution
  \item \textbf{Event-Driven}: Triggered by events from AWS services or custom applications
\end{itemize}


\textbf{Benefits}:
\begin{itemize}
  \item No server management overhead
  \item Continuous automatic scaling
  \item Cost-effective for variable workloads
  \item Built-in high availability and fault tolerance
\end{itemize}


\textbf{Common Use Cases}:
\begin{itemize}
  \item Real-time file processing
  \item Data transformation and ETL
  \item Serverless backends for web/mobile apps
  \item IoT backends
  \item Scheduled tasks (cron jobs)
\end{itemize}


\textbf{Lambda vs EC2 Comparison}

\begin{longtable}{lll}
\toprule
\textbf{Aspect} & \textbf{AWS Lambda} & \textbf{Amazon EC2} \\
\midrule
\textbf{Management} & Fully managed, zero administration & You manage OS, patches, scaling \\
\textbf{Scaling} & Automatic, instant (0-10,000+ concurrent) & Manual or Auto Scaling (minutes) \\
\textbf{Pricing} & Per request + duration (100ms increments) & Per hour/second of instance runtime \\
\textbf{Max Duration} & 15 minutes per invocation & Unlimited (runs continuously) \\
\textbf{Cold Start} & Yes (50-200ms initial delay) & No (always running) \\
\textbf{State} & Stateless (ephemeral storage) & Stateful (persistent storage) \\
\textbf{Best For} & Event-driven, sporadic workloads & Long-running, steady workloads \\
\bottomrule
\end{longtable}

\textbf{Real-World Use Case: Image Processing Service}

\textbf{Scenario}: Photography platform processes user-uploaded images (resize, thumbnail, watermark).

\textbf{Lambda Solution}:
\begin{verbatim}
Architecture:
User uploads image to S3
  └─> S3 triggers Lambda function
      └─> Lambda processes image
          └─> Saves processed images back to S3
          └─> Updates DynamoDB with metadata
\end{verbatim}

\textbf{Cost Analysis} (1 million images per month):
\begin{itemize}
  \item Average processing time: 3 seconds per image
  \item Memory allocation: 1024 MB
  \item Compute: 1M requests × 3 seconds × \$0.0000166667/GB-second = \$50
  \item Requests: 1M × \$0.20 per 1M = \$0.20
  \item \textbf{Total: \$50.20/month}
\end{itemize}


\textbf{EC2 Comparison}:
\begin{itemize}
  \item t3.medium running 24/7: \textasciitilde{}\$30/month (On-Demand)
  \item BUT: Needs management, updates, monitoring
  \item AND: Wastes capacity during low-traffic periods
  \item \textbf{Total with overhead: \$100-150/month}
\end{itemize}


\textbf{Lambda Advantages for this use case}:
\begin{itemize}
  \item 66\% cost savings
  \item Zero server management
  \item Automatic scaling (handles traffic spikes)
  \item Only pay for actual processing time
\end{itemize}


\textbf{Cost Comparison Example: Lambda Pricing}

\begin{longtable}{lllllll}
\toprule
\textbf{Monthly Requests} & \textbf{Avg Duration} & \textbf{Memory} & \textbf{Compute Cost} & \textbf{Request Cost} & \textbf{Total Cost} & \textbf{Equivalent EC2} \\
\midrule
\textbf{100,000} & 200ms & 512 MB & \$0.17 & \$0.02 & \$0.19 & t3.micro (\$7.59) \\
\textbf{1,000,000} & 1 second & 1024 MB & \$16.67 & \$0.20 & \$16.87 & t3.small (\$15.18) \\
\textbf{10,000,000} & 500ms & 512 MB & \$41.67 & \$2.00 & \$43.67 & t3.medium (\$30.37) \\
\textbf{100,000,000} & 200ms & 256 MB & \$33.33 & \$20.00 & \$53.33 & m5.large (\$70.08) \\
\bottomrule
\end{longtable}

\textbf{Integration Example: Serverless Web Application}

\begin{verbatim}
Architecture:
CloudFront (CDN)
  └─> S3 (Static website hosting)
      └─> API Gateway (REST API)
          └─> Lambda (Business logic)
              ├─> DynamoDB (User data)
              ├─> RDS Aurora Serverless (Transactional data)
              └─> S3 (File storage)
\end{verbatim}

\textbf{Benefits}:
\begin{itemize}
  \item No servers to manage
  \item Automatic scaling from 0 to millions of requests
  \item Pay only for actual usage
  \item High availability built-in
\end{itemize}


\textbf{Common Configuration Mistakes}:
\begin{enumerate}
  \item Not setting appropriate timeout (default 3s, max 15min)
  \item Allocating too much or too little memory (affects CPU allocation)
  \item Not handling cold starts for latency-sensitive applications
  \item Storing state in /tmp (lost between invocations, max 10 GB)
  \item Not implementing proper error handling and retries
  \item Exceeding concurrent execution limit (default 1,000)
  \item Not using Lambda Layers for shared dependencies
  \item Embedding sensitive data in code (use environment variables/Secrets Manager)
\end{enumerate}


\textbf{Service Limits and Quotas}:
\begin{itemize}
  \item Concurrent executions: 1,000 (soft limit, can be increased)
  \item Function timeout: 15 minutes (hard limit)
  \item Deployment package size: 50 MB (zipped), 250 MB (unzipped)
  \item /tmp directory storage: 10 GB
  \item Environment variables: 4 KB total
  \item Memory allocation: 128 MB to 10,240 MB (64 MB increments)
  \item Ephemeral storage: 512 MB to 10,240 MB
\end{itemize}


\textbf{Monitoring and Troubleshooting}:

\begin{enumerate}
  \item \textbf{CloudWatch Metrics} (automatic):
\end{enumerate}

\begin{itemize}
  \item Invocations
  \item Duration
  \item Errors
  \item Throttles
  \item Concurrent Executions
\end{itemize}


\begin{enumerate}
  \item \textbf{CloudWatch Logs}:
\end{enumerate}

\begin{itemize}
  \item All console.log/print statements
  \item Execution start/end
  \item Error traces
  \item Custom metrics
\end{itemize}


\begin{enumerate}
  \item \textbf{AWS X-Ray} (distributed tracing):
\end{enumerate}

\begin{itemize}
  \item Trace requests through entire application
  \item Identify performance bottlenecks
  \item Visualize service map
\end{itemize}


\begin{enumerate}
  \item \textbf{Common Issues}:
\end{enumerate}

\begin{itemize}
  \item Cold starts: Use provisioned concurrency (extra cost)
  \item Timeout errors: Increase timeout or optimize code
  \item Out of memory: Increase memory allocation
  \item Throttling: Request concurrent execution limit increase
\end{itemize}


\textbf{Performance Optimization Tips}:
\begin{enumerate}
  \item Minimize deployment package size
  \item Use Lambda Layers for shared dependencies
  \item Keep functions warm with scheduled invocations (if needed)
  \item Optimize memory allocation (more memory = more CPU)
  \item Use environment variables for configuration
  \item Connection pooling for database connections
  \item Lazy load dependencies outside handler function
\end{enumerate}


\subsubsection{Compute Service Selection Flowchart}


\begin{verbatim}
START: Which compute service should I use?
│
├─> Need full control over OS and applications?
│   └─> YES: Amazon EC2
│       ├─> Simple setup needed? → Lightsail
│       ├─> Application deployment focus? → Elastic Beanstalk
│       └─> Full control? → EC2
│
├─> Using containers?
│   └─> YES:
│       ├─> Already use Kubernetes? → EKS
│       ├─> Want AWS-native? → ECS
│       ├─> Want serverless containers? → Fargate
│       └─> Batch processing? → AWS Batch
│
└─> Event-driven, short-running tasks?
    └─> YES: AWS Lambda
        ├─> Workflows needed? → Step Functions + Lambda
        ├─> APIs? → API Gateway + Lambda
        └─> Event processing? → EventBridge + Lambda
\end{verbatim}

\textbf{Migration Scenario: Monolith to Serverless}

\textbf{Current State}: Monolithic PHP application on 3 EC2 instances
\begin{itemize}
  \item Monthly cost: \$150 (EC2) + \$50 (RDS) = \$200
  \item Maintenance: 10 hours/month
  \item Scaling: Manual, 30-minute deployment
\end{itemize}


\textbf{Target State}: Serverless architecture
\begin{itemize}
  \item Static content: S3 + CloudFront
  \item APIs: API Gateway + Lambda
  \item Database: Aurora Serverless
  \item Authentication: Cognito
\end{itemize}


\textbf{Migration Steps}:
\begin{enumerate}
  \item Extract static assets to S3 (Week 1)
  \item Create CloudFront distribution (Week 1)
  \item Migrate APIs to Lambda (Weeks 2-4)
  \item Migrate database to Aurora Serverless (Week 5)
  \item Implement Cognito authentication (Week 6)
  \item Decommission EC2 instances (Week 7)
\end{enumerate}


\textbf{After Migration}:
\begin{itemize}
  \item Monthly cost: \$60 (80\% traffic reduction)
  \item Maintenance: 2 hours/month (75\% reduction)
  \item Scaling: Automatic, instant
  \item Deployment: Minutes (CI/CD pipeline)
  \item Performance: 40\% faster (CloudFront CDN)
\end{itemize}


\subsubsection{Amazon Lightsail}


\textbf{Simplified cloud platform for simple workloads}

\begin{itemize}
  \item Easy-to-use virtual private servers (VPS)
  \item Predictable monthly pricing
  \item Bundled resources: compute, storage, networking, DNS
  \item Pre-configured application stacks (WordPress, Magento, LAMP, etc.)
  \item Ideal for:
  \item Simple web applications
  \item Blogs and websites
  \item Small business applications
  \item Development and test environments
  \item Perfect for users new to AWS or with simple requirements
\end{itemize}


\subsubsection{AWS Elastic Beanstalk}


\textbf{Platform as a Service (PaaS)}

Deploy and manage applications without infrastructure complexity.

\textbf{Key Features}:
\begin{itemize}
  \item \textbf{Language Support}: Java, .NET, PHP, Node.js, Python, Ruby, Go, Docker
  \item \textbf{Automatic Management}: Capacity provisioning, load balancing, auto-scaling, health monitoring
  \item \textbf{Developer Control}: Retain full control over underlying AWS resources
  \item \textbf{No Additional Charge}: Pay only for the AWS resources used
  \item \textbf{Quick Deployment}: Deploy applications in minutes
\end{itemize}


\textbf{Best For}:
\begin{itemize}
  \item Web applications
  \item Developers who want to focus on code, not infrastructure
  \item Standard application architectures
\end{itemize}


\subsubsection{Amazon ECS (Elastic Container Service)}


\textbf{Fully managed container orchestration}

Run and scale Docker containers on AWS.

\textbf{Launch Types}:
\begin{itemize}
  \item \textbf{EC2 Launch Type}:
  \item You manage the underlying EC2 instances
  \item More control over infrastructure
  \item Good for cost optimization with Reserved Instances
  \item \textbf{Fargate Launch Type}:
  \item Serverless container deployment
  \item AWS manages the infrastructure
  \item Pay only for container resources
\end{itemize}


\textbf{Features}:
\begin{itemize}
  \item Deep AWS integration
  \item Service discovery
  \item Load balancing
  \item Auto Scaling
\end{itemize}


\textbf{Use Cases}:
\begin{itemize}
  \item Microservices architectures
  \item Batch processing
  \item Machine learning applications
\end{itemize}


\subsubsection{Amazon EKS (Elastic Kubernetes Service)}


\textbf{Managed Kubernetes service}

\begin{itemize}
  \item Fully managed Kubernetes control plane
  \item Compatible with standard Kubernetes tooling and plugins
  \item Automatic Kubernetes version upgrades and patching
  \item Integrates with AWS services (IAM, VPC, CloudWatch)
  \item Multi-AZ control plane for high availability
  \item Best for teams already invested in Kubernetes ecosystem
\end{itemize}


\subsubsection{AWS Fargate}


\textbf{Serverless compute engine for containers}

\begin{itemize}
  \item Works with both ECS and EKS
  \item No need to provision, configure, or scale EC2 instances
  \item Pay only for the vCPU and memory resources your containers use
  \item Automatic scaling and load balancing
  \item Focus on building applications, not managing infrastructure
\end{itemize}


\begin{keypoint}
\textbf{Key Point}: Compute spectrum from most to least control:
\textbf{EC2} (full control) → \textbf{Containers (ECS/EKS)} (moderate control) → \textbf{Lambda/Fargate} (least control, most abstraction)
\end{keypoint}


---

\subsection{Storage Services}


\subsubsection{Amazon S3 (Simple Storage Service)}


\textbf{Object storage service for any amount of data}

S3 is a highly durable, scalable object storage service designed to store and retrieve any amount of data from anywhere.

\paragraph{Key Concepts}


\begin{longtable}{ll}
\toprule
\textbf{Concept} & \textbf{Description} \\
\midrule
\textbf{Buckets} & Containers for objects with globally unique names \\
\textbf{Objects} & Files stored in buckets (up to 5 TB each) \\
\textbf{Keys} & Unique identifier for each object in a bucket \\
\textbf{Durability} & 99.999999999\% (11 nines) \\
\textbf{Availability} & Varies by storage class \\
\bottomrule
\end{longtable}

\paragraph{S3 Storage Classes}


\#\#\#\#\# 1. S3 Standard

\begin{itemize}
  \item \textbf{Use Case}: Frequently accessed data
  \item \textbf{Performance}: Low latency and high throughput
  \item \textbf{Availability}: 99.99\%
  \item \textbf{Cost}: Most expensive storage cost
  \item \textbf{Best For}: Active databases, frequently accessed content, dynamic websites
\end{itemize}


\#\#\#\#\# 2. S3 Intelligent-Tiering

\begin{itemize}
  \item \textbf{Automation}: Automatically moves objects between access tiers
  \item \textbf{Optimization}: Cost optimization for unknown or changing access patterns
  \item \textbf{Tiers}: Frequent Access, Infrequent Access, Archive Instant Access, Archive Access, Deep Archive Access
  \item \textbf{Monitoring Fee}: Small monthly fee per object
  \item \textbf{No Retrieval Fees}: Between Frequent and Infrequent Access tiers
\end{itemize}


\#\#\#\#\# 3. S3 Standard-IA (Infrequent Access)

\begin{itemize}
  \item \textbf{Use Case}: Infrequently accessed data that needs rapid access when required
  \item \textbf{Cost}: Lower storage cost, but retrieval fee applies
  \item \textbf{Availability}: 99.9\%
  \item \textbf{Minimum Duration}: 30-day minimum storage charge
  \item \textbf{Best For}: Backups, disaster recovery, long-term storage
\end{itemize}


\#\#\#\#\# 4. S3 One Zone-IA

\begin{itemize}
  \item \textbf{Storage}: Single Availability Zone (not multiple AZs)
  \item \textbf{Cost}: 20\% less than Standard-IA
  \item \textbf{Availability}: 99.5\%
  \item \textbf{Risk}: Data lost if AZ is destroyed
  \item \textbf{Best For}: Recreatable data, secondary backup copies
\end{itemize}


\#\#\#\#\# 5. S3 Glacier Instant Retrieval

\begin{itemize}
  \item \textbf{Use Case}: Archive data requiring instant access
  \item \textbf{Retrieval}: Millisecond retrieval times
  \item \textbf{Cost}: Lower storage cost than Standard-IA
  \item \textbf{Minimum Duration}: 90-day minimum storage charge
  \item \textbf{Best For}: Medical images, news media assets accessed once per quarter
\end{itemize}


\#\#\#\#\# 6. S3 Glacier Flexible Retrieval

\begin{itemize}
  \item \textbf{Use Case}: Archive data with retrieval times from minutes to hours
  \item \textbf{Retrieval Options}:
  \item \textbf{Expedited}: 1-5 minutes
  \item \textbf{Standard}: 3-5 hours
  \item \textbf{Bulk}: 5-12 hours (lowest cost)
  \item \textbf{Minimum Duration}: 90-day minimum storage charge
  \item \textbf{Best For}: Backup and archive data accessed 1-2 times per year
\end{itemize}


\#\#\#\#\# 7. S3 Glacier Deep Archive

\begin{itemize}
  \item \textbf{Use Case}: Long-term archive and digital preservation
  \item \textbf{Cost}: Lowest cost storage class
  \item \textbf{Retrieval Time}: 12-48 hours
  \item \textbf{Minimum Duration}: 180-day minimum storage charge
  \item \textbf{Best For}: Compliance archives, digital preservation, data retained for 7-10+ years
\end{itemize}


\paragraph{S3 Features}


\textbf{Versioning}
\begin{itemize}
  \item Keep multiple versions of an object
  \item Protect against accidental deletion
  \item Can be enabled/suspended per bucket
\end{itemize}


\textbf{Lifecycle Policies}
\begin{itemize}
  \item Automatically transition objects between storage classes
  \item Automatically delete objects after a specified time
  \item Cost optimization through automation
\end{itemize}


\textbf{Encryption}
\begin{itemize}
  \item \textbf{Server-Side Encryption (SSE)}: S3 encrypts objects
  \item SSE-S3: S3-managed keys
  \item SSE-KMS: AWS KMS-managed keys
  \item SSE-C: Customer-provided keys
  \item \textbf{Client-Side Encryption}: Encrypt before uploading
\end{itemize}


\textbf{Access Control}
\begin{itemize}
  \item Bucket policies (resource-based)
  \item IAM policies (identity-based)
  \item Access Control Lists (ACLs) - legacy
  \item S3 Block Public Access settings
\end{itemize}


\textbf{Additional Features}
\begin{itemize}
  \item Static website hosting
  \item Cross-Region Replication (CRR)
  \item Same-Region Replication (SRR)
  \item Transfer Acceleration (via CloudFront edge locations)
  \item Event notifications
  \item S3 Select (query data using SQL)
\end{itemize}


\textbf{S3 Storage Class Cost Comparison}

\begin{longtable}{llllll}
\toprule
\textbf{Storage Class} & \textbf{Storage Cost (per GB/month)} & \textbf{Retrieval Cost} & \textbf{Minimum Duration} & \textbf{Retrieval Time} & \textbf{Best Use Case} \\
\midrule
\textbf{S3 Standard} & \$0.023 & None & None & Milliseconds & Frequently accessed data \\
\textbf{S3 Intelligent-Tiering} & \$0.023-\$0.0125 + \$0.0025 monitoring & None (auto-tier) & None & Milliseconds & Unknown access patterns \\
\textbf{S3 Standard-IA} & \$0.0125 & \$0.01 per GB & 30 days & Milliseconds & Infrequent access \\
\textbf{S3 One Zone-IA} & \$0.01 & \$0.01 per GB & 30 days & Milliseconds & Recreatable data \\
\textbf{S3 Glacier Instant Retrieval} & \$0.004 & \$0.03 per GB & 90 days & Milliseconds & Archive with instant access \\
\textbf{S3 Glacier Flexible Retrieval} & \$0.0036 & \$0.01-\$0.03 per GB & 90 days & Minutes-hours & Archive, 1-2 times/year \\
\textbf{S3 Glacier Deep Archive} & \$0.00099 & \$0.02 per GB & 180 days & 12-48 hours & Long-term compliance \\
\bottomrule
\end{longtable}

\textbf{Cost Example: 100 TB of Data Storage}

\begin{longtable}{llllll}
\toprule
\textbf{Scenario} & \textbf{Storage Class} & \textbf{Monthly Storage} & \textbf{Retrieval (10\% monthly)} & \textbf{Total Monthly Cost} & \textbf{Annual Cost} \\
\midrule
\textbf{Active Website} & S3 Standard & \$2,355 & \$0 & \$2,355 & \$28,260 \\
\textbf{Backup (weekly access)} & S3 Standard-IA & \$1,280 & \$102 & \$1,382 & \$16,584 \\
\textbf{Compliance Archive} & Glacier Deep Archive & \$102 & \$205 & \$307 & \$3,684 \\
\textbf{Unknown Pattern} & Intelligent-Tiering & \$850-\$2,355 & \$0 & \textasciitilde{}\$1,500 & \$18,000 \\
\bottomrule
\end{longtable}

\textbf{Real-World Use Case: Media Company}

\textbf{Scenario}: Video streaming platform with 500 TB of content.

\textbf{Storage Strategy}:
\begin{enumerate}
  \item \textbf{Recent content} (20 TB, last 30 days): S3 Standard
\end{enumerate}

\begin{itemize}
  \item High access rate, low latency needed
  \item Cost: 20,000 GB × \$0.023 = \$460/month
\end{itemize}


\begin{enumerate}
  \item \textbf{Popular library} (100 TB, accessed weekly): S3 Intelligent-Tiering
\end{enumerate}

\begin{itemize}
  \item Access patterns vary by content popularity
  \item Cost: \textasciitilde{}\$1,700/month (auto-optimizes)
\end{itemize}


\begin{enumerate}
  \item \textbf{Archive content} (300 TB, accessed rarely): Glacier Flexible Retrieval
\end{enumerate}

\begin{itemize}
  \item Old shows, accessed 1-2 times per year
  \item Cost: 300,000 GB × \$0.0036 = \$1,080/month
\end{itemize}


\begin{enumerate}
  \item \textbf{Legal/compliance} (80 TB, 7-year retention): Glacier Deep Archive
\end{enumerate}

\begin{itemize}
  \item Almost never accessed
  \item Cost: 80,000 GB × \$0.00099 = \$79/month
\end{itemize}


\textbf{Total}: \$3,319/month vs \$11,500/month if everything in S3 Standard (71\% savings)

\textbf{S3 Performance Benchmarks}

\begin{longtable}{lll}
\toprule
\textbf{Metric} & \textbf{Performance} & \textbf{Notes} \\
\midrule
\textbf{Request Rate} & 3,500 PUT/COPY/POST/DELETE per second per prefix & Can scale by using multiple prefixes \\
\textbf{Request Rate} & 5,500 GET/HEAD per second per prefix & Automatic scaling \\
\textbf{Throughput} & No limit & Scales automatically \\
\textbf{Latency} & 100-200ms (first byte) & Standard, IA, One Zone-IA \\
\textbf{Durability} & 99.999999999\% (11 nines) & Designed to sustain loss of 2 facilities \\
\textbf{Availability SLA} & 99.9-99.99\% & Varies by storage class \\
\bottomrule
\end{longtable}

\textbf{Common Configuration Mistakes}:
\begin{enumerate}
  \item Using S3 Standard for infrequently accessed data (overpaying)
  \item Not enabling versioning for critical data (risk of data loss)
  \item Not using lifecycle policies (manual tier management)
  \item Allowing public access unintentionally (security risk)
  \item Not enabling server-side encryption (compliance issues)
  \item Not using S3 Transfer Acceleration for global uploads
  \item Storing small objects inefficiently (minimum billable size)
  \item Not implementing proper backup/replication strategy
\end{enumerate}


\textbf{Service Limits and Quotas}:
\begin{itemize}
  \item Buckets per account: 100 (soft limit, can be increased to 1,000)
  \item Object size: 5 TB maximum
  \item Single PUT: 5 GB maximum (use multipart for larger)
  \item Bucket name: 3-63 characters, globally unique
  \item Bucket policy: 20 KB maximum
  \item Tags per object: 10
\end{itemize}


\textbf{Integration Example: Automated Data Pipeline}

\begin{verbatim}
Architecture:
On-premises data
  └─> AWS DataSync / S3 Transfer
      └─> S3 (Standard)
          ├─> Lambda (process new files)
          │   └─> DynamoDB (metadata)
          │
          ├─> Lifecycle Policy (30 days)
          │   └─> S3 Intelligent-Tiering
          │
          ├─> Lifecycle Policy (90 days)
          │   └─> Glacier Flexible Retrieval
          │
          └─> Lifecycle Policy (365 days)
              └─> Glacier Deep Archive
\end{verbatim}

\textbf{Monitoring and Troubleshooting}:

\begin{enumerate}
  \item \textbf{CloudWatch Metrics}:
\end{enumerate}

\begin{itemize}
  \item NumberOfObjects
  \item BucketSizeBytes
  \item AllRequests
  \item 4xxErrors / 5xxErrors
  \item FirstByteLatency
\end{itemize}


\begin{enumerate}
  \item \textbf{S3 Storage Lens} (analytics):
\end{enumerate}

\begin{itemize}
  \item Usage and activity metrics
  \item Cost optimization recommendations
  \item Data protection insights
  \item Access patterns
\end{itemize}


\begin{enumerate}
  \item \textbf{S3 Access Logging}:
\end{enumerate}

\begin{itemize}
  \item Track all requests to bucket
  \item Security and access audits
  \item Usage analysis
\end{itemize}


\begin{enumerate}
  \item \textbf{Common Issues}:
\end{enumerate}

\begin{itemize}
  \item Slow upload: Use S3 Transfer Acceleration or multipart upload
  \item 403 Forbidden: Check bucket policy and IAM permissions
  \item 503 SlowDown: Reduce request rate or use prefixes
  \item High costs: Implement lifecycle policies and use appropriate storage class
\end{itemize}


\textbf{Best Practices}:
\begin{enumerate}
  \item \textbf{Security}:
\end{enumerate}

\begin{itemize}
  \item Enable S3 Block Public Access
  \item Use bucket policies and IAM roles
  \item Enable versioning for critical data
  \item Enable server-side encryption
  \item Use S3 Object Lock for compliance
\end{itemize}


\begin{enumerate}
  \item \textbf{Cost Optimization}:
\end{enumerate}

\begin{itemize}
  \item Use S3 Intelligent-Tiering for unknown patterns
  \item Implement lifecycle policies
  \item Delete incomplete multipart uploads
  \item Use S3 Storage Lens for optimization insights
  \item Consider Requester Pays for public datasets
\end{itemize}


\begin{enumerate}
  \item \textbf{Performance}:
\end{enumerate}

\begin{itemize}
  \item Use prefixes to distribute load
  \item Enable S3 Transfer Acceleration for global users
  \item Use CloudFront for frequently accessed content
  \item Implement multipart upload for large files
  \item Use byte-range fetches for large downloads
\end{itemize}


\begin{enumerate}
  \item \textbf{Data Protection}:
\end{enumerate}

\begin{itemize}
  \item Enable versioning
  \item Use Cross-Region Replication for DR
  \item Implement MFA Delete for critical buckets
  \item Regular backup testing
  \item Use S3 Object Lock for WORM requirements
\end{itemize}


\subsubsection{Storage Service Comparison Table}


\begin{longtable}{llllll}
\toprule
\textbf{Service} & \textbf{Type} & \textbf{Use Case} & \textbf{Shared Access} & \textbf{Pricing Model} & \textbf{Typical Latency} \\
\midrule
\textbf{S3} & Object & Backups, web content, data lakes & Via API/URL & Per GB stored + requests & 100-200ms \\
\textbf{EBS} & Block & EC2 instance storage & Single instance & Per GB provisioned & <10ms \\
\textbf{EFS} & File & Shared file storage & Multiple instances & Per GB used & Low ms \\
\textbf{FSx for Windows} & File & Windows file shares & Multiple instances & Per GB provisioned & Sub-ms \\
\textbf{FSx for Lustre} & File & HPC, ML training & Multiple instances & Per GB provisioned & Sub-ms \\
\textbf{Storage Gateway} & Hybrid & On-premises to cloud & On-premises + cloud & Per GB stored & Varies \\
\textbf{Snow Family} & Physical & Offline data transfer & Physical device & Per device + data transfer & N/A \\
\bottomrule
\end{longtable}

\subsubsection{Amazon EBS (Elastic Block Store)}


\textbf{Block-level storage volumes for EC2 instances}

Persistent block storage that can be attached to EC2 instances, similar to a hard drive.

\textbf{Key Characteristics}:
\begin{itemize}
  \item Persistent storage that exists independently of instance lifetime
  \item Attached to a single EC2 instance at a time
  \item Automatically replicated within its Availability Zone
  \item Snapshots stored in Amazon S3
  \item Can detach and reattach to different instances
  \item Supports encryption at rest and in transit
\end{itemize}


\paragraph{EBS Volume Types}


\begin{longtable}{llll}
\toprule
\textbf{Type} & \textbf{Category} & \textbf{Use Case} & \textbf{Performance} \\
\midrule
\textbf{gp3} & General Purpose SSD & Balanced price/performance & 3,000-16,000 IOPS \\
\textbf{gp2} & General Purpose SSD & Balanced price/performance & Baseline 3 IOPS/GB \\
\textbf{io2/io1} & Provisioned IOPS SSD & Mission-critical, high-performance & Up to 64,000 IOPS \\
\textbf{st1} & Throughput Optimized HDD & Frequently accessed, throughput-intensive & Good for big data, data warehouses \\
\textbf{sc1} & Cold HDD & Infrequently accessed & Lowest cost, file servers \\
\bottomrule
\end{longtable}

\textbf{Best Practices}:
\begin{itemize}
  \item Take regular snapshots for backup
  \item Use Provisioned IOPS for database workloads
  \item General Purpose SSD for most use cases
\end{itemize}


\subsubsection{Amazon EFS (Elastic File System)}


\textbf{Managed NFS file system for EC2}

\begin{itemize}
  \item \textbf{File System Type}: Network File System (NFS) v4.1
  \item \textbf{Shared Access}: Multiple EC2 instances can access simultaneously
  \item \textbf{Automatic Scaling}: Grows and shrinks automatically as you add/remove files
  \item \textbf{Pricing}: Pay only for storage used (no pre-provisioning)
  \item \textbf{Availability}: Regional service spanning multiple Availability Zones
  \item \textbf{Performance Modes}:
  \item \textbf{General Purpose}: Low latency (web serving, CMS)
  \item \textbf{Max I/O}: Higher latency, massively parallel (big data, media processing)
\end{itemize}


\textbf{Use Cases}:
\begin{itemize}
  \item Content management systems
  \item Web serving
  \item Data sharing and collaboration
  \item Development environments
  \item Container storage
\end{itemize}


\textbf{Comparison}:
\begin{itemize}
  \item \textbf{EBS}: Single instance, must be in same AZ
  \item \textbf{EFS}: Multiple instances, cross-AZ access
  \item \textbf{S3}: Object storage, unlimited instances via API
\end{itemize}


\subsubsection{AWS Storage Gateway}


\textbf{Hybrid cloud storage service}

Connects on-premises environments to AWS cloud storage.

\paragraph{Gateway Types}


\textbf{1. File Gateway}
\begin{itemize}
  \item Presents NFS/SMB file shares
  \item Files stored as objects in S3
  \item Local cache for frequently accessed data
  \item Use Case: Cloud-backed file shares
\end{itemize}


\textbf{2. Volume Gateway}
\begin{itemize}
  \item Presents iSCSI block storage volumes
  \item Two modes:
  \item \textbf{Cached Volumes}: Primary data in S3, cache on-premises
  \item \textbf{Stored Volumes}: Primary data on-premises, async backup to S3
  \item Use Case: Block storage backup and disaster recovery
\end{itemize}


\textbf{3. Tape Gateway}
\begin{itemize}
  \item Virtual tape library (VTL)
  \item Integrates with existing backup software
  \item Store virtual tapes in S3 and Glacier
  \item Use Case: Replace physical tape backup infrastructure
\end{itemize}


\textbf{Benefits}:
\begin{itemize}
  \item Seamless cloud integration
  \item Local caching for low latency
  \item Cost-effective storage
  \item Disaster recovery
\end{itemize}


\subsubsection{AWS Snow Family}


\textbf{Physical devices for data migration and edge computing}

Move large amounts of data into and out of AWS using secure physical devices.

\paragraph{Devices}


\begin{longtable}{lll}
\toprule
\textbf{Device} & \textbf{Storage Capacity} & \textbf{Use Case} \\
\midrule
\textbf{Snowcone} & 8 TB usable & Smallest, portable, edge computing \\
\textbf{Snowball Edge Storage Optimized} & 80 TB & Large data migrations \\
\textbf{Snowball Edge Compute Optimized} & 42 TB + compute & Edge computing + storage \\
\textbf{Snowmobile} & 100 PB & Massive data center migration \\
\bottomrule
\end{longtable}

\textbf{Snowball Edge Features}:
\begin{itemize}
  \item Can run EC2 instances and Lambda functions
  \item Cluster multiple devices together
  \item Storage and compute in disconnected environments
\end{itemize}


\textbf{Process}:
\begin{enumerate}
  \item Order device from AWS
  \item AWS ships device to your location
  \item Connect device and copy data
  \item Ship device back to AWS
  \item AWS loads data into your S3 bucket
\end{enumerate}


\textbf{When to Use}:
\begin{itemize}
  \item \textbf{Snowcone/Snowball}: Terabytes to petabytes of data
  \item \textbf{Snowmobile}: 10 PB or more
  \item Limited bandwidth or expensive network transfer
  \item Physical data migration requirements
\end{itemize}


\begin{examtip}
\textbf{Exam Tip}: Choose storage based on use case:
- \textbf{S3}: Object storage, web content, backups
- \textbf{EBS}: EC2 instance block storage, databases
- \textbf{EFS}: Shared file system across multiple instances
- \textbf{Storage Gateway}: Hybrid cloud storage
- \textbf{Snow Family}: Large-scale data migration
\end{examtip}


---

\subsection{Database Services}


\subsubsection{Amazon RDS (Relational Database Service)}


\textbf{Managed relational database service}

RDS makes it easy to set up, operate, and scale relational databases in the cloud.

\textbf{Supported Database Engines}:
\begin{itemize}
  \item MySQL
  \item PostgreSQL
  \item MariaDB
  \item Oracle Database
  \item Microsoft SQL Server
  \item Amazon Aurora
\end{itemize}


\textbf{Key Features}:
\begin{itemize}
  \item \textbf{Automated Management}: Backups, patching, scaling
  \item \textbf{No OS Access}: Fully managed (no SSH access)
  \item \textbf{High Availability}: Multi-AZ deployments
  \item \textbf{Scalability}: Read replicas for read-heavy workloads
  \item \textbf{Security}: Encryption at rest and in transit
  \item \textbf{Monitoring}: CloudWatch integration
\end{itemize}


\textbf{Pricing}: Pay for instance type, storage, backups, and data transfer

\paragraph{Multi-AZ Deployments}


\textbf{High availability and disaster recovery}

\begin{itemize}
  \item \textbf{Replication}: Synchronous replication to standby instance in different AZ
  \item \textbf{Automatic Failover}: Automatic failover to standby if primary fails
  \item \textbf{Purpose}: High availability and disaster recovery (not performance)
  \item \textbf{Downtime}: Minimal during failover (typically under 1 minute)
  \item \textbf{Backup}: Backups taken from standby to avoid I/O impact
\end{itemize}


\textbf{When to Use}: Production workloads requiring high availability

\paragraph{Read Replicas}


\textbf{Scale read-heavy workloads}

\begin{itemize}
  \item \textbf{Replication}: Asynchronous replication from primary
  \item \textbf{Purpose}: Improve read performance, not high availability
  \item \textbf{Location}: Can be in same AZ, different AZ, or different Region
  \item \textbf{Promotion}: Can be promoted to standalone database
  \item \textbf{Limit}: Up to 5 read replicas per primary database
  \item \textbf{Use Cases}: Reporting, analytics, read-heavy applications
\end{itemize}


\textbf{Comparison}:
\begin{itemize}
  \item \textbf{Multi-AZ}: Synchronous, same Region, automatic failover, HA/DR
  \item \textbf{Read Replicas}: Asynchronous, can be cross-Region, read scaling
\end{itemize}


\subsubsection{Amazon Aurora}


\textbf{AWS cloud-native relational database}

\begin{itemize}
  \item \textbf{Compatibility}: MySQL and PostgreSQL compatible
  \item \textbf{Performance}:
  \item 5x faster than standard MySQL
  \item 3x faster than standard PostgreSQL
  \item \textbf{Read Replicas}: Up to 15 Aurora read replicas
  \item \textbf{Storage}: Automatically scales from 10 GB to 128 TB
  \item \textbf{Durability}: 6 copies of data across 3 Availability Zones
  \item \textbf{Availability}: Self-healing storage, automated failover
  \item \textbf{Cost}: More expensive than RDS, but better performance and features
\end{itemize}


\textbf{Aurora Serverless}:
\begin{itemize}
  \item On-demand, auto-scaling configuration
  \item Automatically starts up, scales, and shuts down
  \item Pay per second for resources consumed
  \item Ideal for infrequent, intermittent, or unpredictable workloads
\end{itemize}


\textbf{When to Choose Aurora}:
\begin{itemize}
  \item Need better performance than standard RDS
  \item High availability requirements
  \item Rapid scaling needs
  \item MySQL or PostgreSQL compatibility required
\end{itemize}


\subsubsection{Amazon DynamoDB}


\textbf{Fully managed NoSQL database service}

Fast and flexible NoSQL database for any scale.

\textbf{Key Characteristics}:
\begin{itemize}
  \item \textbf{Type}: Key-value and document database
  \item \textbf{Performance}: Single-digit millisecond latency at any scale
  \item \textbf{Serverless}: Fully managed, no servers to provision
  \item \textbf{Scaling}: Automatic scaling of throughput and storage
  \item \textbf{Availability}: Multi-AZ replication, highly available
\end{itemize}


\textbf{Features}:
\begin{itemize}
  \item \textbf{Global Tables}: Multi-region, multi-active replication
  \item \textbf{DynamoDB Streams}: Capture item-level changes
  \item \textbf{DAX (DynamoDB Accelerator)}: In-memory cache for microsecond latency
  \item \textbf{Transactions}: ACID transactions support
  \item \textbf{Backup}: Point-in-time recovery, on-demand backups
\end{itemize}


\textbf{Pricing Models}:
\begin{itemize}
  \item \textbf{On-Demand}: Pay per request (unpredictable workloads)
  \item \textbf{Provisioned Capacity}: Reserve read/write capacity units (predictable workloads)
\end{itemize}


\textbf{Use Cases}:
\begin{itemize}
  \item Mobile and web applications
  \item Gaming applications
  \item IoT data storage
  \item Session management
  \item Real-time bidding
\end{itemize}


\subsubsection{Amazon ElastiCache}


\textbf{In-memory caching service}

Managed Redis or Memcached for improved application performance.

\textbf{Supported Engines}:
\begin{itemize}
  \item \textbf{Redis}: Advanced data structures, persistence, replication, pub/sub
  \item \textbf{Memcached}: Simple caching, multi-threading
\end{itemize}


\textbf{Benefits}:
\begin{itemize}
  \item \textbf{Performance}: Microsecond latency
  \item \textbf{Reduced Database Load}: Cache frequently accessed data
  \item \textbf{Scalability}: Easily scale in/out
  \item \textbf{Availability}: Multi-AZ with automatic failover (Redis)
\end{itemize}


\textbf{Common Use Cases}:
\begin{itemize}
  \item Database query result caching
  \item Session stores for web applications
  \item Real-time analytics
  \item Leaderboards and counting
  \item Message queues (Redis)
\end{itemize}


\textbf{When to Use}:
\begin{itemize}
  \item Improve read performance
  \item Reduce database costs
  \item Store session data
  \item Real-time applications
\end{itemize}


\subsubsection{Amazon Redshift}


\textbf{Fully managed data warehouse}

Fast, scalable data warehouse for analytics.

\textbf{Key Features}:
\begin{itemize}
  \item \textbf{Scale}: Petabyte-scale data warehouse
  \item \textbf{Storage}: Columnar storage format
  \item \textbf{Processing}: Massively parallel processing (MPP)
  \item \textbf{Query Language}: Standard SQL
  \item \textbf{Integration}: Integrates with BI tools (Tableau, Power BI, QuickSight)
  \item \textbf{Performance}: 10x better performance than traditional data warehouses
\end{itemize}


\textbf{Use Cases}:
\begin{itemize}
  \item Business intelligence and analytics
  \item Historical data analysis
  \item Complex queries on large datasets
  \item Data consolidation from multiple sources
\end{itemize}


\textbf{Redshift Spectrum}:
\begin{itemize}
  \item Query data directly in S3 without loading
  \item Extend queries beyond Redshift data warehouse
\end{itemize}


\subsubsection{Amazon Neptune}


\textbf{Fully managed graph database}

\begin{itemize}
  \item \textbf{Type}: Graph database service
  \item \textbf{Models}: Supports Property Graph and RDF graph models
  \item \textbf{Query Languages}: Gremlin and SPARQL
  \item \textbf{Performance}: Fast query execution on billions of relationships
  \item \textbf{Availability}: Multi-AZ, read replicas
\end{itemize}


\textbf{Use Cases}:
\begin{itemize}
  \item Social networking applications
  \item Recommendation engines
  \item Fraud detection
  \item Knowledge graphs
  \item Network and IT operations
\end{itemize}


\subsubsection{Amazon DocumentDB}


\textbf{MongoDB-compatible document database}

\begin{itemize}
  \item \textbf{Compatibility}: MongoDB-compatible (supports MongoDB APIs)
  \item \textbf{Management}: Fully managed by AWS
  \item \textbf{Scaling}: Scale storage and compute independently
  \item \textbf{Availability}: Multi-AZ, automated backups
  \item \textbf{Performance}: 2x throughput improvement over MongoDB
\end{itemize}


\textbf{Use Cases}:
\begin{itemize}
  \item Content management systems
  \item Product catalogs
  \item User profiles
  \item Mobile and web app backends
\end{itemize}


\begin{keypoint}
\textbf{Key Point}: Database selection guide:
- \textbf{RDS/Aurora}: Relational databases, ACID transactions, SQL
- \textbf{DynamoDB}: NoSQL key-value, high scale, low latency
- \textbf{Redshift}: Data warehouse, analytics, BI
- \textbf{ElastiCache}: In-memory caching, session storage
- \textbf{Neptune}: Graph databases, relationships
- \textbf{DocumentDB}: Document database, MongoDB workloads
\end{keypoint}


\subsubsection{Database Service Detailed Comparison}


\begin{longtable}{lllllll}
\toprule
\textbf{Database} & \textbf{Type} & \textbf{Engine Options} & \textbf{Scaling} & \textbf{High Availability} & \textbf{Best For} & \textbf{Pricing Model} \\
\midrule
\textbf{RDS} & Relational & MySQL, PostgreSQL, MariaDB, Oracle, SQL Server, Aurora & Vertical (instance resize) & Multi-AZ & Traditional RDBMS & Instance + storage \\
\textbf{Aurora} & Relational & MySQL, PostgreSQL & Auto storage + read replicas & Built-in Multi-AZ & High-performance RDBMS & Instance + I/O \\
\textbf{DynamoDB} & NoSQL Key-Value & N/A & Automatic horizontal & Built-in Multi-AZ & Massive scale, low latency & On-Demand or provisioned \\
\textbf{ElastiCache} & In-Memory & Redis, Memcached & Horizontal (add nodes) & Multi-AZ (Redis) & Caching, session store & Node hours \\
\textbf{Redshift} & Data Warehouse & PostgreSQL-compatible & Add nodes to cluster & Single-AZ (snapshots for backup) & Analytics, BI & Node hours \\
\textbf{Neptune} & Graph & Gremlin, SPARQL & Vertical + read replicas & Multi-AZ & Relationship queries & Instance + I/O \\
\textbf{DocumentDB} & Document & MongoDB-compatible & Vertical + read replicas & Multi-AZ & Document store & Instance + storage \\
\textbf{Timestream} & Time Series & N/A & Automatic & Multi-AZ & IoT, metrics & Storage + queries \\
\bottomrule
\end{longtable}

\subsubsection{Database Selection Flowchart}


\begin{verbatim}
START: Which database service should I use?
│
├─> Need SQL and ACID transactions?
│   └─> YES: Relational Database
│       ├─> Maximum performance needed? → Aurora
│       ├─> Oracle/SQL Server licensing? → RDS (Oracle/SQL Server)
│       ├─> MySQL/PostgreSQL? → RDS or Aurora
│       └─> Analytics/BI workload? → Redshift
│
├─> Need to cache or session storage?
│   └─> YES: ElastiCache
│       ├─> Advanced data structures? → Redis
│       └─> Simple caching? → Memcached
│
├─> Need document database?
│   └─> YES:
│       ├─> MongoDB workloads? → DocumentDB
│       └─> Flexible NoSQL? → DynamoDB
│
├─> Need graph database?
│   └─> YES: Neptune
│
├─> Need NoSQL at massive scale?
│   └─> YES: DynamoDB
│       ├─> Predictable traffic? → Provisioned capacity
│       └─> Variable traffic? → On-Demand capacity
│
└─> Time-series data (IoT, metrics)?
    └─> YES: Timestream
\end{verbatim}

\subsubsection{Real-World Database Use Cases}


\textbf{Use Case 1: E-Commerce Platform}

\textbf{Requirements}:
\begin{itemize}
  \item Product catalog: 1M products
  \item User accounts: 10M users
  \item Orders: 1M transactions/month
  \item Shopping carts: Temporary, high read/write
\end{itemize}


\textbf{Architecture}:
\begin{verbatim}
Frontend (S3 + CloudFront)
  └─> API Gateway + Lambda
      ├─> DynamoDB (Shopping carts, sessions)
      │   - On-Demand capacity
      │   - Single-digit millisecond latency
      │   - Cost: \~{}\$200/month
      │
      ├─> ElastiCache Redis (Product catalog cache)
      │   - cache.t3.medium: \$0.068/hour
      │   - Cost: \~{}\$50/month
      │   - 99\% cache hit rate
      │
      ├─> Aurora MySQL (Orders, inventory)
      │   - db.r5.large: \$0.29/hour
      │   - Multi-AZ for HA
      │   - Cost: \~{}\$430/month
      │
      └─> Redshift (Analytics, business intelligence)
          - dc2.large: \$0.25/hour (2 nodes)
          - Cost: \~{}\$360/month
\end{verbatim}

\textbf{Total Database Cost}: \textasciitilde{}\$1,040/month

\textbf{Performance}:
\begin{itemize}
  \item Cart operations: <10ms (DynamoDB)
  \item Product lookups: <5ms (ElastiCache)
  \item Order processing: <50ms (Aurora)
  \item Analytics queries: seconds to minutes (Redshift)
\end{itemize}


\textbf{Use Case 2: Social Media Application}

\textbf{Requirements}:
\begin{itemize}
  \item User profiles and relationships
  \item Activity feeds
  \item Real-time messaging
  \item Content recommendations
\end{itemize}


\textbf{Architecture}:
\begin{verbatim}
Application Layer
  ├─> Neptune (User relationships, friend graphs)
  │   - db.r5.large: \$0.348/hour
  │   - Cost: \~{}\$254/month
  │   - Query: "Friends of friends" in milliseconds
  │
  ├─> DynamoDB (User profiles, posts, activity feeds)
  │   - Global Tables for multi-region
  │   - Cost: \~{}\$500/month (variable)
  │
  ├─> ElastiCache Redis (Real-time messaging, sessions)
  │   - cache.r5.large cluster
  │   - Cost: \~{}\$200/month
  │
  └─> DocumentDB (Content metadata, analytics)
      - db.r5.large: \$0.29/hour
      - Cost: \~{}\$212/month
\end{verbatim}

\textbf{Total}: \textasciitilde{}\$1,166/month for millions of users

\subsubsection{Database Cost Comparison Examples}


\textbf{Scenario: MySQL Database for Production Application}

\begin{longtable}{llllll}
\toprule
\textbf{Workload} & \textbf{AWS Service} & \textbf{Configuration} & \textbf{Monthly Cost} & \textbf{Management Effort} & \textbf{Scalability} \\
\midrule
\textbf{Small} (< 100 GB) & RDS MySQL & db.t3.medium, 100 GB & \$75 & Low & Vertical \\
\textbf{Small Serverless} & Aurora Serverless v2 & 0.5-1 ACU & \$45-90 & Minimal & Automatic \\
\textbf{Medium} (500 GB) & RDS MySQL Multi-AZ & db.m5.large, 500 GB & \$385 & Low & Vertical \\
\textbf{Medium} & Aurora MySQL & db.r5.large, 500 GB & \$430 & Low & Vertical + Horizontal \\
\textbf{Large} (2 TB) & Aurora MySQL & db.r5.2xlarge + 5 read replicas & \$1,850 & Low & High \\
\textbf{On-premises equivalent} & Self-managed & 3 servers, licenses, staff & \$3,000+ & High & Manual \\
\bottomrule
\end{longtable}

\textbf{Scenario: NoSQL Database - DynamoDB vs Self-Managed}

\begin{longtable}{lll}
\toprule
\textbf{Metric} & \textbf{DynamoDB} & \textbf{Self-Managed Cassandra on EC2} \\
\midrule
\textbf{Setup Time} & Minutes & Days to weeks \\
\textbf{Management} & Fully managed & You manage everything \\
\textbf{Scaling} & Automatic, instant & Manual cluster management \\
\textbf{Performance} & Single-digit milliseconds & Tuning required \\
\textbf{High Availability} & Built-in, multi-AZ & Must configure \\
\textbf{Cost (10GB, 1M reads/writes per day)} & \textasciitilde{}\$25/month & \textasciitilde{}\$150/month (instances + management) \\
\textbf{Cost (1TB, 100M reads/writes per day)} & \textasciitilde{}\$1,250/month & \textasciitilde{}\$1,500/month \\
\textbf{Staff Required} & None & 1-2 DBAs \\
\bottomrule
\end{longtable}

\subsubsection{Database Migration Scenarios}


\textbf{Scenario 1: Oracle to Aurora PostgreSQL}

\textbf{Current State}: On-premises Oracle RAC
\begin{itemize}
  \item Database size: 2 TB
  \item Monthly license cost: \$5,000
  \item Hardware and maintenance: \$3,000/month
  \item \textbf{Total}: \$8,000/month
\end{itemize}


\textbf{Migration Strategy}:
\begin{enumerate}
  \item \textbf{Assessment} (Week 1):
\end{enumerate}

\begin{itemize}
  \item Use AWS Schema Conversion Tool (SCT)
  \item Identify incompatible schema objects
  \item Estimate conversion effort
\end{itemize}


\begin{enumerate}
  \item \textbf{Schema Conversion} (Weeks 2-3):
\end{enumerate}

\begin{itemize}
  \item Convert DDL using SCT
  \item Modify incompatible SQL
  \item Test converted schema
\end{itemize}


\begin{enumerate}
  \item \textbf{Data Migration} (Week 4):
\end{enumerate}

\begin{itemize}
  \item Use AWS DMS for initial load
  \item Setup ongoing replication
  \item Validate data integrity
\end{itemize}


\begin{enumerate}
  \item \textbf{Cutover} (Week 5):
\end{enumerate}

\begin{itemize}
  \item Stop writes to source
  \item Final sync with DMS
  \item Switch application to Aurora
  \item Monitor performance
\end{itemize}


\textbf{After Migration}:
\begin{itemize}
  \item Aurora PostgreSQL: db.r5.2xlarge Multi-AZ
  \item Monthly cost: \$950
  \item \textbf{Savings}: \$7,050/month (88\% reduction)
  \item \textbf{ROI}: Migration cost recovered in 2 months
\end{itemize}


\textbf{Scenario 2: Self-Managed MongoDB to DocumentDB}

\textbf{Current State}: MongoDB on EC2
\begin{itemize}
  \item 3 r5.xlarge instances: \$438/month
  \item Management overhead: 20 hours/month
  \item Backup storage: \$100/month
  \item \textbf{Total}: \$538/month + management time
\end{itemize}


\textbf{Migration Process}:
\begin{enumerate}
  \item Create DocumentDB cluster
  \item Use mongodump/mongorestore
  \item Test application compatibility
  \item Switch connection strings
  \item Decommission EC2 instances
\end{enumerate}


\textbf{After Migration}:
\begin{itemize}
  \item DocumentDB: db.r5.large Multi-AZ
  \item Monthly cost: \$424/month
  \item \textbf{Savings}: \$114/month + management time
  \item \textbf{Benefits}: Automated backups, patching, monitoring
\end{itemize}


\subsubsection{Common Database Configuration Mistakes}


\textbf{RDS/Aurora}:
\begin{enumerate}
  \item Not enabling automated backups
  \item Single-AZ for production databases
  \item Not using read replicas for read-heavy workloads
  \item Over-provisioning instance size
  \item Not enabling encryption at rest
  \item Public accessibility enabled
  \item Not monitoring slow query logs
  \item Inadequate maintenance window planning
\end{enumerate}


\textbf{DynamoDB}:
\begin{enumerate}
  \item Not using on-demand for variable workloads
  \item Over-provisioning read/write capacity
  \item Poor partition key design (hot partitions)
  \item Not using DAX for read-heavy workloads
  \item Not implementing TTL for temporary data
  \item Missing Global Secondary Indexes
  \item Not using DynamoDB Streams for change capture
  \item Storing large items (> 400 KB inefficient)
\end{enumerate}


\textbf{ElastiCache}:
\begin{enumerate}
  \item Not implementing proper key expiration
  \item Single-node for production (no HA)
  \item Wrong cache eviction policy
  \item Not monitoring cache hit ratio
  \item Storing too much data per key
  \item Not using cluster mode for Redis
  \item Connection pooling not implemented
  \item Cache stampede not handled
\end{enumerate}


\subsubsection{Database Service Limits}


\textbf{RDS}:
\begin{itemize}
  \item DB instances per Region: 40 (can be increased)
  \item Max storage: 64 TB (SQL Server), 128 TB (others)
  \item Read replicas: 5 per primary (15 for Aurora)
  \item Automated backup retention: 35 days max
  \item Manual snapshots: 100 per Region (can be increased)
\end{itemize}


\textbf{DynamoDB}:
\begin{itemize}
  \item Table size: Unlimited
  \item Item size: 400 KB max
  \item Partition throughput: 3,000 RCU / 1,000 WCU per partition
  \item Global Secondary Indexes: 20 per table
  \item Local Secondary Indexes: 5 per table
  \item Tables per Region: 2,500 (soft limit)
\end{itemize}


\textbf{ElastiCache}:
\begin{itemize}
  \item Nodes per Region: 300 (soft limit)
  \item Nodes per cluster: 90 (Redis), 40 (Memcached)
  \item Parameter groups: 150 per Region
  \item Subnet groups: 150 per Region
\end{itemize}


\subsubsection{Database Monitoring and Troubleshooting}


\textbf{RDS/Aurora Monitoring}:
\begin{enumerate}
  \item \textbf{CloudWatch Metrics}:
\end{enumerate}

\begin{itemize}
  \item DatabaseConnections
  \item CPUUtilization
  \item FreeableMemory
  \item ReadLatency / WriteLatency
  \item ReadIOPS / WriteIOPS
\end{itemize}


\begin{enumerate}
  \item \textbf{Performance Insights}:
\end{enumerate}

\begin{itemize}
  \item Top SQL queries by load
  \item Wait events analysis
  \item Database load visualization
  \item Cost: Free for 7 days retention, paid for longer
\end{itemize}


\begin{enumerate}
  \item \textbf{Enhanced Monitoring}:
\end{enumerate}

\begin{itemize}
  \item OS-level metrics (50+ metrics)
  \item Real-time monitoring (1-second granularity)
  \item Process and thread information
\end{itemize}


\textbf{DynamoDB Monitoring}:
\begin{enumerate}
  \item \textbf{CloudWatch Metrics}:
\end{enumerate}

\begin{itemize}
  \item ConsumedReadCapacityUnits
  \item ConsumedWriteCapacityUnits
  \item UserErrors / SystemErrors
  \item ThrottledRequests
  \item ConditionalCheckFailedRequests
\end{itemize}


\begin{enumerate}
  \item \textbf{DynamoDB Contributor Insights}:
\end{enumerate}

\begin{itemize}
  \item Most accessed items
  \item Throttled requests
  \item Hot partitions identification
\end{itemize}


\begin{enumerate}
  \item \textbf{Common Issues}:
\end{enumerate}

\begin{itemize}
  \item Hot partitions: Redesign partition key
  \item Throttling: Increase capacity or use on-demand
  \item Large items: Break into smaller items
  \item High costs: Review capacity settings and use on-demand
\end{itemize}


\subsubsection{Database Best Practices}


\textbf{General Practices}:
\begin{enumerate}
  \item \textbf{Security}:
\end{enumerate}

\begin{itemize}
  \item Enable encryption at rest and in transit
  \item Use IAM database authentication
  \item Store credentials in Secrets Manager
  \item Apply least privilege access
  \item Regular security patching
\end{itemize}


\begin{enumerate}
  \item \textbf{High Availability}:
\end{enumerate}

\begin{itemize}
  \item Multi-AZ deployments for production
  \item Regular backup testing
  \item Automated failover testing
  \item Cross-Region replication for DR
\end{itemize}


\begin{enumerate}
  \item \textbf{Performance}:
\end{enumerate}

\begin{itemize}
  \item Right-size instances based on metrics
  \item Use read replicas for read scaling
  \item Implement connection pooling
  \item Monitor slow queries
  \item Regular index maintenance
\end{itemize}


\begin{enumerate}
  \item \textbf{Cost Optimization}:
\end{enumerate}

\begin{itemize}
  \item Use Reserved Instances for steady workloads
  \item Right-size instances (avoid over-provisioning)
  \item Delete old snapshots
  \item Use Aurora Serverless for variable workloads
  \item Implement data lifecycle policies
\end{itemize}


---

\subsection{Networking and Content Delivery}


\subsubsection{Amazon VPC (Virtual Private Cloud)}


\textbf{Isolated virtual network in AWS}

VPC allows you to provision a logically isolated section of the AWS Cloud where you can launch AWS resources in a virtual network you define.

\textbf{Core Concepts}:
\begin{itemize}
  \item \textbf{Your Private Network}: Define your own IP address range
  \item \textbf{CIDR Blocks}: Define IP address range (e.g., 10.0.0.0/16)
  \item \textbf{Subnets}: Divide VPC into subnets across Availability Zones
  \item \textbf{Route Tables}: Control traffic routing
  \item \textbf{Isolation}: Complete control over networking environment
\end{itemize}


\paragraph{VPC Components}


\textbf{Subnets}

Segments of your VPC's IP address range where you launch AWS resources.

\begin{itemize}
  \item \textbf{Public Subnet}:
  \item Has a route to an Internet Gateway
  \item Instances can have public IP addresses
  \item Accessible from the internet
  \item \textbf{Private Subnet}:
  \item No direct route to the internet
  \item Instances typically use private IP addresses only
  \item Access internet via NAT Gateway
\end{itemize}


\textbf{Internet Gateway (IGW)}
\begin{itemize}
  \item Horizontally scaled, redundant, highly available
  \item Allows communication between VPC and the internet
  \item Supports IPv4 and IPv6
  \item One IGW per VPC
\end{itemize}


\textbf{NAT Gateway}
\begin{itemize}
  \item Enables instances in private subnet to access the internet
  \item Prevents internet from initiating connections to private instances
  \item Managed by AWS (highly available within AZ)
  \item Alternative: NAT Instance (EC2 instance, customer-managed)
\end{itemize}


\textbf{Route Tables}
\begin{itemize}
  \item Control where network traffic is directed
  \item Each subnet must be associated with a route table
  \item Determines routing for traffic leaving the subnet
\end{itemize}


\paragraph{Security}


\textbf{Security Groups}

Virtual firewall for EC2 instances (instance-level).

\begin{itemize}
  \item \textbf{Level}: Instance level (first layer of defense)
  \item \textbf{Rules}: Allow rules only (no deny rules)
  \item \textbf{Statefulness}: Stateful - return traffic automatically allowed
  \item \textbf{Default}: All inbound traffic denied, all outbound allowed
  \item \textbf{Example}: Allow HTTP (port 80) from anywhere, SSH (port 22) from specific IP
\end{itemize}


\textbf{Network ACLs (Access Control Lists)}

Firewall for subnets (subnet-level).

\begin{itemize}
  \item \textbf{Level}: Subnet level (second layer of defense)
  \item \textbf{Rules}: Both allow and deny rules
  \item \textbf{Statefulness}: Stateless - must explicitly allow return traffic
  \item \textbf{Evaluation}: Rules processed in order (numbered)
  \item \textbf{Default}: Default NACL allows all inbound/outbound traffic
\end{itemize}


\textbf{Comparison}:

\begin{longtable}{lll}
\toprule
\textbf{Feature} & \textbf{Security Groups} & \textbf{Network ACLs} \\
\midrule
\textbf{Level} & Instance & Subnet \\
\textbf{State} & Stateful & Stateless \\
\textbf{Rules} & Allow only & Allow and Deny \\
\textbf{Rule Processing} & All rules evaluated & Rules in number order \\
\textbf{Applies To} & Instances & All instances in subnet \\
\bottomrule
\end{longtable}

\paragraph{Connectivity}


\textbf{VPC Peering}
\begin{itemize}
  \item Connect two VPCs privately
  \item Route traffic using private IP addresses
  \item Can peer VPCs across accounts and Regions
  \item Non-transitive (no chaining)
\end{itemize}


\textbf{VPN Gateway}
\begin{itemize}
  \item Connect on-premises network to VPC via VPN
  \item Encrypted connection over the internet
  \item Site-to-Site VPN
\end{itemize}


\textbf{Direct Connect}
\begin{itemize}
  \item Dedicated private connection from on-premises to AWS
  \item Bypass the public internet
  \item Covered in detail below
\end{itemize}


\subsubsection{Amazon CloudFront}


\textbf{Global Content Delivery Network (CDN)}

Deliver content to users with low latency and high transfer speeds.

\textbf{Key Features}:
\begin{itemize}
  \item \textbf{Edge Locations}: 400+ globally distributed edge locations
  \item \textbf{Caching}: Cache content closer to end users
  \item \textbf{Origin Support}: S3, EC2, ELB, on-premises servers
  \item \textbf{Performance}: Reduced latency for global users
  \item \textbf{Security}: Integration with AWS Shield (DDoS protection) and AWS WAF
\end{itemize}


\textbf{Use Cases}:
\begin{itemize}
  \item Static content delivery (images, CSS, JavaScript)
  \item Dynamic content delivery
  \item Video streaming (live and on-demand)
  \item API acceleration
  \item Software distribution
\end{itemize}


\textbf{How It Works}:
\begin{enumerate}
  \item User requests content
  \item Request routed to nearest edge location
  \item CloudFront checks cache
  \item If cached, content delivered immediately
  \item If not cached, CloudFront retrieves from origin, caches, and delivers
\end{enumerate}


\textbf{Benefits}:
\begin{itemize}
  \item Improved performance
  \item Cost reduction (reduced origin load)
  \item Global reach
  \item Enhanced security
\end{itemize}


\subsubsection{Amazon Route 53}


\textbf{Highly available and scalable DNS web service}

\textbf{Core Functions}:
\begin{enumerate}
  \item \textbf{Domain Registration}: Register domain names
  \item \textbf{DNS Routing}: Route internet traffic to resources
  \item \textbf{Health Checking}: Monitor resource health and route traffic accordingly
\end{enumerate}


\textbf{Routing Policies}:

\begin{longtable}{lll}
\toprule
\textbf{Policy} & \textbf{Description} & \textbf{Use Case} \\
\midrule
\textbf{Simple} & Single resource & One web server \\
\textbf{Weighted} & Distribute traffic by percentage & A/B testing, gradual migration \\
\textbf{Latency} & Route to lowest latency endpoint & Global applications \\
\textbf{Failover} & Active-passive failover & Disaster recovery \\
\textbf{Geolocation} & Route based on user's location & Content localization, compliance \\
\textbf{Geoproximity} & Route based on resource and user location & Bias traffic to specific locations \\
\textbf{Multi-value} & Return multiple IP addresses & Simple load balancing with health checks \\
\bottomrule
\end{longtable}

\textbf{Features}:
\begin{itemize}
  \item 100\% availability SLA
  \item Global anycast network
  \item DNSSEC for domain security
  \item Integration with AWS services
  \item Traffic flow for complex routing
\end{itemize}


\subsubsection{Elastic Load Balancing (ELB)}


\textbf{Automatically distribute incoming traffic across multiple targets}

Load balancers improve application availability and fault tolerance.

\paragraph{Load Balancer Types}


\textbf{Application Load Balancer (ALB)}

Layer 7 load balancer for HTTP/HTTPS traffic.

\begin{itemize}
  \item \textbf{OSI Layer}: Layer 7 (Application)
  \item \textbf{Protocols}: HTTP, HTTPS, gRPC
  \item \textbf{Routing}: Path-based, host-based, HTTP header-based
  \item \textbf{Targets}: EC2 instances, containers, IP addresses, Lambda functions
  \item \textbf{Features}:
  \item SSL/TLS termination
  \item WebSocket support
  \item HTTP/2 support
  \item Sticky sessions
  \item Authentication (OIDC, SAML)
  \item \textbf{Best For}: Web applications, microservices, container-based applications
\end{itemize}


\textbf{Network Load Balancer (NLB)}

Layer 4 load balancer for TCP/UDP traffic.

\begin{itemize}
  \item \textbf{OSI Layer}: Layer 4 (Transport)
  \item \textbf{Protocols}: TCP, UDP, TLS
  \item \textbf{Performance}: Millions of requests per second, ultra-low latency
  \item \textbf{Static IP}: Static IP addresses per AZ
  \item \textbf{Targets}: EC2 instances, containers, IP addresses
  \item \textbf{Features}:
  \item Extreme performance
  \item Static/Elastic IP addresses
  \item TLS termination
  \item Preserve source IP
  \item \textbf{Best For}: High performance, low latency requirements, TCP/UDP applications
\end{itemize}


\textbf{Gateway Load Balancer}

Layer 3 load balancer for virtual appliances.

\begin{itemize}
  \item \textbf{OSI Layer}: Layer 3 (Network)
  \item \textbf{Protocol}: IP
  \item \textbf{Use Case}: Deploy, scale, and manage third-party virtual appliances
  \item \textbf{Examples}: Firewalls, intrusion detection systems, deep packet inspection
  \item \textbf{Features}:
  \item GENEVE protocol support
  \item High availability for appliances
  \item Centralized security management
\end{itemize}


\textbf{Classic Load Balancer (CLB)}

Previous generation load balancer.

\begin{itemize}
  \item \textbf{Status}: Being phased out
  \item \textbf{OSI Layer}: Layer 4 and Layer 7
  \item \textbf{Recommendation}: Use ALB or NLB for new applications
\end{itemize}


\textbf{Choosing a Load Balancer}:
\begin{itemize}
  \item \textbf{ALB}: HTTP/HTTPS applications, microservices
  \item \textbf{NLB}: TCP/UDP applications, extreme performance
  \item \textbf{GWLB}: Third-party virtual appliances
\end{itemize}


\subsubsection{AWS Direct Connect}


\textbf{Dedicated private network connection}

Establish a dedicated private connection from on-premises to AWS.

\textbf{Key Features}:
\begin{itemize}
  \item \textbf{Connection Type}: Private, dedicated network connection
  \item \textbf{Bandwidth}: 1 Gbps, 10 Gbps, or 100 Gbps
  \item \textbf{Bypass Internet}: Traffic doesn't traverse the public internet
  \item \textbf{Performance}: Consistent network performance
  \item \textbf{Access}: Connect to both public AWS services and VPCs
  \item \textbf{Cost}: Reduced bandwidth costs for large data transfer
\end{itemize}


\textbf{Benefits}:
\begin{itemize}
  \item Predictable network performance
  \item Reduced bandwidth costs
  \item Consistent latency
  \item Enhanced security (private connection)
  \item Access to public and private AWS resources
\end{itemize}


\textbf{Setup Time}: Weeks to months to provision

\textbf{Use Cases}:
\begin{itemize}
  \item Large data transfers
  \item Real-time data feeds
  \item Hybrid cloud architectures
  \item Accessing VPCs across multiple Regions
\end{itemize}


\subsubsection{AWS VPN}


\textbf{Encrypted connection over the internet}

\paragraph{Site-to-Site VPN}


Connect on-premises network to AWS VPC.

\begin{itemize}
  \item \textbf{Connection}: Encrypted IPsec VPN tunnel over the internet
  \item \textbf{Components}: Customer Gateway (on-premises) + Virtual Private Gateway (AWS)
  \item \textbf{Setup Time}: Minutes to hours
  \item \textbf{Cost}: Lower cost than Direct Connect
  \item \textbf{Bandwidth}: Limited by internet connection
\end{itemize}


\paragraph{Client VPN}


Connect individual users to AWS or on-premises networks.

\begin{itemize}
  \item \textbf{Use Case}: Remote user access
  \item \textbf{Connection}: Encrypted TLS VPN connection
  \item \textbf{Access}: AWS VPC resources and on-premises resources
\end{itemize}


\textbf{VPN vs Direct Connect}:

\begin{longtable}{lll}
\toprule
\textbf{Feature} & \textbf{Site-to-Site VPN} & \textbf{Direct Connect} \\
\midrule
\textbf{Connection} & Over internet & Private dedicated line \\
\textbf{Setup Time} & Minutes & Weeks/months \\
\textbf{Cost} & Lower & Higher \\
\textbf{Bandwidth} & Internet-dependent & 1-100 Gbps \\
\textbf{Security} & Encrypted & Private (optionally encrypted) \\
\textbf{Performance} & Variable & Consistent \\
\bottomrule
\end{longtable}

---

\subsection{Management and Governance}


\subsubsection{AWS CloudFormation}


\textbf{Infrastructure as Code (IaC)}

Define and provision AWS infrastructure using template files.

\textbf{Key Concepts}:
\begin{itemize}
  \item \textbf{Templates}: JSON or YAML files defining resources
  \item \textbf{Stacks}: Collection of AWS resources managed as a single unit
  \item \textbf{Change Sets}: Preview changes before applying
\end{itemize}


\textbf{Features}:
\begin{itemize}
  \item \textbf{Automation}: Automate infrastructure provisioning
  \item \textbf{Version Control}: Track infrastructure changes over time
  \item \textbf{Consistency}: Ensure consistent deployments across environments
  \item \textbf{Reusability}: Reuse templates across accounts and Regions
  \item \textbf{Rollback}: Automatic rollback on errors
  \item \textbf{Cost}: No additional charge (pay for resources created)
\end{itemize}


\textbf{Use Cases}:
\begin{itemize}
  \item Disaster recovery
  \item Environment replication (dev, test, prod)
  \item Compliance and governance
  \item Infrastructure version control
\end{itemize}


\textbf{Benefits}:
\begin{itemize}
  \item Faster provisioning
  \item Reduced errors
  \item Consistent environments
  \item Easier management of complex architectures
\end{itemize}


\subsubsection{AWS CloudTrail}


\textbf{Governance, compliance, and auditing}

Log and monitor all API activity in your AWS account.

\textbf{Key Features}:
\begin{itemize}
  \item \textbf{API Logging}: Records all API calls in your account
  \item \textbf{Who, What, When, Where}: Comprehensive audit trail
  \item \textbf{Enabled by Default}: 90-day event history automatically
  \item \textbf{Long-term Storage}: Store logs in S3 for retention beyond 90 days
  \item \textbf{Integration}: CloudWatch Logs for monitoring and alerting
\end{itemize}


\textbf{Event Types}:
\begin{itemize}
  \item \textbf{Management Events}: Control plane operations (create instance, modify security group)
  \item \textbf{Data Events}: Data plane operations (S3 object access, Lambda function invocations)
  \item \textbf{Insights Events}: Unusual API activity detection
\end{itemize}


\textbf{Use Cases}:
\begin{itemize}
  \item Security analysis and compliance auditing
  \item Troubleshooting operational issues
  \item Change tracking
  \item Incident investigation
  \item Compliance requirements
\end{itemize}


\textbf{Benefits}:
\begin{itemize}
  \item Complete visibility into account activity
  \item Simplified compliance auditing
  \item Security analysis and incident response
  \item Operational troubleshooting
\end{itemize}


\subsubsection{Amazon CloudWatch}


\textbf{Monitoring and observability service}

Collect and track metrics, collect and monitor logs, set alarms, and automatically react to changes.

\paragraph{CloudWatch Components}


\textbf{CloudWatch Metrics}
\begin{itemize}
  \item Numerical time-series data points
  \item Built-in metrics for most AWS services (CPU, network, disk)
  \item Custom metrics for application-specific data
  \item Retention: Up to 15 months
\end{itemize}


\textbf{CloudWatch Logs}
\begin{itemize}
  \item Collect and store log files from resources
  \item Monitor and troubleshoot systems and applications
  \item Log Groups and Log Streams organization
  \item Retention policies (never expire to 1 day)
\end{itemize}


\textbf{CloudWatch Alarms}
\begin{itemize}
  \item Trigger actions based on metric thresholds
  \item States: OK, ALARM, INSUFFICIENT\_DATA
  \item Actions: SNS notifications, Auto Scaling, EC2 actions
  \item Composite alarms for complex conditions
\end{itemize}


\textbf{CloudWatch Events / EventBridge}
\begin{itemize}
  \item Event-driven automation
  \item Respond to state changes in AWS resources
  \item Schedule automated actions (cron jobs)
  \item Integration with Lambda, SQS, SNS, and more
\end{itemize}


\textbf{CloudWatch Dashboards}
\begin{itemize}
  \item Customizable home pages for monitoring
  \item Visualize metrics and logs
  \item Share across teams
\end{itemize}


\textbf{Use Cases}:
\begin{itemize}
  \item Application monitoring
  \item Performance optimization
  \item Troubleshooting
  \item Capacity planning
  \item Automated responses to operational changes
\end{itemize}


\subsubsection{AWS Systems Manager}


\textbf{Operational hub for AWS resources}

Centralized operations management for AWS and on-premises resources.

\textbf{Key Capabilities}:

\textbf{Operations Management}
\begin{itemize}
  \item \textbf{Session Manager}: Secure shell access without SSH keys or bastion hosts
  \item \textbf{Run Command}: Execute commands on multiple instances
  \item \textbf{Patch Manager}: Automate OS and application patching
  \item \textbf{Maintenance Windows}: Schedule operations tasks
\end{itemize}


\textbf{Configuration Management}
\begin{itemize}
  \item \textbf{Parameter Store}: Centralized storage for configuration data and secrets
  \item \textbf{State Manager}: Maintain consistent configuration
  \item \textbf{Inventory}: Collect metadata from managed instances
\end{itemize}


\textbf{Insights}
\begin{itemize}
  \item \textbf{OpsCenter}: Centralized operational issues management
  \item \textbf{CloudWatch Dashboard Integration}: Unified view
\end{itemize}


\textbf{Benefits}:
\begin{itemize}
  \item Reduced operational overhead
  \item Improved security (no SSH keys needed)
  \item Centralized management
  \item Automated operations
\end{itemize}


\textbf{Common Use Cases}:
\begin{itemize}
  \item Patch management across fleet
  \item Securely access instances
  \item Store application secrets
  \item Automate operational tasks
\end{itemize}


\subsubsection{AWS Trusted Advisor}


\textbf{Best practice recommendations}

Real-time guidance to help provision resources following AWS best practices.

\paragraph{Five Pillars of Recommendations}


\textbf{1. Cost Optimization}
\begin{itemize}
  \item Identify unused or underutilized resources
  \item Recommendations to reduce costs
  \item Examples: Idle RDS instances, unattached EBS volumes, unused Elastic IPs
\end{itemize}


\textbf{2. Performance}
\begin{itemize}
  \item Improve service performance
  \item Examples: High utilization of EC2 instances, EBS throughput optimization
\end{itemize}


\textbf{3. Security}
\begin{itemize}
  \item Close security gaps
  \item Examples: S3 bucket permissions, security group rules, MFA on root account
\end{itemize}


\textbf{4. Fault Tolerance}
\begin{itemize}
  \item Increase availability and redundancy
  \item Examples: Multi-AZ deployments, EBS snapshot age, Route 53 health checks
\end{itemize}


\textbf{5. Service Limits (Service Quotas)}
\begin{itemize}
  \item Check usage against service limits
  \item Avoid service disruptions from hitting limits
\end{itemize}


\paragraph{Access Levels}


\textbf{Free (All Customers)}
\begin{itemize}
  \item 7 core checks:
  \item S3 bucket permissions
  \item Security groups - specific ports unrestricted
  \item IAM use
  \item MFA on root account
  \item EBS public snapshots
  \item RDS public snapshots
  \item Service limits for common services
\end{itemize}


\textbf{Business or Enterprise Support}
\begin{itemize}
  \item All checks (50+ checks)
  \item Automated notifications
  \item AWS Support API access
  \item Programmatic access
\end{itemize}


\textbf{Dashboard}:
\begin{itemize}
  \item Color-coded status: Red (action recommended), Yellow (investigation recommended), Green (no problem)
  \item Downloadable reports
\end{itemize}


\subsubsection{AWS Control Tower}


\textbf{Set up and govern multi-account AWS environment}

Automated setup and governance for multi-account AWS environments.

\textbf{Key Features}:
\begin{itemize}
  \item \textbf{Landing Zone}: Automated multi-account setup based on best practices
  \item \textbf{Guardrails}: Governance rules for compliance
  \item \textbf{Preventive}: Prevent policy violations (using SCPs)
  \item \textbf{Detective}: Detect policy violations (using AWS Config)
  \item \textbf{Account Factory}: Automated account provisioning and configuration
  \item \textbf{Dashboard}: Centralized visibility and compliance monitoring
\end{itemize}


\textbf{Built On}:
\begin{itemize}
  \item AWS Organizations
  \item AWS Service Catalog
  \item AWS CloudFormation
  \item AWS IAM Identity Center (SSO)
  \item AWS Config
\end{itemize}


\textbf{Use Cases}:
\begin{itemize}
  \item Setting up new multi-account environments
  \item Governance for enterprise organizations
  \item Compliance requirements
  \item Account standardization
\end{itemize}


\subsubsection{AWS Service Catalog}


\textbf{Create and manage catalogs of approved IT services}

Enable centralized management of commonly deployed IT services.

\textbf{Key Features}:
\begin{itemize}
  \item \textbf{Product Portfolio}: Catalog of approved CloudFormation templates
  \item \textbf{Access Control}: Control who can deploy which services
  \item \textbf{Versioning}: Manage product versions
  \item \textbf{Constraints}: Define rules for product usage
  \item \textbf{Self-Service}: End users deploy approved resources
\end{itemize}


\textbf{Benefits}:
\begin{itemize}
  \item Standardized deployments
  \item Centralized management
  \item Governance and compliance
  \item Cost control
  \item Faster time to market
\end{itemize}


\textbf{Use Cases}:
\begin{itemize}
  \item Standardize infrastructure deployments
  \item Control cloud resource sprawl
  \item Enable self-service for developers
  \item Maintain compliance
\end{itemize}


---

\subsection{Additional Services}


\subsubsection{Amazon SNS (Simple Notification Service)}


\textbf{Pub/sub messaging service}

Fully managed messaging service for application-to-application (A2A) and application-to-person (A2P) communication.

\textbf{Key Concepts}:
\begin{itemize}
  \item \textbf{Topics}: Communication channels
  \item \textbf{Publishers}: Send messages to topics
  \item \textbf{Subscribers}: Receive messages from topics
  \item \textbf{Message Filtering}: Subscribers receive only relevant messages
\end{itemize}


\textbf{Supported Protocols}:
\begin{itemize}
  \item HTTP/HTTPS
  \item Email/Email-JSON
  \item SMS
  \item Amazon SQS
  \item AWS Lambda
  \item Mobile push notifications
\end{itemize}


\textbf{Use Cases}:
\begin{itemize}
  \item Application alerts and notifications
  \item Push notifications to mobile devices
  \item Email and SMS messaging
  \item Fanout pattern (one message to many subscribers)
  \item Decoupled microservices
\end{itemize}


\textbf{Benefits}:
\begin{itemize}
  \item Simple setup and operation
  \item High throughput
  \item Message durability and delivery
  \item Message filtering
\end{itemize}


\subsubsection{Amazon SQS (Simple Queue Service)}


\textbf{Fully managed message queue service}

Decouple and scale microservices, distributed systems, and serverless applications.

\textbf{Queue Types}:

\textbf{Standard Queues}
\begin{itemize}
  \item \textbf{Throughput}: Unlimited
  \item \textbf{Delivery}: At-least-once delivery (messages may be delivered multiple times)
  \item \textbf{Ordering}: Best-effort ordering
  \item \textbf{Use Case}: High throughput applications where occasional duplicates are acceptable
\end{itemize}


\textbf{FIFO Queues}
\begin{itemize}
  \item \textbf{Throughput}: Up to 3,000 messages per second (higher with batching)
  \item \textbf{Delivery}: Exactly-once processing
  \item \textbf{Ordering}: Strict ordering guaranteed
  \item \textbf{Use Case}: When message order is critical
\end{itemize}


\textbf{Features}:
\begin{itemize}
  \item \textbf{Retention}: Messages retained up to 14 days
  \item \textbf{Visibility Timeout}: Hide message while being processed
  \item \textbf{Dead-Letter Queues}: Handle failed messages
  \item \textbf{Delay Queues}: Postpone delivery of messages
  \item \textbf{Long Polling}: Reduce empty responses and costs
\end{itemize}


\textbf{Use Cases}:
\begin{itemize}
  \item Decouple application components
  \item Buffer between producers and consumers
  \item Batch processing
  \item Asynchronous processing
\end{itemize}


\textbf{SNS vs SQS}:
\begin{itemize}
  \item \textbf{SNS}: Push notifications, pub/sub, fanout
  \item \textbf{SQS}: Pull-based queue, decouple components, buffer
  \item \textbf{Together}: SNS sends to multiple SQS queues for fanout architecture
\end{itemize}


\subsubsection{AWS Step Functions}


\textbf{Serverless workflow orchestration}

Coordinate multiple AWS services into serverless workflows.

\textbf{Key Features}:
\begin{itemize}
  \item \textbf{Visual Workflows}: Graphical workflow designer
  \item \textbf{State Machines}: Define workflows as state machines
  \item \textbf{Service Integration}: Direct integration with AWS services
  \item \textbf{Error Handling}: Built-in retry and error handling
  \item \textbf{Standard and Express Workflows}:
  \item \textbf{Standard}: Long-running (up to 1 year), exactly-once execution
  \item \textbf{Express}: High-volume, short duration (up to 5 minutes), at-least-once execution
\end{itemize}


\textbf{Use Cases}:
\begin{itemize}
  \item Orchestrate microservices
  \item Data processing pipelines
  \item Automate IT and security processes
  \item Build complex workflows from Lambda functions
\end{itemize}


\textbf{Benefits}:
\begin{itemize}
  \item Visual workflow management
  \item Built-in error handling
  \item Serverless and scalable
  \item Audit trail of executions
\end{itemize}


\subsubsection{Amazon EventBridge}


\textbf{Serverless event bus service}

Connect applications using events from AWS services, SaaS applications, and custom applications.

\textbf{Key Concepts}:
\begin{itemize}
  \item \textbf{Events}: State changes or notifications
  \item \textbf{Event Buses}: Receive and route events
  \item \textbf{Rules}: Match events and route to targets
  \item \textbf{Targets}: Destinations for events (Lambda, SQS, SNS, etc.)
\end{itemize}


\textbf{Event Sources}:
\begin{itemize}
  \item AWS services (EC2, S3, etc.)
  \item SaaS applications (Salesforce, Datadog, etc.)
  \item Custom applications
\end{itemize}


\textbf{Features}:
\begin{itemize}
  \item \textbf{Schema Registry}: Discover and manage event schemas
  \item \textbf{Archive and Replay}: Store and replay events
  \item \textbf{Cross-Account Events}: Send events across AWS accounts
  \item \textbf{Filtering}: Advanced event pattern matching
\end{itemize}


\textbf{Previously}: CloudWatch Events (EventBridge is the enhanced version)

\textbf{Use Cases}:
\begin{itemize}
  \item React to state changes in AWS services
  \item Schedule automated actions
  \item Integrate SaaS applications
  \item Event-driven architectures
\end{itemize}


\subsubsection{AWS Batch}


\textbf{Fully managed batch processing}

Run batch computing workloads at any scale.

\textbf{Key Features}:
\begin{itemize}
  \item \textbf{Automatic Provisioning}: Dynamically provisions compute resources
  \item \textbf{Job Queues}: Queue and prioritize jobs
  \item \textbf{Job Definitions}: Define how jobs are run
  \item \textbf{Scaling}: Automatically scales based on job volume
  \item \textbf{Integration}: Uses EC2 and Spot Instances
\end{itemize}


\textbf{Use Cases}:
\begin{itemize}
  \item Data processing and analysis
  \item Image and video rendering
  \item Financial risk modeling
  \item Scientific simulations
  \item ETL (Extract, Transform, Load) jobs
\end{itemize}


\textbf{Benefits}:
\begin{itemize}
  \item No infrastructure management
  \item Cost optimization with Spot Instances
  \item Automatic scaling
  \item Job dependency management
\end{itemize}


\subsubsection{Amazon Athena}


\textbf{Interactive query service for S3}

Analyze data in Amazon S3 using standard SQL.

\textbf{Key Features}:
\begin{itemize}
  \item \textbf{Serverless}: No infrastructure to manage
  \item \textbf{SQL Queries}: Standard SQL support
  \item \textbf{Pay Per Query}: Charged based on data scanned
  \item \textbf{Integration}: Works with AWS Glue Data Catalog
  \item \textbf{Formats}: Supports CSV, JSON, ORC, Avro, Parquet
\end{itemize}


\textbf{Use Cases}:
\begin{itemize}
  \item Ad-hoc data analysis
  \item Log analysis
  \item Business intelligence
  \item Query CloudTrail logs
  \item Analyze application logs
\end{itemize}


\textbf{Benefits}:
\begin{itemize}
  \item No ETL required
  \item Fast query performance
  \item Cost-effective (pay only for queries)
  \item Easy to use (just SQL)
\end{itemize}


\textbf{Best Practices}:
\begin{itemize}
  \item Use columnar formats (Parquet, ORC) for better performance
  \item Partition data to reduce scanned data
  \item Compress data to reduce costs
\end{itemize}


\subsubsection{AWS Glue}


\textbf{Fully managed ETL (Extract, Transform, Load) service}

Prepare data for analytics and machine learning.

\textbf{Key Components}:

\textbf{AWS Glue Data Catalog}
\begin{itemize}
  \item Centralized metadata repository
  \item Stores table definitions, schemas, and metadata
  \item Integration point for Athena, EMR, Redshift Spectrum
\end{itemize}


\textbf{AWS Glue Crawlers}
\begin{itemize}
  \item Automatically discover and catalog data
  \item Infer schemas
  \item Update the Data Catalog
\end{itemize}


\textbf{AWS Glue ETL Jobs}
\begin{itemize}
  \item Serverless ETL operations
  \item Generate code in Python or Scala
  \item Visual ETL editor
  \item Built-in transformations
\end{itemize}


\textbf{Use Cases}:
\begin{itemize}
  \item Prepare data for analytics
  \item Data lake management
  \item Database migration
  \item Data pipeline automation
\end{itemize}


\textbf{Benefits}:
\begin{itemize}
  \item Serverless (no infrastructure management)
  \item Automatic schema discovery
  \item Integrated with AWS analytics services
  \item Cost-effective
\end{itemize}


\subsubsection{Amazon QuickSight}


\textbf{Business intelligence (BI) service}

Create visualizations, perform ad-hoc analysis, and get business insights.

\textbf{Key Features}:
\begin{itemize}
  \item \textbf{Serverless}: No servers to manage
  \item \textbf{Visualizations}: Interactive dashboards and visualizations
  \item \textbf{ML Insights}: Automatic insights powered by machine learning
  \item \textbf{SPICE Engine}: In-memory calculation engine for fast performance
  \item \textbf{Sharing}: Share dashboards with users and groups
  \item \textbf{Embedded Analytics}: Embed in applications
\end{itemize}


\textbf{Data Sources}:
\begin{itemize}
  \item AWS services (RDS, Aurora, Redshift, S3, Athena)
  \item On-premises databases
  \item SaaS applications (Salesforce, Jira)
  \item File uploads (Excel, CSV, JSON)
\end{itemize}


\textbf{Pricing}:
\begin{itemize}
  \item Pay per session (users pay only when accessing dashboards)
  \item Author and Reader pricing tiers
\end{itemize}


\textbf{Use Cases}:
\begin{itemize}
  \item Business intelligence and reporting
  \item Data visualization
  \item Embedded analytics
  \item Ad-hoc analysis
\end{itemize}


\subsubsection{Amazon Kinesis}


\textbf{Real-time data streaming}

Collect, process, and analyze real-time streaming data.

\paragraph{Kinesis Services}


\textbf{Kinesis Data Streams}
\begin{itemize}
  \item Capture and store data streams in real-time
  \item Shards for parallel processing
  \item Retain data for 1-365 days
  \item Use Case: Custom real-time applications
\end{itemize}


\textbf{Kinesis Data Firehose}
\begin{itemize}
  \item Load streaming data into data stores
  \item Destinations: S3, Redshift, Elasticsearch, Splunk, HTTP endpoints
  \item Near real-time (60 seconds minimum)
  \item Automatic scaling
  \item Use Case: Simple data ingestion pipeline
\end{itemize}


\textbf{Kinesis Data Analytics}
\begin{itemize}
  \item Analyze streaming data with SQL or Apache Flink
  \item Real-time analytics and insights
  \item Use Case: Real-time dashboards, metrics, anomaly detection
\end{itemize}


\textbf{Kinesis Video Streams}
\begin{itemize}
  \item Capture and store video streams
  \item Process and analyze video
  \item Use Case: Video analytics, machine learning on video
\end{itemize}


\textbf{Common Use Cases}:
\begin{itemize}
  \item Real-time log and event data collection
  \item IoT device telemetry
  \item Clickstream analysis
  \item Real-time analytics
  \item Video processing and analysis
\end{itemize}


\subsubsection{Amazon SageMaker}


\textbf{Build, train, and deploy machine learning models}

Fully managed service for the complete machine learning workflow.

\textbf{Key Capabilities}:

\textbf{Build}
\begin{itemize}
  \item Integrated Jupyter notebooks
  \item Pre-built algorithms and frameworks
  \item Data labeling service (Ground Truth)
  \item Feature Store
\end{itemize}


\textbf{Train}
\begin{itemize}
  \item One-click training
  \item Automatic model tuning (hyperparameter optimization)
  \item Distributed training
  \item Managed Spot training for cost savings
\end{itemize}


\textbf{Deploy}
\begin{itemize}
  \item One-click deployment
  \item Real-time and batch inference
  \item Multi-model endpoints
  \item Auto-scaling
\end{itemize}


\textbf{Additional Features}:
\begin{itemize}
  \item SageMaker Studio (ML IDE)
  \item Model monitoring and debugging
  \item MLOps capabilities
  \item Pre-built ML solutions
\end{itemize}


\textbf{Use Cases}:
\begin{itemize}
  \item Predictive analytics
  \item Recommendation systems
  \item Fraud detection
  \item Image and text classification
  \item Time series forecasting
\end{itemize}


\subsubsection{Amazon Rekognition}


\textbf{Image and video analysis}

Add image and video analysis to applications using deep learning.

\textbf{Capabilities}:
\begin{itemize}
  \item \textbf{Object and Scene Detection}: Identify thousands of objects and scenes
  \item \textbf{Facial Analysis}: Detect faces and analyze attributes (age, gender, emotions)
  \item \textbf{Face Comparison}: Compare faces for verification
  \item \textbf{Face Recognition}: Identify known faces in images and videos
  \item \textbf{Celebrity Recognition}: Recognize thousands of celebrities
  \item \textbf{Text Detection}: Detect and extract text from images
  \item \textbf{Content Moderation}: Detect inappropriate content
  \item \textbf{Video Analysis}: Track people, objects, activities in videos
\end{itemize}


\textbf{Use Cases}:
\begin{itemize}
  \item User verification
  \item Content moderation
  \item Searchable image library
  \item Sentiment analysis
  \item Security and surveillance
  \item Media analysis
\end{itemize}


\subsubsection{Amazon Comprehend}


\textbf{Natural Language Processing (NLP) service}

Extract insights and relationships from text using machine learning.

\textbf{Capabilities}:
\begin{itemize}
  \item \textbf{Sentiment Analysis}: Determine positive, negative, neutral, or mixed sentiment
  \item \textbf{Entity Recognition}: Identify people, places, brands, events, etc.
  \item \textbf{Key Phrase Extraction}: Extract important phrases from text
  \item \textbf{Language Detection}: Identify the dominant language
  \item \textbf{Topic Modeling}: Organize documents by topic
  \item \textbf{Custom Classification}: Train custom models for specific domains
\end{itemize}


\textbf{Use Cases}:
\begin{itemize}
  \item Customer feedback analysis
  \item Document classification
  \item Social media monitoring
  \item Knowledge management
  \item Business intelligence from documents
\end{itemize}


\subsubsection{Amazon Lex}


\textbf{Build conversational interfaces (chatbots)}

Build chatbots and voice assistants using the same technology as Amazon Alexa.

\textbf{Key Features}:
\begin{itemize}
  \item \textbf{Automatic Speech Recognition (ASR)}: Convert speech to text
  \item \textbf{Natural Language Understanding (NLU)}: Understand intent
  \item \textbf{Multi-turn Conversations}: Context management across conversation
  \item \textbf{8 kHz Telephony Audio}: Support for phone calls
  \item \textbf{Integration}: Connect to AWS Lambda, mobile apps, messaging platforms
\end{itemize}


\textbf{Use Cases}:
\begin{itemize}
  \item Customer service chatbots
  \item Virtual assistants
  \item Voice-enabled applications
  \item Interactive Voice Response (IVR) systems
  \item Information bots
\end{itemize}


\textbf{Components}:
\begin{itemize}
  \item \textbf{Intents}: Actions users want to perform
  \item \textbf{Utterances}: Phrases users might say
  \item \textbf{Slots}: Parameters for intents
  \item \textbf{Fulfillment}: Lambda function to fulfill the intent
\end{itemize}


\subsubsection{AWS Migration Hub}


\textbf{Track application migrations}

Centralized location to track progress of application migrations across multiple AWS and partner solutions.

\textbf{Key Features}:
\begin{itemize}
  \item \textbf{Single Dashboard}: View migration status across tools
  \item \textbf{Migration Tracking}: Track server and database migrations
  \item \textbf{Integration}: Works with AWS migration tools and partner tools
  \item \textbf{Application Grouping}: Group servers by application
\end{itemize}


\textbf{Integrated Tools}:
\begin{itemize}
  \item AWS Application Migration Service
  \item AWS Database Migration Service
  \item CloudEndure Migration
  \item ATADATA ATAmotion
  \item RiverMeadow Server Migration
\end{itemize}


\textbf{Benefits}:
\begin{itemize}
  \item Centralized visibility
  \item Track progress across multiple tools
  \item Identify and troubleshoot issues
  \item Plan and execute migrations
\end{itemize}


\subsubsection{AWS Database Migration Service (DMS)}


\textbf{Migrate databases to AWS}

Migrate databases to AWS quickly and securely while the source database remains operational.

\textbf{Key Features}:
\begin{itemize}
  \item \textbf{Minimal Downtime}: Source database remains operational during migration
  \item \textbf{Continuous Data Replication}: Ongoing replication after initial migration
  \item \textbf{Homogeneous Migrations}: Same database engine (Oracle to Oracle)
  \item \textbf{Heterogeneous Migrations}: Different database engines (Oracle to Aurora)
  \item \textbf{Schema Conversion}: AWS Schema Conversion Tool (SCT) for heterogeneous migrations
\end{itemize}


\textbf{Supported Sources}:
\begin{itemize}
  \item Oracle, SQL Server, MySQL, PostgreSQL, MongoDB, SAP, DB2
  \item On-premises and cloud databases
\end{itemize}


\textbf{Supported Targets}:
\begin{itemize}
  \item Amazon RDS, Aurora, Redshift, DynamoDB, S3, DocumentDB
\end{itemize}


\textbf{Use Cases}:
\begin{itemize}
  \item Migrate to cloud
  \item Replicate for development/test
  \item Database consolidation
  \item Continuous data replication
\end{itemize}


\textbf{Benefits}:
\begin{itemize}
  \item Cost-effective
  \item Minimal downtime
  \item Supports most databases
  \item Reliable and self-healing
\end{itemize}


\subsubsection{AWS Application Discovery Service}


\textbf{Discover on-premises applications for migration planning}

Collect information about on-premises data centers to plan migrations.

\textbf{Discovery Methods}:

\textbf{Agentless Discovery (Application Discovery Service Agentless Collector)}
\begin{itemize}
  \item VMware environment only
  \item Collects VM inventory, configuration, performance
  \item No agent installation required
\end{itemize}


\textbf{Agent-based Discovery (Application Discovery Agent)}
\begin{itemize}
  \item Install agent on servers
  \item Collects system configuration, performance, running processes, network connections
  \item Works on physical and virtual servers
\end{itemize}


\textbf{Data Collected}:
\begin{itemize}
  \item Server specifications (CPU, memory, disk)
  \item Resource utilization
  \item Network dependencies
  \item Running processes and applications
\end{itemize}


\textbf{Integration}:
\begin{itemize}
  \item Data exported to Amazon S3
  \item Integrate with AWS Migration Hub
  \item Visualize dependencies with AWS Application Discovery Service
\end{itemize}


\textbf{Use Cases}:
\begin{itemize}
  \item Migration planning
  \item Identify dependencies between applications
  \item Right-size target infrastructure
  \item Total cost of ownership (TCO) analysis
\end{itemize}


---

\subsection{Review Questions}


Test your knowledge of AWS Cloud Technology and Services:

\subsubsection{Question 1}

\textbf{Which AWS service provides a serverless compute platform that automatically scales and charges based on the number of requests and compute time?}

A) Amazon EC2
B) AWS Lambda
C) Amazon Lightsail
D) AWS Elastic Beanstalk

<details>
<summary>Click to reveal answer</summary>

\textbf{Answer: B) AWS Lambda}

Lambda is serverless, automatically scales, and charges based on requests and compute time. EC2 requires instance management, Lightsail has predictable pricing, and Elastic Beanstalk is a PaaS that still provisions underlying resources.
</details>

---

\subsubsection{Question 2}

\textbf{Your company needs to store frequently accessed files that must be available immediately. Which S3 storage class would be the MOST cost-effective?}

A) S3 Glacier Deep Archive
B) S3 Standard-IA
C) S3 Standard
D) S3 One Zone-IA

<details>
<summary>Click to reveal answer</summary>

\textbf{Answer: C) S3 Standard}

For frequently accessed data requiring immediate availability, S3 Standard is the appropriate choice. Glacier Deep Archive has long retrieval times, and IA (Infrequent Access) classes charge retrieval fees that would add up with frequent access.
</details>

---

\subsubsection{Question 3}

\textbf{Which EC2 pricing model would be MOST appropriate for a production database that must run continuously for the next three years?}

A) On-Demand Instances
B) Spot Instances
C) Reserved Instances
D) Dedicated Hosts

<details>
<summary>Click to reveal answer</summary>

\textbf{Answer: C) Reserved Instances}

For steady-state workloads running continuously, Reserved Instances provide the best cost savings (up to 75\%). On-Demand is most expensive, Spot Instances can be interrupted (unsuitable for databases), and Dedicated Hosts are for specific compliance/licensing needs.
</details>

---

\subsubsection{Question 4}

\textbf{What is the PRIMARY difference between Security Groups and Network ACLs?}

A) Security Groups are stateful, Network ACLs are stateless
B) Security Groups apply to subnets, Network ACLs apply to instances
C) Security Groups allow deny rules, Network ACLs only allow rules
D) Security Groups are free, Network ACLs have additional costs

<details>
<summary>Click to reveal answer</summary>

\textbf{Answer: A) Security Groups are stateful, Network ACLs are stateless}

Security Groups are stateful (return traffic automatically allowed), while Network ACLs are stateless (must explicitly allow return traffic). Security Groups apply to instances, NACLs apply to subnets. Security Groups only have allow rules, NACLs have both allow and deny rules. Both are free.
</details>

---

\subsubsection{Question 5}

\textbf{Which database service would be BEST for a social networking application that needs to efficiently query relationships between users?}

A) Amazon RDS
B) Amazon DynamoDB
C) Amazon Neptune
D) Amazon Redshift

<details>
<summary>Click to reveal answer</summary>

\textbf{Answer: C) Amazon Neptune}

Neptune is a graph database designed for highly connected data and relationship queries (social networks, recommendation engines). RDS is for relational data, DynamoDB for key-value data, and Redshift for analytics/data warehousing.
</details>

---

\subsubsection{Question 6}

\textbf{Your application needs to send notifications to multiple subscribers via email, SMS, and HTTP endpoints. Which service should you use?}

A) Amazon SQS
B) Amazon SNS
C) AWS Step Functions
D) Amazon EventBridge

<details>
<summary>Click to reveal answer</summary>

\textbf{Answer: B) Amazon SNS}

SNS is a pub/sub messaging service that can send notifications to multiple subscribers across multiple protocols (email, SMS, HTTP). SQS is a message queue, Step Functions orchestrates workflows, and EventBridge is for event-driven architectures.
</details>

---

\subsubsection{Question 7}

\textbf{Which AWS service allows you to define infrastructure as code using JSON or YAML templates?}

A) AWS CloudTrail
B) AWS CloudFormation
C) AWS Config
D) AWS Systems Manager

<details>
<summary>Click to reveal answer</summary>

\textbf{Answer: B) AWS CloudFormation}

CloudFormation uses templates (JSON/YAML) to define infrastructure as code. CloudTrail logs API calls, Config tracks resource configurations, and Systems Manager manages operations.
</details>

---

\subsubsection{Question 8}

\textbf{What is the MINIMUM number of Availability Zones in an AWS Region?}

A) 1
B) 2
C) 3
D) 4

<details>
<summary>Click to reveal answer</summary>

\textbf{Answer: C) 3}

All AWS Regions have a minimum of 3 Availability Zones to provide high availability and fault tolerance.
</details>

---

\subsubsection{Question 9}

\textbf{Which AWS service provides a private, dedicated network connection from your on-premises data center to AWS?}

A) AWS VPN
B) AWS Direct Connect
C) Amazon CloudFront
D) AWS Transit Gateway

<details>
<summary>Click to reveal answer</summary>

\textbf{Answer: B) AWS Direct Connect}

Direct Connect provides a dedicated private network connection. VPN uses encrypted connection over the internet, CloudFront is a CDN, and Transit Gateway connects VPCs and on-premises networks.
</details>

---

\subsubsection{Question 10}

\textbf{Your company needs to migrate 100 TB of data to AWS. Network bandwidth is limited. Which service should you use?}

A) AWS DataSync
B) AWS Snowball
C) Amazon S3 Transfer Acceleration
D) AWS Direct Connect

<details>
<summary>Click to reveal answer</summary>

\textbf{Answer: B) AWS Snowball}

For large data migrations with limited bandwidth, Snowball Edge (80 TB capacity) is ideal. You would use 2 devices for 100 TB. DataSync is for ongoing transfers, Transfer Acceleration speeds up S3 uploads over internet, and Direct Connect takes weeks to provision.
</details>

---

\subsubsection{Question 11}

\textbf{Which service would you use to analyze data in S3 using standard SQL without moving the data?}

A) Amazon Redshift
B) Amazon Athena
C) AWS Glue
D) Amazon EMR

<details>
<summary>Click to reveal answer</summary>

\textbf{Answer: B) Amazon Athena}

Athena allows you to query data directly in S3 using SQL without loading data into a database. Redshift is a data warehouse (requires loading data), Glue is for ETL, and EMR is for big data processing frameworks.
</details>

---

\subsubsection{Question 12}

\textbf{What AWS service provides real-time guidance to help you provision resources following AWS best practices?}

A) AWS CloudTrail
B) AWS Trusted Advisor
C) AWS Inspector
D) AWS Config

<details>
<summary>Click to reveal answer</summary>

\textbf{Answer: B) AWS Trusted Advisor}

Trusted Advisor provides real-time best practice recommendations across 5 categories. CloudTrail logs API calls, Inspector assesses application security, and Config tracks resource configurations.
</details>

---

\subsubsection{Question 13}

\textbf{Which load balancer type operates at Layer 7 and can route traffic based on URL path?}

A) Classic Load Balancer
B) Network Load Balancer
C) Application Load Balancer
D) Gateway Load Balancer

<details>
<summary>Click to reveal answer</summary>

\textbf{Answer: C) Application Load Balancer}

ALB operates at Layer 7 (application layer) and supports advanced routing including path-based and host-based routing. NLB operates at Layer 4, Gateway Load Balancer at Layer 3, and Classic Load Balancer is being phased out.
</details>

---

\subsubsection{Question 14}

\textbf{Which AWS service would you use to coordinate multiple AWS services into a serverless workflow?}

A) Amazon SQS
B) Amazon SNS
C) AWS Step Functions
D) AWS Lambda

<details>
<summary>Click to reveal answer</summary>

\textbf{Answer: C) AWS Step Functions}

Step Functions orchestrates multiple AWS services into serverless workflows with visual workflow design. SQS is a message queue, SNS is pub/sub messaging, and Lambda executes individual functions.
</details>

---

\subsubsection{Question 15}

\textbf{Your application requires block storage for an EC2 instance. Which service should you use?}

A) Amazon S3
B) Amazon EBS
C) Amazon EFS
D) AWS Storage Gateway

<details>
<summary>Click to reveal answer</summary>

\textbf{Answer: B) Amazon EBS}

EBS provides block storage for EC2 instances. S3 is object storage, EFS is file storage for multiple instances, and Storage Gateway is for hybrid cloud storage.
</details>

---

\subsubsection{Question 16}

\textbf{Which DynamoDB capacity mode should you choose for unpredictable, variable workloads?}

A) Provisioned Capacity
B) Reserved Capacity
C) On-Demand
D) Spot Capacity

<details>
<summary>Click to reveal answer</summary>

\textbf{Answer: C) On-Demand}

On-Demand capacity mode is ideal for unpredictable workloads as you pay per request. Provisioned Capacity requires setting read/write capacity units in advance. Reserved and Spot Capacity don't exist for DynamoDB.
</details>

---

\subsubsection{Question 17}

\textbf{What is the purpose of AWS CloudFront Edge Locations?}

A) Run EC2 instances closer to users
B) Cache content closer to users for faster delivery
C) Store backups in multiple locations
D) Host databases in multiple regions

<details>
<summary>Click to reveal answer</summary>

\textbf{Answer: B) Cache content closer to users for faster delivery}

Edge Locations are part of CloudFront CDN and cache content to reduce latency. They don't run EC2 instances, store backups, or host databases.
</details>

---

\subsubsection{Question 18}

\textbf{Which AWS service logs all API calls made in your AWS account for auditing and compliance?}

A) Amazon CloudWatch
B) AWS CloudTrail
C) AWS Config
D) AWS Inspector

<details>
<summary>Click to reveal answer</summary>

\textbf{Answer: B) AWS CloudTrail}

CloudTrail logs all API calls for governance, compliance, and auditing. CloudWatch monitors metrics and logs, Config tracks resource configurations, and Inspector assesses security vulnerabilities.
</details>

---

\subsubsection{Question 19}

\textbf{Which storage service is best for shared file storage accessible by multiple EC2 instances simultaneously?}

A) Amazon EBS
B) Amazon S3
C) Amazon EFS
D) Instance Store

<details>
<summary>Click to reveal answer</summary>

\textbf{Answer: C) Amazon EFS}

EFS provides shared NFS file system accessible by multiple EC2 instances simultaneously. EBS attaches to single instance, S3 is object storage (not file system), and Instance Store is ephemeral.
</details>

---

\subsubsection{Question 20}

\textbf{Your company needs to run Docker containers without managing the underlying infrastructure. Which service combination is MOST appropriate?}

A) Amazon ECS with EC2 launch type
B) Amazon ECS with Fargate launch type
C) Amazon EKS with EC2 nodes
D) Amazon EC2 with Docker installed

<details>
<summary>Click to reveal answer</summary>

\textbf{Answer: B) Amazon ECS with Fargate launch type}

ECS with Fargate is serverless - you don't manage any infrastructure. ECS with EC2 and EKS with EC2 require managing EC2 instances. Running Docker on EC2 requires full infrastructure management.
</details>

---

\subsubsection{Question 21}

\textbf{A company needs to process large amounts of data for scientific simulations. The workload runs for 8 hours per day with predictable schedules. Which EC2 pricing model provides the BEST cost optimization?}

A) On-Demand Instances with Auto Scaling
B) Spot Instances
C) Reserved Instances with Scheduled Reserved Instances
D) Dedicated Hosts

<details>
<summary>Click to reveal answer</summary>

\textbf{Answer: C) Reserved Instances with Scheduled Reserved Instances}

Scheduled Reserved Instances allow you to reserve capacity for predictable recurring schedules (daily, weekly, monthly). Since the workload runs predictably for 8 hours/day, this provides significant savings compared to On-Demand while ensuring capacity availability. Spot Instances could be interrupted, and Dedicated Hosts are unnecessarily expensive for this use case.
</details>

---

\subsubsection{Question 22}

\textbf{Which storage service provides the LOWEST cost for storing 200 TB of data that must be retained for 10 years for compliance but will never be accessed unless required by auditors?}

A) S3 Standard
B) S3 Glacier Flexible Retrieval
C) S3 Glacier Deep Archive
D) S3 One Zone-IA

<details>
<summary>Click to reveal answer</summary>

\textbf{Answer: C) S3 Glacier Deep Archive}

Glacier Deep Archive offers the lowest storage cost (\$0.00099 per GB/month) and is specifically designed for long-term archival storage with retrieval times of 12-48 hours. The 200 TB would cost approximately \$200/month compared to \$4,600/month in S3 Standard. The long retrieval time is acceptable for compliance data rarely accessed.
</details>

---

\subsubsection{Question 23}

\textbf{Your application needs to make millions of cache lookups per second with sub-millisecond latency. Which service should you use?}

A) Amazon RDS with read replicas
B) Amazon DynamoDB with DAX
C) Amazon ElastiCache
D) Amazon Aurora

<details>
<summary>Click to reveal answer</summary>

\textbf{Answer: B) Amazon DynamoDB with DAX}

DynamoDB Accelerator (DAX) provides microsecond latency for millions of requests per second, making it ideal for extreme performance requirements. ElastiCache could also work, but DAX is specifically designed for DynamoDB and provides the best integration. RDS and Aurora have higher latency (milliseconds).
</details>

---

\subsubsection{Question 24}

\textbf{A company wants to analyze customer behavior by querying relationships like "customers who bought this also bought that." Which database is MOST appropriate?}

A) Amazon RDS
B) Amazon DynamoDB
C) Amazon Neptune
D) Amazon Redshift

<details>
<summary>Click to reveal answer</summary>

\textbf{Answer: C) Amazon Neptune}

Neptune is a graph database designed for querying highly connected data and relationships. It excels at queries like friend-of-friend, recommendations, and relationship traversals. RDS and DynamoDB would require complex joins or denormalization, and Redshift is for analytics, not real-time relationship queries.
</details>

---

\subsubsection{Question 25}

\textbf{Which combination provides the MOST cost-effective solution for a serverless web application with unpredictable traffic?}

A) EC2 Auto Scaling + RDS
B) Lambda + DynamoDB On-Demand + S3
C) ECS Fargate + Aurora Serverless
D) Lightsail + RDS

<details>
<summary>Click to reveal answer</summary>

\textbf{Answer: B) Lambda + DynamoDB On-Demand + S3}

This combination provides true serverless scaling with pay-per-use pricing. Lambda charges only for actual executions, DynamoDB On-Demand charges per request, and S3 charges for storage and requests. This is ideal for unpredictable traffic with zero waste during low-traffic periods. Option C is also serverless but more expensive.
</details>

---

\subsubsection{Question 26}

\textbf{Your company needs to ensure that EC2 instances in a private subnet can download software updates from the internet without being directly accessible from the internet. What should you configure?}

A) Internet Gateway
B) NAT Gateway
C) Virtual Private Gateway
D) VPC Peering

<details>
<summary>Click to reveal answer</summary>

\textbf{Answer: B) NAT Gateway}

NAT Gateway enables instances in private subnets to initiate outbound connections to the internet while preventing inbound connections from the internet. An Internet Gateway would require instances to be in public subnets with public IPs, making them directly accessible.
</details>

---

\subsubsection{Question 27}

\textbf{Which AWS service automatically distributes incoming application traffic across multiple targets and can route based on URL path?}

A) Network Load Balancer
B) Application Load Balancer
C) Classic Load Balancer
D) CloudFront

<details>
<summary>Click to reveal answer</summary>

\textbf{Answer: B) Application Load Balancer}

ALB operates at Layer 7 and supports content-based routing including path-based, host-based, and HTTP header-based routing. NLB operates at Layer 4 and doesn't inspect application-layer content. CloudFront is a CDN, not a load balancer.
</details>

---

\subsubsection{Question 28}

\textbf{A company needs to migrate 500 TB of data to AWS but has limited network bandwidth (10 Mbps). What is the MOST efficient migration method?}

A) AWS DataSync over internet
B) AWS Direct Connect
C) AWS Snowball Edge devices
D) S3 Transfer Acceleration

<details>
<summary>Click to reveal answer</summary>

\textbf{Answer: C) AWS Snowball Edge devices}

At 10 Mbps, transferring 500 TB would take approximately 463 days. Snowball Edge devices (80 TB capacity each) can transfer this data in weeks. You would order 7 devices, copy data locally, and ship them to AWS. Direct Connect takes weeks to provision, and DataSync/Transfer Acceleration would still be limited by bandwidth.
</details>

---

\subsubsection{Question 29}

\textbf{Which RDS feature provides automatic failover to a standby instance in a different Availability Zone?}

A) Read Replicas
B) Multi-AZ deployment
C) Automated Backups
D) Manual Snapshots

<details>
<summary>Click to reveal answer</summary>

\textbf{Answer: B) Multi-AZ deployment}

Multi-AZ provides synchronous replication to a standby instance and automatic failover (typically under 1 minute). Read Replicas use asynchronous replication and require manual promotion. Backups and snapshots don't provide automatic failover.
</details>

---

\subsubsection{Question 30}

\textbf{Your application needs to store session data that expires after 24 hours. Which service combination is MOST cost-effective?}

A) RDS with custom cleanup scripts
B) DynamoDB with Time-To-Live (TTL)
C) S3 with Lifecycle policies
D) ElastiCache with manual deletion

<details>
<summary>Click to reveal answer</summary>

\textbf{Answer: B) DynamoDB with Time-To-Live (TTL)}

DynamoDB TTL automatically deletes expired items at no additional cost, making it ideal for session data. ElastiCache could work but requires configuration and memory management. RDS and S3 are not optimized for high-frequency session data with automatic expiration.
</details>

---

\subsubsection{Question 31}

\textbf{Which service would you use to create a visual workflow that coordinates multiple Lambda functions with error handling and retry logic?}

A) Amazon SQS
B) AWS Step Functions
C) Amazon EventBridge
D) AWS Batch

<details>
<summary>Click to reveal answer</summary>

\textbf{Answer: B) AWS Step Functions}

Step Functions provides visual workflow orchestration with built-in error handling, retries, and state management. It's specifically designed to coordinate multiple AWS services including Lambda. SQS is just a message queue, EventBridge routes events, and Batch is for batch computing.
</details>

---

\subsubsection{Question 32}

\textbf{A company wants to analyze application logs stored in S3 using SQL without loading data into a database. Which service should they use?}

A) Amazon Redshift
B) Amazon Athena
C) Amazon EMR
D) AWS Glue

<details>
<summary>Click to reveal answer</summary>

\textbf{Answer: B) Amazon Athena}

Athena allows direct SQL queries against data in S3 without loading or transformation. You pay only for queries run (\$5 per TB scanned). Redshift requires loading data, EMR requires cluster management, and Glue is for ETL (though it works well with Athena).
</details>

---

\subsubsection{Question 33}

\textbf{Which AWS service provides recommendations for cost optimization, security, performance, and fault tolerance?}

A) AWS CloudTrail
B) AWS Config
C) AWS Trusted Advisor
D) AWS Inspector

<details>
<summary>Click to reveal answer</summary>

\textbf{Answer: C) AWS Trusted Advisor}

Trusted Advisor provides real-time best practice recommendations across five categories: cost optimization, performance, security, fault tolerance, and service limits. CloudTrail logs API calls, Config tracks configurations, and Inspector assesses security vulnerabilities.
</details>

---

\subsubsection{Question 34}

\textbf{Your application experiences a 10x traffic increase during a 2-hour window every day. Which compute solution provides automatic scaling with minimal management?}

A) EC2 with manual scaling
B) EC2 with Auto Scaling
C) Lambda with CloudWatch Events
D) Lightsail

<details>
<summary>Click to reveal answer</summary>

\textbf{Answer: C) Lambda with CloudWatch Events}

Lambda automatically scales from zero to thousands of concurrent executions with no configuration needed. While EC2 Auto Scaling works, it takes minutes to provision new instances. Lambda scales instantly and you pay only for the 2-hour peak period. CloudWatch Events can trigger scheduled scaling if needed.
</details>

---

\subsubsection{Question 35}

\textbf{Which database service provides automatic scaling of both storage and compute capacity?}

A) Amazon RDS
B) Amazon Aurora Serverless
C) Amazon DynamoDB
D) Amazon Redshift

<details>
<summary>Click to reveal answer</summary>

\textbf{Answer: B) Amazon Aurora Serverless}

Aurora Serverless automatically adjusts database capacity based on application needs, scaling both compute (Aurora Capacity Units) and storage. DynamoDB can auto-scale throughput but not in the same way. RDS requires manual instance resizing for compute, though storage can auto-scale. Redshift requires manual cluster resizing.
</details>

---

\subsubsection{Question 36}

\textbf{A company needs to connect their VPC to an on-premises data center with a consistent, private connection at 10 Gbps. Which service should they use?}

A) Site-to-Site VPN
B) AWS Direct Connect
C) VPC Peering
D) Transit Gateway

<details>
<summary>Click to reveal answer</summary>

\textbf{Answer: B) AWS Direct Connect}

Direct Connect provides dedicated private network connectivity at speeds from 1 Gbps to 100 Gbps with consistent performance. Site-to-Site VPN goes over the internet with variable performance, VPC Peering connects VPCs (not on-premises), and Transit Gateway connects networks but doesn't provide the physical connection.
</details>

---

\subsubsection{Question 37}

\textbf{Which S3 storage class automatically moves objects between access tiers based on changing access patterns?}

A) S3 Standard
B) S3 Intelligent-Tiering
C) S3 Standard-IA
D) S3 Glacier

<details>
<summary>Click to reveal answer</summary>

\textbf{Answer: B) S3 Intelligent-Tiering}

Intelligent-Tiering automatically moves objects between Frequent Access and Infrequent Access tiers based on access patterns, optimizing costs without performance impact or operational overhead. Other storage classes require manual management or lifecycle policies.
</details>

---

\subsubsection{Question 38}

\textbf{Your application needs shared file storage accessible by multiple EC2 instances across different Availability Zones. Which service should you use?}

A) Amazon EBS
B) Amazon EFS
C) Instance Store
D) Amazon S3

<details>
<summary>Click to reveal answer</summary>

\textbf{Answer: B) Amazon EFS}

EFS provides shared NFS file storage accessible by multiple EC2 instances simultaneously across AZs. EBS can only attach to one instance at a time, Instance Store is ephemeral and instance-specific, and S3 is object storage (not a file system).
</details>

---

\subsubsection{Question 39}

\textbf{Which service would you use to track WHO made WHAT change to AWS resources and WHEN?}

A) Amazon CloudWatch
B) AWS CloudTrail
C) AWS Config
D) AWS X-Ray

<details>
<summary>Click to reveal answer</summary>

\textbf{Answer: B) AWS CloudTrail}

CloudTrail logs all API calls (who, what, when, where) for governance, compliance, and auditing. CloudWatch monitors metrics and logs, Config tracks resource configurations over time, and X-Ray traces application requests.
</details>

---

\subsubsection{Question 40}

\textbf{A company needs to run Docker containers without managing EC2 instances. Which service combination should they use?}

A) ECS with EC2 launch type
B) ECS with Fargate launch type
C) EKS with EC2 nodes
D) EC2 with Docker installed

<details>
<summary>Click to reveal answer</summary>

\textbf{Answer: B) ECS with Fargate launch type}

ECS with Fargate is serverless container orchestration where AWS manages all infrastructure. You define containers and Fargate handles provisioning, scaling, and management. ECS with EC2 and EKS with EC2 require managing instances.
</details>

---

\subsubsection{Question 41}

\textbf{Which database is optimized for storing and querying time-series data from IoT devices?}

A) Amazon RDS
B) Amazon DynamoDB
C) Amazon Timestream
D) Amazon Redshift

<details>
<summary>Click to reveal answer</summary>

\textbf{Answer: C) Amazon Timestream}

Timestream is purpose-built for time-series data with automatic data lifecycle management and built-in time-series analytics functions. While DynamoDB can store time-series data, Timestream is optimized for this use case with better performance and cost efficiency.
</details>

---

\subsubsection{Question 42}

\textbf{Your application needs to send notifications to thousands of subscribers via multiple protocols (email, SMS, mobile push). Which service should you use?}

A) Amazon SQS
B) Amazon SNS
C) Amazon EventBridge
D) Amazon SES

<details>
<summary>Click to reveal answer</summary>

\textbf{Answer: B) Amazon SNS}

SNS (Simple Notification Service) is a pub/sub messaging service that can send notifications to multiple subscribers across different protocols (email, SMS, HTTP, mobile push, SQS, Lambda). SQS is for queuing, EventBridge for event routing, and SES is only for email.
</details>

---

\subsubsection{Question 43}

\textbf{Which service helps you discover and catalog data sources to prepare for migration to AWS?}

A) AWS Migration Hub
B) AWS Application Discovery Service
C) AWS Database Migration Service
D) AWS DataSync

<details>
<summary>Click to reveal answer</summary>

\textbf{Answer: B) AWS Application Discovery Service}

Application Discovery Service collects information about on-premises servers including configuration, performance, and dependencies to plan migrations. Migration Hub tracks progress, DMS migrates databases, and DataSync transfers data.
</details>

---

\subsubsection{Question 44}

\textbf{A company needs to ensure their CloudFormation templates follow organizational standards and prevent non-compliant resources from being created. Which service should they use?}

A) AWS Config
B) AWS CloudTrail
C) AWS Control Tower
D) AWS Organizations

<details>
<summary>Click to reveal answer</summary>

\textbf{Answer: C) AWS Control Tower}

Control Tower provides guardrails (preventive and detective) to enforce compliance across accounts. Preventive guardrails use SCPs to prevent non-compliant actions. Config detects but doesn't prevent, CloudTrail logs actions, and Organizations provides account management but not guardrails.
</details>

---

\subsubsection{Question 45}

\textbf{Which ElastiCache engine supports complex data structures like sorted sets, lists, and pub/sub messaging?}

A) Memcached
B) Redis
C) Both support these features equally
D) Neither supports these features

<details>
<summary>Click to reveal answer</summary>

\textbf{Answer: B) Redis}

Redis supports advanced data structures (strings, hashes, lists, sets, sorted sets), persistence, replication, and pub/sub messaging. Memcached is simpler and only supports simple key-value caching with multi-threading. For advanced use cases, Redis is the better choice.
</details>

---

\subsection{Summary}


Domain 3: Cloud Technology and Services represents the largest portion (34\%) of the AWS Certified Cloud Practitioner exam. This domain covers:

\textbf{Key Areas}:
\begin{itemize}
  \item \textbf{Global Infrastructure}: Regions, AZs, Edge Locations, and specialized infrastructure
  \item \textbf{Compute}: From full control (EC2) to fully serverless (Lambda)
  \item \textbf{Storage}: Object (S3), block (EBS), file (EFS), and hybrid (Storage Gateway)
  \item \textbf{Databases}: Relational, NoSQL, caching, analytics, and specialized databases
  \item \textbf{Networking}: VPC, load balancing, content delivery, and connectivity
  \item \textbf{Management}: IaC, monitoring, logging, governance, and operations
  \item \textbf{Additional Services}: Messaging, analytics, ML/AI, and migration tools
\end{itemize}


\textbf{Study Tips}:
\begin{enumerate}
  \item Understand the use cases for each service
  \item Know when to choose one service over another
  \item Remember pricing models, especially for EC2 and S3
  \item Understand high availability and fault tolerance patterns
  \item Know the differences between similar services (e.g., SNS vs SQS, RDS vs DynamoDB)
\end{enumerate}


---

\href{03-security-compliance.md}{← Previous: Security and Compliance} | \href{05-billing-support.md}{Next: Billing, Pricing, and Support →}
